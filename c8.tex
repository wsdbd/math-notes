\section{函数的极限和连续性}
\subsection{函数的极限}

\subsubsection{函数的极限}
\paragraph{}
有函数$f(x)$,函数上有一个点$a$, 如果对任何的$\varepsilon > 0$,存在一个数$\delta > 0$, 使得任何的$x$, 且$0 < |x -a| < \delta$,满足
$$
|f(x) - A| < \varepsilon
$$
那么,就说当$x \to a$时,函数$f(x)$的极限为$A$.  将函数的极限表示为
$$
\lim_{x\to a}f(x) = A
$$

\subsubsection{函数极限的代数运算}
\paragraph{}
设有函数$f(x), g(x)$,且$\lim_{x\to a} f(x) = A$, $\lim_{x\to a} g(x) = B$
\begin{enumerate}
\item $\lim_{x\to a}(f+ g)(x) = A + B$
\item $\lim_{x\to a}(f \cdot g)(x) = A \cdot B$
\item $\lim_{x\to a}(f / g)(x) = A / B$, 如果$B \neq 0$, 且$g(x) \neq 0$
\end{enumerate}

\subsubsection{函数极限不等式}
\paragraph{}
设有函数$f(x), g(x)$,且$\lim_{x\to a} f(x) = A$, $\lim_{x\to a} g(x) = B, $
\begin{enumerate}
\item 如果$A < B$,那么在点$a$的一个去心领域中,其中的每个点$f(x) < g(x)$
\item 如果有函数$h(x)$,并且$f(x) \leq h(x) \leq g(x)$,并且如果$\lim_{x\to a} f(x) = \lim_{x\to a} g(x) = A$,那么$\lim_{x\to a} h(x) = A$.
\item 如果在$a$的某个去心领域中,$f(x) > g(x)$那么$A \geq B$, 如果$f(x) \geq g(x)$,那么$A \geq B$.
\item 如果在$a$的某个去心领域中,$f(x) > B$那么$A \geq B$, 如果$f(x) \geq B$,那么$A \geq B$.

\end{enumerate}


\subsubsection{函数极限存在的判断}
\paragraph{}
\textbf{a}. 函数$f(x)$,当$x \to a$存在极限$A$的充分必要条件是,对任何收敛于$a$的数列$\{ x_n \}, x_n \in E\\a$,数列$\{f(x_n)\}$收敛,且收敛于$A$.

\paragraph{}
\textbf{b}. 单调递增函数$f(x)$,若有极限点$s = \sup E$,那么$x \to s$时,函数有极限的充分必要条件是它上有界. 对极限点$i = \inf E$,那么当$x \to i$时,函数有极限的充分必要条件是它下有界.


\subsubsection{函数极限的例子}

\paragraph{}
\textbf{1}. $\lim_{x\to 0} \frac{\sin x}{x} = 1$

\begin{figure}[h]
\centering
\ includegraphics[scale=0.3]{draw-08-01.png}
\caption{半径为$1$的单位圆}
\end{figure}

\paragraph{}
证明\:  如图所示,一个半径为$1$的单位圆,那么$\overline{0C} = \cos{x}$, $\overline{BC} = \sin{x}$, $\overline{OA} = \overline{OB} = 1$, $\stackrel\frown{AB} = x$, $\stackrel\frown{CD} = x \cos{x}$,  扇形$\sphericalangle{OCD}$,扇形$\sphericalangle{OAB}$,三角形$\bigtriangleup{OAB}$的面积分别为
\begin{align*}
S_{\sphericalangle{OCD}} & = \frac{1}{2} x \cos{x}^2  \\
S_{\sphericalangle{OAB}} & = \frac{1}{2}  x \\
S_{\bigtriangleup{OAB}} & = \frac{1}{2} \sin{x}
\end{align*}
因为$ S_{\sphericalangle{OCD}} < S_{\bigtriangleup{OAB}} < S_{\sphericalangle{OAB}}$,所以
$$
\frac{1}{2} x \cos{x}^2 <\frac{1}{2} \sin{x} <    \frac{1}{2} x
$$
这里只考虑$x > 0$的情况, $x < 0$的情况可以类似的证明,也就是说
$$
\cos{x}^2 < \frac{\sin{x}}{x} < 1
$$
因为$\lim_{x\to 0} cos{x}^2 = 1$,而且$\lim_{x \to 0} 1 = 1$,根据极限不等式,那么$\lim_{x\to 0} \frac{\sin{x}}{x} = 1$


\paragraph{}
\textbf{2}.  $\lim_{x \to \infty} (1 + \frac{1}{x})^x = e$

\paragraph{}
证明\: 在(7.1.6小节)里将数$e$定义为
$$
\lim_{n \to \infty} (1 + \frac{1}{n})^n
$$
这里尝试推广到任意的实数$x$而不是正整数.  为了方便,使用程序语言常用的$floor(x)$代表对数$x$向下取整,因此有不等式
$$
(1 + \frac{1}{floor(x) + 1})^{floor(x)} < (1 + \frac{1}{x})^x < (1 + \frac{1}{floor(x)})^{floor(x) + 1}
$$
这里因为$\frac{1}{floor(x) + 1} < \frac{1}{x}$, $floor(x) \leq x$, $\frac{1}{floor(x)} \geq \frac{1}{x}$, $floor(x) + 1 > x$. 并且我们知道
$$
\lim_{n\to \infty} (1 + \frac{1}{n + 1})^{n} = \lim_{n\to \infty} \frac{(1 + \frac{1}{n+1})^{n + 1}}{(1 + \frac{1}{n+1})} = \frac{ \lim_{n\to \infty} (1 + \frac{1}{n+1})^{n + 1}}{ \lim_{n\to \infty} (1 + \frac{1}{n+1}) } = \frac{e}{1} = e
$$
而
$$
\lim_{n\to \infty}  (1 + \frac{1}{n})^{n + 1} = \lim_{n\to \infty} (1 + \frac{1}{n})^{n} \cdot  \lim_{n\to \infty} (1 + \frac{1}{n}) = e \cdot 1 = e
$$
因此由极限的不等式可知$e \leq \lim_{x \to \infty} (1 + \frac{1}{x})^x \leq e$,
所以
$$
\lim_{n \to \infty} (1 + \frac{1}{n})^n
$$


\subsubsection{大O和小o}
\paragraph{}
如果对于函数$f(x)$和$g(x)$,如果
$$
\lim_{x\to a} \frac{f(x)}{g(x)} = 1
$$
则称当$x \to a$时,$f(x)$和$g(x)$渐近等价,记为$f(x) \sim g(x)$

\paragraph{}
如果对于函数$f(x)$和$g(x)$,如果存在$a$的一个领域,使得
$$
|\frac{f(x)}{g(x)}| \leq K
$$
$K$为任意的实数,则记为$f(x) = O(g(x))$, 或者说如果$\lim_{x\to a}\frac{f(x)}{g(x)}$存在则$f(x) = O(g(x))$

\paragraph{}
如果对于函数$f(x)$和$g(x)$,$\lim_{x\to a}\frac{f(x)}{g(x)} = 0$,则记为$f(x)= o(g(x))$.

\paragraph{}
大$O$和小$o$在求极限的时候很有用,今后将在函数的级数展开中再次讨论.

\subsubsection{左右极限}
\paragraph{}
对于函数$f(x)$,在定义域$E$中的某一点$x_0$,如果对于定义域中任何点$x_{0^-} < x_0$,由这些点组成的数列${x_{0^-}}$收敛于$a$,称这样的极限为$x_0$处的\textbf{左极限}, 记为$\lim_{x\to x_{0^-}} = a$,同样的定义$x_{0^+} > x_0$的数列组成的极限为\textbf{右极限},记为$\lim_{x\to x_{0^+}} = b$. 如果$\lim_{x\to x_{0^-}} = \lim_{x\to x_{0^+}}$,则函数在$x_0$处存在极限,且它们的值相同.


\subsection{连续性}

\subsubsection{连续}
\paragraph{}
定义函数$f(x)$在定义域$E$中一点$x_0$,如果
$$
\lim_{x \to x_0} f(x) = f(x_0)
$$
那么称函数$f(x)$在$x_0$这个点\textbf{连续}.

\paragraph{}
如果函数$f(x)$在定义域中$E$处处连续,称$f(x)$为集合$E$的\textbf{连续函数}.

\subsubsection{间断点}
\paragraph{}
如果函数$f(x)$在$x_0$处不连续,称$x_0$为函数$f(x)$的\textbf{间断点},  如果函数$f(x)$在$x_0$处$\lim_{x \to x_{0^-}} f(x) \neq f(x_0)$或者$\lim_{x \to x_{0^+}} f(x) \neq f(x_0)$那么称这样的间断点为\textbf{第一类间断点},也称为可去间断点.  如果$f(x)$在$x_0$处的左右极限有一个不存在,那么称之为\textbf{第二类间断点}.

\subsubsection{连续函数的性质}
\paragraph{}
\textbf{a}. 如果函数$f(x)$在集合$E$(定义域)的连续函数,那么函数在某个点$a$的邻域中有界. 

\paragraph{}
\textbf{b}. \textbf{波尔查诺定理}\, 若函数$f(x)$在闭区间$[a, b]$内连续的,如果$f(a) \cdot f(b) < 0$,那么存在一个点$c$使得$f(x) = 0$. 


\paragraph{}
\textbf{c}. \textbf{介值定理}\, 也称为波尔察诺-柯西第二定理 如果连续函数$f(x)$在点$a,b$中取值为$f(a) = A, f(b) = B, A < B$, 那么对任意的$C, A < C < B$,在$a, b$之间必然存在一点$c$,使得$f(c) = C$. 

\paragraph{}
\textbf{d}.  \textbf{魏尔斯特拉最大值定理}\, 函数$f(x)$在闭区间$[a, b]$中连续,则函数$f(x)$必有界.

\subsubsection{一致连续}
\paragraph{}
若$f(x)$在定义域$E$中连续,对于任何一个数$\varepsilon > 0$,存在一个数$\delta > 0$,使得对于$E$中的任何一个点$x_0$满足
$$
|x - x_0| < \delta, |f(x) - f(x_0)| < \varepsilon
$$
称函数$f(x)$在$E$中\textbf{一致连续},一致连续意味着$x$的变化足够小时,$f(x)$的变化也在足够小的范围内. 

\paragraph{}
若函数$f(x)$在闭区间$[a, b]$内连续,则它在这个区间内\textbf{一致连续}.




