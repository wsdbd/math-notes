\section{概率}

\subsection{样本空间和事件}
\paragraph{}
\textbf{定义} 将一个试验的所有的可能的结果组成的集合,称为该试验的\textbf{样本空间}. 样子空间的子集称之为\textbf{事件}. 空子集称为一个\textbf{不可能事件}. 如果两个事件$E$和$F$,并且$E\cap F= \emptyset$,那么称事件$E$和$F$是\textbf{互不相容的}

\paragraph{}
例如扔一次骰子,那么样本空间就是$S = \{1, 2, 3, 4, 5, 6\}$,那么将扔出3点,或者扔出大于3点,都称之为事件,扔出7则为一个不可能事件,扔出2和扔出3为互不相容的事件.

\subsection{概率}

\paragraph{}
我们将样本空间$S$中的事件$E$发生的可能性记为$P(E)$

\subsubsection{古典概率}
\paragraph{}
\textbf{古典概率}也称为\textbf{拉普拉斯概率},古典概率的概念是现代概率论的特殊的例子,如果一个试验所有的单位事件数量有限,并且是\textbf{等可能性}的,那么事件$E$在事件空间$S$中概率
$$
P(E) =  \frac{|E|}{|S|}
$$


\subsubsection{统计概率}
\paragraph{}
\textbf{统计概率}是指,如果一个试验的某个事件,重复无数次,该事件出现频率存在一个极限值,这个值就是事件的概率,也就是
$$
P(E) = \lim_{n\to \infty } f_n(E)
$$

\subsubsection{现代概率论}
\paragraph{}
现在概率论指的建立公理化系统的概率论,对于样本空间$S$,事件$E$的概率记为$P(E)$,那么
\begin{enumerate}
\item $ 0 \leq P(E) \leq 1$
\item $P(S) = 1$
\item 对互不相容的事件$E_1, E_2, \cdots$,那么
$$
P\left(\cup_{i=1}^\infty E_i\right) = \sum_{i=1}^\infty P(E_i)
$$
\end{enumerate}




