\section{排列组合}

\paragraph{}
一个集合$S$的一个划分是指这个集合的子集$S_1, S_2, \cdots, S_n$,满足
$S = S_1 \cup S_2 \cup \cdots \cup S_n$,并且$S_i \cap S_j = \emptyset (i \neq j)$. 

\paragraph{}
另外将集合$S$的元素的数量记为$|S|$.

\subsection{计数}

\subsubsection{加法}
\paragraph{}
如果一个集合的$S$的一个划分$S_1, S_2, \cdots, S_n$,那么
$$
|S| = |S_1|+|S_2| + \cdots + |S_n|
$$
例如$S =  \{1, 2, 3, 4, 5, 6\}$,那么$|S| = |\{1\}| + |\{2, 3\}| + |\{4\}| + |\{5\}| + |\{6\}| = 1 + 2 + 1  + 1 + 1  = 6$

\subsubsection{乘法}
\paragraph{}
设$S = \{(a, b) | a \in A, b \in B\}$,并且$|A| = p, |B| = q$,则
$$
|S| = p \times q
$$

\subsubsection{减法}
设集合$A \subset S$,那么集合A的\textbf{补}记为$\overline{A}$,有时候也记为$A^c$,且$\overline{A} = S \setminus A = \{x \in S: x \notin A\} $,那么
$$
|A| = |S| - |\overline{A}|
$$

\subsubsection{除法}
设集合$S$,将$S$划分成$k$个相同数量的子集,那么,那个子集的数量为
$|S_i| =  |S| / k$, 当然前提是存在这样的划分.

\subsection{排列}

\paragraph{}
一个元素个数为$n$的集合$S$的\textbf{r排列},其中$ r \leq n $,表示为将集合$ S $中的元素组成个数为$ r $的有序对组成的集合, 例如$S = \{a, b, c\}$,那么$S$的2排列为
$$
 \{(a, b), (a, c), (b, a), (b, c), (c, a), (c, b)\}
$$
将这个r排列的个数记为$P(n , r)$,也有时候记为$A_n^r$

\paragraph{}
\textbf{定理} 对于正整数$n, r, r \leq n$,那么
$$
P(n, r) = n \times (n - 1) \times \cdots \times (n - r + 1)
$$



\subsection{组合}
\paragraph{}
如果说排列是集合$S$的有序选择,那么组合就是集合的无序选择,对于排列的r排列,那么\textbf{r组合}就表示集合$S$中元素数量为$r$的子集的集合,例如$S = \{a, b, c\}$,那么$S$的2组合为
$$
\{ \{a, b\}, \{a, c\}, \{b,c\} \}
$$
一般将r组合的数量记为${n \choose r }$,也有记为$C_n^r$

\paragraph{}
\textbf{定理} 对于正整数$n, r, r \leq n$,那么
$$
{n \choose r} = \frac{n!}{r!(n-r)!}
$$

\subsubsection{二项式系数}
\paragraph{}
如果用组合的思想,展开一个$(x + y)^n = (x+y)(x+y)\cdots(x+y)$,  

