\section{向量空间}

\subsection{向量空间的定义}
把一个$R^{m\times n}$的元素组成的集合,以及对这个集合定义了加法和标量乘法,称之为\textbf{向量空间}. 向量空间的元素一般用小写加粗的字母表示,一般而言,向量是一个$m\times 1$或者$1 \times n$的矩阵,但是一个$m\times n$的矩阵同样可以表示为一个行向量或者列向量,所以说,在这里为了方便也将矩阵看成是一个向量.  在(2.2节)里提到矩阵的代数运算,这里用向量的方式再列一次. 其本质上是一样,矩阵是其中的特例. 在(1.6节)里提到欧氏空间,可以看成是$R^n$中的向量空间. 
\newline
\textbf{加法}\:
\begin{enumerate}
\item 设有两个向量$\textbf{u}, \textbf{v}$,那么$\textbf{u} + \textbf{v} = \textbf{v} + \textbf{u}$, 即满足交换率
\item 设有向量$\textbf{u}, \textbf{v}, \textbf{w}$,那么$\textbf{u} + (\textbf{v} + \textbf{w}) = (\textbf{u} + \textbf{v}) + \textbf{w}$, 即分配率
\item 存在一个向量$\textbf{0}$,即加法的单位元, $\textbf{u} + \textbf{0} = \textbf{u}$
\item 存在加法的逆元,即$\textbf{-u}$为向量$\textbf{u}$的逆元
\end{enumerate}
\textbf{标量乘法}\:
\begin{enumerate}
\item 标量1为乘法的单位元,$1 \cdot \textbf{u} = \textbf{u}$
\item $(\alpha \beta)\textbf{u} = \alpha (\beta \textbf{u})$
\item $\alpha \textbf{u} =\textbf{u}  \alpha$
\end{enumerate}
\textbf{分配率}\:
\begin{enumerate}
\item $\alpha(\textbf{u} + \textbf{v}) = \alpha \textbf{u} + \alpha \textbf{v}$
\item $(\alpha + \beta)\textbf{u} = \alpha \textbf{u} + \beta \textbf{u}$
\end{enumerate}


\subsection{子空间}
\paragraph{}
如果集合$S$是向量空间$V$的一个子集,并且对加法和标量乘法封闭
\begin{enumerate}
\item 若$\textbf{x} \in S$,那么$\alpha \textbf{x} \in S$
\item 若$\textbf{x} \in S, \textbf{y} \in S$,那么$\textbf{x} + \textbf{y} \in S$
\end{enumerate}
将$S$称之为$V$的\textbf{子空间}. 如果一个空间只有一个向量并且这个向量是$\textbf{0}$,称之为\textbf{零子空间}

\paragraph{}
对于一个$m\times n$的齐次方程组$A\textbf{x} = 0$,解的集合组成的空间称为\textbf{零空间}, 记为$N(A)$.  这个和\textbf{零子空间}不一样. 但是例如
$$
A = \begin{bmatrix}
    1 & 1 \\
    1 & -1
  \end{bmatrix} 
$$
$N(A)$就是一个零空间,解集只有0,而
$$
A = \begin{bmatrix}
    1 & 1 
  \end{bmatrix} 
$$
的解集就形如
$$
\{(0, 0)^T, (1, -1)^T, (2, -2)^T, (-1, 1)^T, \cdots\}
$$

\subsection{基和维数}

\subsubsection{线性生成空间}
\paragraph{}
设向量空间$V$中某些元素组成的集合$S = \{\textbf{v}_1, \textbf{v}_2, \cdots, \textbf{v}_n\}, S \subset V$,将它们的线性组合所组成的集合$W$,显然满足向量空间的定义,称之为\textbf{线性生成空间},将$S$称为$W$的\textbf{生成集合}. 记为$W = Span(S) := \{\alpha_1 \textbf{v}_1 + \alpha_2 \textbf{v}_2 + \cdots\}$, 如果是集合$S$的生成空间为$V$,那么$S$也称为$V$的\textbf{生成集合}.

\paragraph{}
例如
$$
R^3 = Span(\{ (1, 0, 0)^T, (0, 1, 0)^T, (0, 0, 1)^T \}), R^3 = Span(\{(1, 0, 0)^T, (0, 1, 0)^T, (0, 0, 1)^T, (2, 4, 6)^T\})
$$

\subsubsection{线性无关}
\paragraph{}
对于向量空间$V$中的向量$\textbf{v}_1, \textbf{v}_2, \cdots, \textbf{v}_n$,如果标量$a_1, a_2, \cdots, a_n$全为0的时候,下面的等式才满足
$$
a_1\textbf{v}_1 + a_2\textbf{v}_2 + \cdots + a_n\textbf{v}_n = 0
$$
将$\textbf{v}_1, \textbf{v}_2, \cdots, \textbf{v}_n$是\textbf{线性无关}的,相反的只要存在非零的$a_k$,那么称它们是\textbf{线性相关}的.

\paragraph{}
例如
$$
\textbf{e}_1 = (1, 0, 0)^T, \textbf{e}_2 = (0, 1, 0)^T, \textbf{e}_3 = (0, 0, 1), \textbf{v}_1 = (1, 1, 0)^T
$$
显然,$\textbf{e}_1$和$\textbf{e}_2$是线性无关的, $\textbf{e}_1, \textbf{e}_2, \textbf{e}_3$也是线性无关的,$\textbf{e}_1, \textbf{v}_1$是线性无关的, $\textbf{e}_1, \textbf{e}_2, \textbf{v}_1$是线性相关的.

\paragraph{}
特别的若是一个$n\times n$的方阵$A$, 如果一组向量组成$A$,如果$det(A) = 0$,那么可以说明,这些向量是线性相关的,如果$det(A) \neq 0$那么这些向量是线性无关的.


\subsubsection{基和维数}
\paragraph{}
如果一组向量$\textbf{v}_1, \textbf{v}_2, \cdots, \textbf{v}_n$,它们是线性无关的,并且生成向量空间$V$. 那么称这些向量为$V$的一组\textbf{基}, 这些向量就是$V$的最小的生成集. 把形如$\textbf{e}_1, \textbf{e}_2, \cdots, \textbf{e}_n$称为\textbf{标准基}.  

\paragraph{}
把一个向量空间$V$的一组基的个数称为向量空间的$维数$记为$\dim V$, 零子空间的维数为0.  对于矩阵而言, 矩阵的维数就是矩阵的\textbf{秩}, 记为$rank(A)$,  矩阵秩有个简单的性质$rank(A) = rand(A^T)$.  维数从几何的观点看,可以看成一组基可以生成一个几维的空间.  维数为2就是我们所说的平面,维数为3就是三维的空间.  

\subsubsection{坐标变换}
\paragraph{}
坐标的概念大家应该比较熟悉,用向量空间的语言来说,例如二维的笛卡尔坐标系,就相当于用二维的标准基$\textbf{e}_1 = (1, 0)^T, \textbf{e}_2 = (0, 1)^T$为基,对于二维空间$R^2$里的任何一个向量$\textbf{p} = x \textbf{e}_1 + y \textbf{e}_2$,那么$\textbf{p}$的坐标就是$(x, y)$, 这就是笛卡尔坐标系. 那么,如果是以$\textbf{v}_1 = (2, 0), \textbf{v}_2 = (0, 2)$为基的$R^2$的向量空间, 如果有一个点$\textbf{p} = (1, 1)^T$是以$\textbf{e}_1, \textbf{e}_2$为基下的坐标,那么在$\textbf{v}_1, \textbf{v}_2$下,$\textbf{p} = 1/2 \textbf{v}_1 + 1/2 \textbf{v}_2$,也就是$(1/2, 1/2)$是$\textbf{p}$在$\textbf{v}_1, \textbf{v}_2$为基下的\textbf{坐标}.

\paragraph{}
接下来考察坐标在两组基之间的变换,有一组基$\textbf{u} = \{ (\textbf{u}_1, \textbf{u}_2, \cdots, \textbf{u}_n) \}$, 其中的第$i$个元素用标准基来表示,那么$\textbf{u}_i = c_{i1}\textbf{e}_1 + c_{i2}\textbf{e}_2 + \cdots + c_{in}\textbf{e}_n$, 同样的有一组基$\textbf{v} = \{ (\textbf{v}_1, \textbf{v}_2, \cdots, \textbf{v}_n) \}$, 第$i$个元素用标准基表示,那么$\textbf{v}_i = c_{i1}' \textbf{e}_1 + c_{i2}' \textbf{e}_2 + \cdots c_{in}' \textbf{e}_n$, 如果有一个向量$\textbf{p}$在基$U$下的坐标为$(x_1, x_2, \cdots, x_n)$,那么在基为$V$下的坐标为$(x_1', x_2', \cdots, x_n')$
因此,把它们都转换成标准基下的坐标,分别是
\begin{align*}
x_{i(e)} & = (x_1, x_2, \cdots, x_n) \cdot  \begin{bmatrix}
    c_{11} & c_{21} & \cdots & c_{n1} \\
    c_{12} & c_{22} & \cdots & c_{n2} \\
    & & \vdots & \\
    c_{1n} & c_{2n} & \cdots & c_{nn} 
  \end{bmatrix}   \\
x_{i(e)}' & =  (x_1', x_2', \cdots, x_n') \cdot  \begin{bmatrix}
    c_{11}' & c_{21}' & \cdots & c_{n1}' \\
    c_{12}' & c_{22}' & \cdots & c_{n2}' \\
    & & \vdots & \\
    c_{1n}' & c_{2n}' & \cdots & c_{nn}' 
  \end{bmatrix}
\end{align*}
分别将各自后边的矩阵记为$U, V$,那么$\textbf{x}^T U = \textbf{x}'^T V$, 其中
\begin{align*}
U & = \begin{bmatrix}
    c_{11} & c_{21} & \cdots & c_{n1} \\
    c_{12} & c_{22} & \cdots & c_{n2} \\
    & & \vdots & \\
    c_{1n} & c_{2n} & \cdots & c_{nn} 
  \end{bmatrix} \\
 V  & = \begin{bmatrix}
    c_{11}' & c_{21}' & \cdots & c_{n1}' \\
    c_{12}' & c_{22}' & \cdots & c_{n2}' \\
    & & \vdots & \\
    c_{1n}' & c_{2n}' & \cdots & c_{nn}' 
  \end{bmatrix}
\end{align*}
因此说,$\textbf{x}' = \textbf{x}^T U V^{-1}$.  这里将$U V^{-1}$将之为\textbf{基变换矩阵}. 

\paragraph{}
例如,两组基$ \textbf{u} = \{ (1, 0)^T, (0, 1)^T \}, \textbf{v} = \{ (2, 0)^T, (0, 2)^T \} $, 在$\textbf{u}$基中某个元素的坐标为$(1, 2)$,那么
$$
U = \begin{bmatrix}
   1 & 0 \\
   0 & 1  
  \end{bmatrix}, V = \begin{bmatrix}
   2 & 0 \\
   0 & 2 
  \end{bmatrix}
$$
则
$$
U V^{-1} =  \begin{bmatrix}
   1 & 0 \\
   0 & 1  
  \end{bmatrix} \cdot  \begin{bmatrix}
   1/2 & 0 \\
   0 & 1/2
  \end{bmatrix} = \begin{bmatrix}
   1/2 & 0 \\
   0 & 1/2
  \end{bmatrix}
$$
因此,新的坐标为
$$
(x, y) = (1, 2) \cdot \begin{bmatrix}
   1/2 & 0 \\
   0 & 1/2
  \end{bmatrix} = (1/2, 1)
$$
复杂一点的,如果$ \textbf{u} = \{ (1, 0)^T, (0, 1)^T \}, \textbf{v} = \{ (1, 0)^T, (1, 1)^T \} $那么
$$
U V^{-1} =  \begin{bmatrix}
   1 & 0 \\
   0 & 1  
  \end{bmatrix} \cdot  \begin{bmatrix}
  1  & -1 \\
   0 & 1
  \end{bmatrix} = \begin{bmatrix}
   1 & -1 \\
   0 & 1
  \end{bmatrix}
$$
因此,新的坐标为
$$
(x, y) = (1, 2) \cdot \begin{bmatrix}
   1 & -1 \\
   0 & 1
  \end{bmatrix} = (1, 1)
$$

\subsection{直和}
\paragraph{}
对于向量空间$V$,若其子空间$W_1, W_2, \cdots, W_n$,中各个子空间分别选取一个向量,使$\textbf{v} \in V$可以表示成
$$
\textbf{v} = \textbf{w}_1 + \textbf{w}_2 + \cdots + \textbf{w}_n
$$
则可以将$V$记成$V = W_1 \oplus W_2 \oplus \cdots \oplus W_n$. 称之为\textbf{直和}. 类似于基的概念,$W_1, W_2, \cdots, W_n$是线性无关的. 

