\documentclass[UTF8]{ctexart}
\title{数学笔记}
\author{林建}
\date{\today}
\usepackage{xeCJK}
\usepackage{amsmath}
\usepackage{enumitem} 
\usepackage{amssymb}
\usepackage{bookmark}

\setCJKmainfont{PingFang SC}

\usepackage{titlesec}
\newcommand{\sectionbreak}{\clearpage}

\begin{document}
\maketitle

\tableofcontents


\sectionbreak

\part{集合论}


\section{集合}

\paragraph{}
集合是数学的一个基础概念. 大部分数学分支都可以用集合论的语言来方便的描述. 集合通俗的讲就是一些具有类似性质的事物组成的整体. 

\subsection{符号以及集合的初等运算}

\subsubsection{符号}
\paragraph{}
一般将集合记为$\{...\}$, 如果集全只有一个元素$a$,可以将集合$\{a\}$简单的记为$a$.
\paragraph{}
集合$\{1, 2\}$是无序的,它和$\{2, 1\}$是同一个集合,有时需要用到有序的数对,因此为了方便引入了$(x, y)$来表示有序对,假设两个集合$A$和$B$,则先后从$A$和$B$选取一个元素,产生了一组有序的元素,这些元素组成的集合$M$,例如$a \in A$并且$b \in B$,以及$(a, b) \in M$, 有序对用集合可定义为:
$$
(a, b) := \{\{a\},\{a, b\}\}
$$
通常用符号$:=$来表示定义.

\subsubsection{包含}
\paragraph{}
假设$x$为集合$X$的一个元素,用符号记为:
$$
x \in X
$$, 相反的,如果$y$不是集合$X$的元素,记为$y \notin X$.
假设两个集合$X$和$Y$, 它们的元素相同, 记为:
$$
X = Y
$$
假设集合$A$是集合$B$的\textbf{子集},即集合$A$中的元素都是集合$B$中的元素, 记为
$$
A \subset B
$$
如果$A \subset B$并且$A \neq B$, 则称$A$为$B$的\textbf{真子集}. 另外, 一般将\textbf{空集}记为$\emptyset$

\subsubsection{集合的初等运算}
\paragraph{}
\textbf{并} \:  将集合$A$和$B$的所有元素组成的集合$M$称为$A$和$B$的并集, 记为:
$$
A \cup B := \{x\in M | (x \in A) \lor (x \in B)\}
$$

\paragraph{}
\textbf{交} \:  假设有集合$A$和$B$, 如果元素$x$同时是$A$和$B$的元素, 而这些元素组成的集合$M$称为$A$和$B$的交集, 记为:
$$
A \cap B := \{ x \in M | (x \in A) \land (x \in B) \}
$$

\paragraph{}
\textbf{差} \:  假设有集合$A$和$B$, 如果元素$a$是$A$的元素, 并且$a$不是$B$的元素,这些元素组成的集合$M$称为$A$和$B$的差, 记为:
$$
A - B := \{ x \in M | (x \in A) \land (x \ni B) \}
$$
有时候差也用$A\setminus B$表示
$$
A \setminus B := \{ x \in M | (x \in A) \land (x \ni B) \}
$$

\paragraph{}
\textbf{对称差}\: 假设有集合$A$和$B$, 如果元素$c$只属于$A$或者只属于$B$,定义这样元素组成的集合为$A$和$B$的对称差记为
$$
A\Delta B = (A - B) \cup (B - A)
$$


\paragraph{}
\textbf{补} \:  假设有集合$X$, $A$为它的子集,则$A$在集合$X$中的补集, 表示所有的$x$是$X$中的元素,但不是$A$的元素, 记为$A^c$


\paragraph{}
\textbf{直积(笛卡尔积)} \:  假设两个集合$X$和$Y$, 分别从$X$和$Y$中取一个元素, 组成一个有序对, 称之为集合的直积. 记为:
$$
X \times Y := \{(x, y) | (x \in X) \land (y \in Y) \}
$$
例如,$R$可以表示为数轴上的点,也就是一条直线,面$R \times R$简记为$R^2$,就可以表示一个平面,而$R \times R \times R$简记为$R^3$,就可以表示为一个三维的空间了.


\subsection{公理化集合论}

\paragraph{}
1870年代由康托尔(Georg Cantor)和理察·戴德金(Richard Dedekind)提出的朴素集合论, 而朴素集合论定义并不严格,例如罗素悖论: 由所有不包含集合自身的集合所构成的集合,因此数学家们开始研究更严格的理论.  目前广泛认同的一个公理化的集合论是策梅洛-弗兰克尔集合论(Zermelo-Fraenkel), 并且加上选择公理,这一套公理称为ZFC. 

\begin{enumerate}
\item \textbf{外延公理}  \: 两个集合相同, 当且仅当它们的元素相同
\item \textbf{分类公理}  \: 给出任何的集合$S$以及命题$P(x)$, 存在一个\textbf{子集}包含使命题$P(x)$成立的元素. 即子集存在.
\item \textbf{配对公理} \:  对于集合$X, Y$, 存在一个集合$Z = \{X, Y\}$, 即存在一个集合, 它的元素是集合$X$和$Y$.
\item \textbf{并集公理} \: 对于一个集合的集合$S$, 存在一个并集, 使它的元素是$S$的元素的元素.
\item \textbf{空集公理} \: 存在一个不包含任何元素的集合, 称为\textbf{空集}, 记为$\{\}$.
\item \textbf{无穷公理} \: 存在着集合$S$, 空集$\{\}$为其中的一个元素,且对于任何元素$x$, $x \cup \{x\}$也是其元素.
\item \textbf{替代公理} \: 假设$F(x, y)$是一个命题,它使得集体$X$中的任何的元素$x_0$, 存在一个元素$y_0$, 使得$F(x, y)$成立, 而这些$y$所组成的集合存在.
\item \textbf{幂集公理} \: 假设集合$X$, 则存在着一个集合, 它的元素为$X$的一切子集.
\item \textbf{正规公理} \: 非空集合$X$中存在一个元素$x$, 它与$X$本身的交集为空集. 
\paragraph{}
\item \textbf{选择公理} \: 对于任意多个非空集组成的族, 存在一个集合$C$, 其元素为族中每个集合$X$的一个元素, 即$C \cap X = \{x\}$. 
\end{enumerate}

\subsection{映射}
\paragraph{}
对于集合$X$和$Y$,将某种对应关系$f$使每一个$x \in X$都有一个$y \in Y$与之对应. 这种对应关系称之为\textbf{映射}, \textbf{映射}也称之为\textbf{函数}. 其中$X$将为函数的\textbf{定义域}, $Y$称为函数的\textbf{到达域}. 而
$$
f(X) := \{ y \in Y | (x \in X) \land (y = f(x)) \}
$$
称为函数的\textbf{值域}, 注意到,$Y$是到达域,而且$f(X) \subset Y$. 
另外,也常用这些符号来表示:
$$
f: X \to Y, X \stackrel{f}{\to} Y
$$

\subsubsection{映射的分类}
\paragraph{}
\textbf{单射} \: 指的是将不同的变量映射到不同的值,也就是说,如果$x_1 \neq x_2$则$f(x_1) \neq f(x_2)$

\paragraph{}
\textbf{满射} \: 指的是值域就是到达域,这意味着,是否满射和选取的到达域其实是有关系的.

\paragraph{}
\textbf{双射} \:  即是单射又是满射的映射称为\textbf{双射}, 也称之为\textbf{一一映射}.

\subsubsection{反函数}
\paragraph{}
既然有$f: X \to Y$,那么应该也存在一个集合$Y$到集合$X$的对应. 根据函数的定义,每一个$x$只有一个$y$与之对应,所以反函数不一定存在,考虑仅单射而非满射的情况下,那么必然存在一些元素$y \in Y$,无法映射到$X$之中,而考虑仅满射而非单射的情况下,那么必然存在一些元素$y \in Y$,存在多个$x \in X$, 不满足函数的定义,所以有反函数的充分必要条件是双射.  将函数$f$的反函数记为$f^{-1}$,也称函数$f$\textbf{可逆}.

\subsubsection{函数的复合}
\paragraph{}
定义一种函数的复合运算,若有$f: X \to Y$,$g: Y \to Z$,将函数的复合记为:
$$
f \circ g := f(g(x))
$$

\subsection{实数与复数}

\subsubsection{有序集}
\paragraph{}
定义一种关系,它的符号是$<$, 如果一个集合$S$的元素$x, y$,它至少在下面的其中一个关系式中成立,
$$ 
x < y, \, x = y, \,  y < x
$$
并且,如果$x < y, y < z$其中,$x,y,z \in S$,则$x < z$

\paragraph{}
如果一个集合,定义了一个序的关系,如果是部分元素满足这种关系,称之为\textbf{偏序集}. 如果所有的元素都满足这种关系的,称之为\textbf{全序集}. 例如复数就是一个偏序集,实数就是全序集. 以下为了方便通常把全序集称为有序集.

\paragraph{}
如果$S$是一个有序集,$E \subset S$,如果存在$b \in S$使得任意的$x \in E$,满足$x <= b$,则称$E$有\textbf{上界},$b$为$E$的一个上界,若对于任何的$c \in S$并且$c < b$,$c$不是$E$的上界,则我们称$b$为$E$的上确界,也称之为最小上界. 同样的可以定义\textbf{下界},\textbf{下确界}. 上确界和下确界用符号表示为
\begin{align*}
b &= supE  \\
a &= infE  
\end{align*}



\subsubsection{归纳集}
\paragraph{}
 如果有一个集合$S$,如果$a \in S$,并且$a + 1 \in S$,则称这样的集合为\textbf{归纳集}.  这里将包括$0$的最小归纳集,称为\textbf{自然数},使用符号$N$表示(为了方便,我们将0也视为自然数).  如果我们将$\emptyset$记为0,而$\emptyset \cup \{\emptyset\} = \{\emptyset\}$记为1,2就可以表示成$ 2 := \{0, 1\} = \{\emptyset, \{\emptyset\}\}, \cdots, n + 1 := \{0, 1, \dotsc, n\}$


\subsubsection{域}
\paragraph{}
将定义了\textbf{加法}和\textbf{乘法}的集合将之为\textbf{域}. 

\paragraph{}
\textbf{加法} \: 用符号$+$表示,对于集合$S$, 有$x, y, z \in S$它满足以下的条件
\begin{enumerate}[itemindent=2em]
\item 有0元,称0为单位元
\item $x + y \in S$
\item  $x + y = y + x$,即满足交换率.
\item $x + (y + z) = (x + y) + z$, 满足结合率.
\item $x$有\textbf{逆元}记为$-x$,且$x + (-x) = 0$.
\end{enumerate}

\paragraph{}
\textbf{乘法} \: 用符号$\cdot$表示,对于集合$S$, 有$x, y, z \in S$它满足以下的条件
\begin{enumerate}[itemindent=2em]
\item 有单位元1
\item $x \cdot y \in S$
\item  $x \cdot y = y \cdot x$,即满足交换率.
\item $x \cdot (y \cdot z) = (x \cdot y) \cdot z$, 满足结合率.
\item 对$x \in S$,且$x \neq 0$,有\textbf{逆元}记为$x^{-1}$,且$x \cdot x^{-1} = 1$.
\end{enumerate}
需要说明的是,这里的加法和乘法并非传统意义上的加法和乘法, 其意义是指两种操作或者两种映射,说成加法和减法是为了直观的理解. 

\paragraph{}
同时,对于加法和乘法,又满足分配率,即
$$
x \cdot (y + z) = x \cdot y + x \cdot z
$$

\subsubsection{整数}
\paragraph{}
\textbf{整数}是包含自然数以及它们的负元的集合,用符号$Z$表示. 

\paragraph{}
\textbf{算术基本定理}\: 对于任何$n > 1$的整数,都可以表示成素数的积
$$
n = p_1 \cdots p_i
$$

\subsubsection{实数域}
\paragraph{}
先介绍一下实数集的一个子集\textbf{有理数}. 整数的商$a/b, b \neq 0$称之为\textbf{有理数},用符号$Q$表示. \textbf{实数域}是这样的一个\textbf{域}不仅有加法和乘法,并且实数集是一个\textbf{全序集},另外这个实数集满足\textbf{完备性}即\textbf{最小上界公理},这样的实数集记为$R$.

\paragraph{}
\textbf{最小上界公理}\:  令$A \subset R$,并且$A \neq \emptyset$,如果$A$有上界,则$A$有最小上界.

\paragraph{}
有了最小上界公理,就可以把有理数$Q$扩充为实数$R$,令$S \subset Q$,且$S = \{x| x < \sqrt{2}\}$, 由有理数的定义可以知道$\sqrt{2}$不是有理数,所以集合$S$在有理数内有上界, 但是没有上确界. 也就是说有理数中间是有缝隙, 因此在有理数的基础上扩充把不是有理数的实数称为\textbf{无理数},这就形成了实数集. 

\paragraph{}
顺便,根据有理数的定义,证明为什么$\sqrt{2}$不是有理数,显然$\sqrt{2}$不是整数,令$\sqrt{2} = p/q$,其中$p,q$互素,等式变成$2q^2 = p^2$,也就是说$p$必然是偶数,令$p = 2m$,则$q^2 = 2m^2$,那个$q$也必然是偶数,这与$p, q$互素矛盾,故$\sqrt{2}$不是有理数. 虽然这个证明很简单,不过这里中间忽略了幂的操作的合法性以及为什么没有上确界,这里就不加证明了. 

\subsubsection{复数域}
\paragraph{}
\textbf{复数集}表示有序对$(x, y)$组成的集合,\textbf{复数域}定义了以下的加法和乘法
\begin{align*}
(x_1, y_1) + (x_2, y_2) &= (x_1 + x_2, y_1 + y_2) \\
(x_1, y_1) \cdot (x_2, y_2) &= (x_1 \cdot x_2 - y_1 \cdot y_2, x_1 \cdot y_2 + x_2 \cdot y_1) 
\end{align*}
其中的0元用$(0, 0)$表示,1元用$(1, 0)$表示,用符号$i := (0, 1)$,根据乘法, $i^2 = (-1, 0)$,一般将形如$(a, 0)$的复数等同于实数$a$,因此$i^2 = -1$.
\paragraph{}
\textbf{定义}\, 复数的共轭,设$z = (a, b)$,则它的共轭$\overline{z} = (a, -b)$,和实数一样,定义一个复数的绝对值,$|z| = \sqrt{z\cdot \overline{z}}$. 因此,如果$z = (a, b)$,那么$|z| = \sqrt{a^2 + b^2}$.
\paragraph{}
\textbf{定义}\, 对于复指数$e^z$,$z = x + iy$,则复指数为一个复数$e^z = e^x(\cos{y} + i\sin{y})$.

\paragraph{}
\textbf{定义}\, 设$z = (x, y) = x + iy$,是一个非零复数,则
$$
Log(z) = \log{|z|} + iarg(z)
$$


\paragraph{}
\textbf{定义}\, 设$z = (x, y) = x + iy$,是一个非零复数,存在
$$
x = |z|\cos{\theta}, y = |z|\sin{\theta}, -\pi < \theta <= \pi
$$
的实数$\theta$称之为$z$的\textbf{辐角主值}, 记为$\theta = arg(z)$,通常也使用$z = r \exp{i\theta}$, 其中$r = |z|, \theta = arg(z) + 2\pi n$,$n$为任意整数.
\paragraph{}
\textbf{定义}\, 设$z$为复数,则
\begin{align*}
\cos{z} = \frac{e^{iz} + e^{-iz}}{2} \\
\sin{z} = \frac{e^{iz} - e^{-iz}}{2i} 
\end{align*}

\subsection{可数集}
\paragraph{}
称集合里元素的个数称为集合的\textbf{基数}或者\textbf{势}, 记为$card(S)$. 如果一个集合和自然数集合等势,那么称这个集合为\textbf{可数集}. 即$card(S) = card(N)$,如果集合是有限的,那么可以称为\textbf{至多可数集}. 如果不是有限的可数集称为\textbf{无限可数集}. 通常为了方便如果$card(S) \leq card(N)$都称为是\textbf{可数集}. 通过可数集可以知道无限和无限之间也是有区别的.

\subsection{欧氏空间}

一个$n$元的有序组$(x_1, x_2, \cdots, x_n)$的集, 以及定义了加法和标量乘法,称这样的集合为一个$n$维的空间. 为了方便,用向量$\textbf{x}$来表示, 因此也可以称之为\textbf{向量空间}. $n$维空间的加法是这样定义的
$$
\textbf{x} + \textbf{y} = (x_1 + y_1, \cdots, x_n + y_n
$$
其标量乘法
$$
a\textbf{x} = (ax_1, \cdots, ax_n)
$$
它们的\textbf{内积},表示为
$$
\textbf{x} \cdot \textbf{y} = \sum_{i = 1}^n x_i y_i
$$
将$x$的\textbf{范数}定义为
$$
|\textbf{x}| = \sqrt{\textbf{x} \cdot \textbf{x}}
$$




\sectionbreak
\part{线性代数}

\section{矩阵}

\subsection{矩阵与向量}
\paragraph{}
假设有一个$m \times n$即$m$行以及$n$列的元素组成的阵列称之为\textbf{矩阵}, 形如
\[
A=
  \begin{bmatrix}
    a_{11} & a_{12} & \cdots &  a_{1n} \\
    a_{21} & a_{22} & \cdots &  a_{2n} \\
    \vdots & \vdots & \ddots &  \vdots \\
    a_{m1} & a_{m2} & \cdots &  a_{mn}
  \end{bmatrix}
\]
可以简单的记为$A = (a_{ij})$,特别的一个$n\times n$的矩阵,称之为\textbf{方阵}. 矩阵也可以看成是一个$R^{m\times n}$的空间,集合论里提到了集合,域,再到$m \times n$维称之为空间. 

\paragraph{}
如果一个矩阵只有一行或者一列,分别称之为\textbf{行向量}和\textbf{列向量},例如行向量$(x_1, x_2, \cdots, x_n)$, 列向量
$$
 \begin{bmatrix}
    y_1 \\
    y_2 \\
    \vdots \\
    y_n
\end{bmatrix}
$$
通常将列向量称为向量,用加粗的字母表示,行向量用列向量的转置表示, 例如
$$
\textbf{x}^T = (x_1, x_2, \cdots, x_n), \textbf{y} =   \begin{bmatrix}
    y_1 \\
    y_2 \\
    \vdots \\
    y_n
\end{bmatrix}
$$
矩阵的转置,将在之后作说明. 因此矩阵同时也可以使用向量来表示, 将距阵
\[
A=
  \begin{bmatrix}
    a_{11} & a_{12} & \cdots &  a_{1n} \\
    a_{21} & a_{22} & \cdots &  a_{2n} \\
    \vdots & \vdots & \ddots &  \vdots \\
    a_{m1} & a_{m2} & \cdots &  a_{mn}
  \end{bmatrix}
\]
简化表示为,$A = (\textbf{a}_1, \textbf{a}_2, \cdots, \textbf{a}_n)$, 或者
$$
A = \begin{bmatrix}
    \textbf{a}_1^T \\
    \textbf{a}_2^T \\
    \vdots  \\
    \textbf{a}_n^T \\
  \end{bmatrix}
$$


\subsection{代数运算}
\paragraph{}
\textbf{加法} \: 对于$m\times n$的矩阵$A$和$B$,$A + B$也是一个$m\times n$的矩阵,令$A = (a_{ij}), B = (b_{ij})$,则$A + B = (a_{ij} + b_{ij})$. 这里将一个全是零的矩阵称之为零矩阵,也就是矩阵加法的单位元,记为0, 则$A + 0 = A$.  例如
$$
 \begin{bmatrix}
    1 & 2 & 3 & 4 \\
    0 & 0 & 0 & 0 
  \end{bmatrix}  + 
  \begin{bmatrix}
    1 & 2 & 3 & 4 \\
    4 & 3 & 2 & 1 
  \end{bmatrix}  = 
  \begin{bmatrix}
    2 & 4 & 6 & 8 \\
    4 & 3 & 2 & 1 
  \end{bmatrix}
$$

\paragraph{}
\textbf{标量乘法} \: 设$A = (a_{ij})$是一个$m\times n$的矩阵, $\alpha$是实数,$\alpha A$也是一个$m\times n$的矩阵,并且$\alpha A = (\alpha a_{ij})$. 例如
$$
2 \cdot 
 \begin{bmatrix}
    1 & 2 & 3 & 4 \\
    0 & 1 & 0 & 0 
  \end{bmatrix}  = 
  \begin{bmatrix}
    2 & 4 & 6 & 8 \\
    0 & 2 & 0 & 0 
  \end{bmatrix}
$$

\paragraph{}
\textbf{矩阵乘法} \: 若$A = (a_{ij})$是一个$m\times n$的矩阵,$B = (b_{ij})$是一个$n\times p$的矩阵,那么$C = A \cdot B = (c_{ij})$是一个$m\times p$的矩阵,注意到矩阵的乘法只有当第一个矩阵的列数和第二个矩阵的行数相同才可以, 所以一般而言$A\cdot B \neq B \cdot A$.  那么$c_{ij} = \textbf{a}_i^T \cdot \textbf{b}_j$ ,在第1章里有提到向量的内积, 即
$$
\textbf{x} \cdot \textbf{y} = \sum_{i = 1}^n x_i y_i
$$,例如
$$
\begin{bmatrix}
    1 & 2 & 3 & 4 \\
    0 & 1 & 0 & 0 
  \end{bmatrix}  \cdot 
  \begin{bmatrix}
    1 & 1 \\
    2 & 0 \\ 
    3 & 1 \\
    5 & 1 \\
  \end{bmatrix} = 
  \begin{bmatrix}
    1 \cdot 1 + 2 \cdot 2 + 3 \cdot 3 + 4 \cdot 5 & 1 \cdot 1 + 2 \cdot 0 + 3 \cdot 1 + 4 \cdot 1 \\
    0 \cdot 1 + 1 \cdot 2 + 0 \cdot 3 + 0 \cdot 5 & 0 \cdot 1 + 1 \cdot 0 + 0 \cdot 1 + 0 \cdot 1
  \end{bmatrix} = 
  \begin{bmatrix}
    34 & 8 \\
    2 & 0
  \end{bmatrix}
$$
特别的,如果一个矩阵是$n\times n$的方阵,这个方阵的对角元素都是$1$,这样的矩阵称之为\textbf{单位矩阵},记为$I$. 设有一个$m\times n$的矩阵$A$,那么$A\cdot I = A$. $I \cdot A = A$,其中前一个单位矩阵$I$是$n\times n$的矩阵,后一个单位矩阵$I$是一个$m\times m$的矩阵. 单位矩阵$I$形如
$$
I = \begin{bmatrix}
    1 & &  \\
       & \ddots & \\ 
       & & 1 \\
  \end{bmatrix}
$$
为了方便,将$I$的第$j$列,记为向量$\textbf{e}_j$, 也就是$j$行的元素是$1$(单位矩阵$i=j$时元素为1),其它的都是$0$,单位矩阵$I$可以简化表示为$I = (\textbf{e}_1, \textbf{e}_2, \cdots, \textbf{e}_n)$.
\newline
设$A$是一个$n\times n$的矩阵,如果存在一个矩阵$A^{-1} \cdot A = I$而且$A\cdot A^{-1} = I$,将$A^{-1}$称为$A$的\textbf{逆}. 如果一个矩阵存在逆,称这个矩阵为\textbf{可逆}的矩阵(有时也称为\textbf{非奇异的}).

\paragraph{}
\textbf{转置} \:  若$A = (a_{ij})$是一个$m\times n$的矩阵,那么将$B = A^T$记为$A$的转置,$B$是一个$n \times m$的矩阵,并且$B = (b_{ij}) = (a_{ji})$. 例如
$$
\begin{bmatrix}
    34 & 8 \\
    2 & 0
  \end{bmatrix}^T = 
  \begin{bmatrix}
    34 & 2 \\
    8 & 0
  \end{bmatrix}
$$
以及
$$
\begin{bmatrix}
    1 & 2 & 3 & 4\\
    4 & 3  & 2 & 1
  \end{bmatrix}^T = 
  \begin{bmatrix}
    1 & 4 \\
    2 & 3 \\
    3 & 2 \\
    4 & 1
  \end{bmatrix}
$$

\paragraph{}
矩阵的加法和乘法满足分配率和结合率,但是只有加法满足交换率,乘法不满足交换率, 不过标量乘法满足交换率. 
\begin{align*}
& A + B  = B + A, \alpha A = A \alpha \\
& A + (B + C) = (A + B) + C, A(BC) = (AB)C, \alpha (\beta A) = (\alpha \beta) A\\
& A (B + C) = AB + AC, \alpha (A + B) = \alpha A + \alpha B
\end{align*}

\paragraph{}
再介绍一下转置的一些性质
\begin{align*}
& (A^T)^T = A \\
& (\alpha A)^T = \alpha A^T \\
& (A + B)^T = A^T + B^T \\
& (AB)^T = B^T A^T
\end{align*}
最后一条,可以使用向量来证明,令$A =  \begin{bmatrix}
    \textbf{a}_1^T\\
    \textbf{a}_2^T\\
    \vdots \\
     \textbf{a}_m^T
  \end{bmatrix}, B = (\textbf{b}_1, \textbf{b}_2), \cdots, \textbf{b}_p$, 那么$(AB)^T = (c_{ij}), B^T A^T = (c_{ij}')$, 因为$c_{ij} = \textbf{a}_j^T \textbf{b}_i, c_{ij}' = \textbf{b}_i^T \textbf{a}_j$,所以$c_{ij} = c_{ij}'$,所以等式成立.

\paragraph{}
\textbf{矩阵的迹} \: 设一个$n\times n$的方阵$A$,它的迹为对角线上元素的和记为
$$
tr(A) = \sum_{i = 1}^n a_{ii}
$$


\subsection{初等变换}
\paragraph{}
\textbf{第一型初等变换} \: 交换两行,第一型初等矩阵形如
$$
 \begin{bmatrix}
   1 &  & \cdots &  &  & 0 \\
    & \ddots & &  & & \\
    &  & 0 & 1 &  & \\
    & & 1 & 0 & &  \\
    &   & & & \ddots & \\
    & & & & & 1 
  \end{bmatrix}
$$ 
这类的矩阵,一般用符号$E$表示,为了区分,这里将第一型的记为$E_{I}$,设有这样的矩阵
$$
A =  \begin{bmatrix}
    1 & 2 & 3 \\
    0 & 0 & 4 \\
    0 & 5 & 0
  \end{bmatrix}, E_{I} = \begin{bmatrix}
    1 & 0 & 0 \\
    0 & 0 & 1 \\
    0 & 1 & 0  
  \end{bmatrix}
$$
那么
$$
E_{I}A = \begin{bmatrix}
    1 & 2 & 3 \\
    0 & 5 & 0 \\
    0 & 0 & 4
  \end{bmatrix}
$$
可以看见,当左乘第一型的初等矩阵之后,$A$的第二行和第三行互换了.  初等矩阵是用于左乘,考察右乘的情况,那就变成交换两列. 例如
$$
AE_{I} = \begin{bmatrix}
    1 & 3 & 2 \\
    0 & 4 & 0 \\
    0 & 0 & 5
  \end{bmatrix}
$$

\paragraph{}
\textbf{第二型初等变换} \:  某一行乘于某个标量,第二型初等矩阵形如
$$
 \begin{bmatrix}
   1 &  & \cdots &  &  & 0 \\
    & \ddots & &  & & \\
    &  & 1 & 0 &  & \\
    & & 0 & a & &  \\
    &   & & & \ddots & \\
    & & & & & 1 
  \end{bmatrix}
$$ 
这里将第二型的记为$E_{II}$,例如这样的矩阵
$$
A =  \begin{bmatrix}
    1 & 2 & 3 \\
    0 & 4 & 2 \\
    0 & 0 & 5
  \end{bmatrix}, E_{II} = \begin{bmatrix}
    1 & 0 & 0 \\
    0 & 1/2 & 0 \\
    0 & 0 & 1  
  \end{bmatrix}
$$
那么
$$
E_{II}A =  \begin{bmatrix}
    1 & 2 & 3 \\
    0 & 2 & 1 \\
    0 & 0 & 5
  \end{bmatrix}
$$
可以看见,矩阵$A$的第二行的元素值,都乘与了$1/2$. 再考虑右乘
$$
AE_{II} =  \begin{bmatrix}
    1 & 1 & 3 \\
    0 & 2 & 2 \\
    0 & 0 & 5
  \end{bmatrix}
$$
也就是说,如果右乘第二型初等矩阵,那么第二列的值乘与了一个标量


\paragraph{}
\textbf{第三型初等变换} \:  将某一行的倍数加到另一行,这一类初等矩阵形如
$$
 \begin{bmatrix}
   1 &  & \cdots &  &  & 0 \\
    & \ddots & &  & & \\
    &  & 1 & 0 &  & \\
    & & 0 & 1 & &  \\
    &  a & & & \ddots & \\
    & & & & & 1 
  \end{bmatrix}
$$ 
这里将第三型的记为$E_{III}$,例如这样的矩阵
$$
A =  \begin{bmatrix}
    1 & 2 & 3 \\
    0 & 4 & 2 \\
    0 & 2 & 5
  \end{bmatrix}, E_{III} = \begin{bmatrix}
    1 & 0 & 0 \\
    0 & 1 & 0 \\
    0 & -1/2 & 1  
  \end{bmatrix}
$$
那么
$$
E_{III}A =  \begin{bmatrix}
    1 & 2 & 3 \\
    0 & 4 & 2 \\
    0 & 0 & 4
  \end{bmatrix}
$$
可以看见,矩阵$A$的第三行的元素值,都加上了第二行的$-1/2$. 再看右乘
$$
AE_{III} =  \begin{bmatrix}
    1 & 1/2 & 3 \\
    0 & 3 & 2 \\
    0 & -1/2 & 5
  \end{bmatrix}
$$
也就是说,如果右乘第三型初等矩阵,那么第二列的值都加上了第三列的$-1/2$. 

\subsection{相似}
相似矩阵是指若一个$n\times n$的矩阵$A$,存在一个$n\times n$的矩阵$P$并且$P$存在逆矩阵$P^{-1}$,使得
$PAP^-1 = B$,这个矩阵$B$称为和$A$相似. 例如
$$
P = \begin{bmatrix}
    1 & -1 \\
    -1 & 2
  \end{bmatrix}, P^{-1} = \begin{bmatrix}
    2 & 1 \\
    1 & 1
  \end{bmatrix}, A =  \begin{bmatrix}
    1 & 2 \\
    3 & 4
  \end{bmatrix}
$$, 那么
$$
PAP^{-1} =  \begin{bmatrix}
    -6 & 0 \\
    16 & 11
  \end{bmatrix}
$$


\subsection{方程组}
\paragraph{}
现实中很多问题,最后都可以转换成线程方程组来求解.  考虑一个$m\times n$的线性方程组
$$
\begin{array}{lcl}
a_{11}x_1 + a_{12}x_2 + \cdots + a_{1n}x_n & = & b_1  \\
a_{21}x_1 + a_{22}x_2 + \cdots + a_{2n}x_n & = & b_2  \\
 & \vdots &   \\
a_{m1}x_1 + a_{m2}x_2 + \cdots + a_{mn}x_n & = & b_m  
\end{array}
$$
这样的线性方程组,可以用矩阵表示成
$$
\begin{bmatrix}
    a_{11} & a_{12} & \cdots & a_{1n} \\
    a_{11} & a_{12} & \cdots & a_{1n} \\
     &  & \vdots &  \\
    a_{m1} & a_{m2} & \cdots & a_{mn} \\
  \end{bmatrix} \cdot 
  \begin{bmatrix}
    x_{1}  \\
    x_{2} \\
    \vdots  \\
    x_{m} \\
  \end{bmatrix}  = \begin{bmatrix}
    b_{1}  \\
    b_{2} \\
    \vdots  \\
    b_{m} \\
  \end{bmatrix} 
$$
用向量表示就是
$$
A\textbf{x} = \textbf{b}
$$
其中$$A$$就是方程数的系数矩阵,如果把右端项加入系数矩阵当中,就形成了\textbf{增广矩阵}
$$
\left[
{\begin{array}{c|c}
\begin{matrix}
a_{11} & a_{12} & \cdots & a_{1n} \\
a_{21} & a_{22} & \cdots & a_{2n} \\
 & & \vdots &  \\
a_{m1} & a_{m2} & \cdots & a_{mn} 
\end{matrix}&
\begin{matrix}
b_1\\
b_2 \\
\vdots \\
b_m 
\end{matrix}
\end{array}
}
\right]
$$
如果$A\textbf{x} = 0$,这样的方程组称之为\textbf{齐次线性方程组}. 齐次方程组必然有一个\textbf{平凡解}$(0, 0, \cdots, 0)$. 如果方程组非零解的话,称之为\textbf{非平凡解}

\paragraph{}
这里不介绍解方程组的方法,现代计算机可以很方便的求解. 因此只会介绍一些常见的矩阵的类型
\subparagraph{}
\textbf{三角形矩阵} \: 定义这样的一个$n\times n$的矩阵$A = (a_{ij})$,当$i < j$时,$a_{ij} = 0$, 称之为\textbf{上三角矩阵}, 相反的称之为\textbf{下三角矩阵}
例如
$$
\begin{bmatrix}
1 & 2 & 3 \\
0 & 4 & 5 \\
0 & 0 & 6 
\end{bmatrix}
$$是上三角矩阵
$$
\begin{bmatrix}
1 & 0 & 0 \\
2 & 4 & 0 \\
3 & 5 & 6 
\end{bmatrix}
$$是一个下三角矩阵,特别的
$$
\begin{bmatrix}
1 & 0 & 0 \\
0 & 2 & 0 \\
0 & 0 & 3 
\end{bmatrix}
$$即是上三角矩阵,也是下三角矩阵,也称之为\textbf{对角矩阵}
\subparagraph{}
\textbf{行阶梯矩阵} \: 定义这样的一个矩阵
\begin{enumerate}
\item 每一个非零行的第一个非零元为1
\item 第$k$行的元素都不为0的话,如果有第$k+1$行,那么第$k + 1$行的0的个数大于第$k$行的0的个数
\end{enumerate}
例如
$$
\begin{bmatrix}
1 & 2 & 3 \\
0 & 1 & 2 \\
0 & 0 & 1 
\end{bmatrix}, 
\begin{bmatrix}
1 & 2 & 3 \\
0 & 1 & 2 \\
0 & 0 & 0
\end{bmatrix}, 
\begin{bmatrix}
1 & 2 & 3 \\
0 & 0 & 1 \\
0 & 0 & 0
\end{bmatrix}
$$
\subparagraph{}
\textbf{行最简矩阵} \: 定义这样的一个矩阵,它是行阶梯矩阵,并且每一行的第一个非零元也就是1,是该列唯一的非零元, 例如
$$
\begin{bmatrix}
1 & 0 & 0 \\
0 & 1 & 0 \\
0 & 0 & 1
\end{bmatrix},
\begin{bmatrix}
1 & 2 & 0 \\
0 & 0 & 1 \\
0 & 0 & 0
\end{bmatrix}, 
\begin{bmatrix}
1 & 2 & 0 & 0 & 3 \\
0 & 0 & 1 & 0 & 0\\
0 & 0 & 0 & 1 & 0
\end{bmatrix}
$$
\subparagraph{}
\textbf{稀疏矩阵} \: 通常把一个矩阵的很多元素都是0的矩阵称之为\textbf{稀疏矩阵}, 也可以说如果是一个$n\times n$的矩阵,它的非零元素的个数是$O(n)$的话,可以看成是一个稀疏矩阵.


\subsection{行列式}
\paragraph{}
对于一个$n\times n$的矩阵来说,矩阵的行列式,可以看成$n$维空间的平行体的体积, 当$n = 1$时,就是线的长度,$n = 2$时是平行四边行的面积,$n = 3$时是平行六面体的体积.

\paragraph{}
当把一个矩阵映射成一个实数的话,这样可以更方便的考察矩阵的特点以及矩阵之间的关系.  定义一个\textbf{行列式}, 对于一个$n\times n$的矩阵$A = (a_{ij})$, 它的行列式用符号表示为$det(A)$,行列式的计算的公式为
$$
det(A) = a_{i1}(-1)^{i+1} det(M_{i1}) + a_{i2} (-1)^{i+2} det(M_{i2}) + \cdots +  a_{in} (-1)^{i+n} det(M_{in})
$$
其中,$M_{in}$是某一行某个元素的\textbf{子式}, 它把元素所在的行和列去掉后的矩阵,形如
$$
M_{ij} = \begin{bmatrix}
a_{11} & \cdots & a_{1(j-1)} & a_{1(j+1)} & \cdots & a_{1n} \\
 & \vdots &  &  &  \\
a_{(i-1)1} & \cdots & a_{(i-1)(j-1)} & a_{(i-1)(j+1)} & \cdots & a_{(i - 1)n} \\
a_{(i+1)1} & \cdots & a_{(i+1)(j-1)} & a_{(i+1)(j+1)} & \cdots & a_{(i + 1)n} \\
 & \vdots &  &  &  \\
 a_{n1} & \cdots & a_{n(j-1)} & a_{n(j+1)} & \cdots & a_{nn} \\
\end{bmatrix}
$$
定义\textbf{余子式}
$$
A_{ij} = (-1)^{i+j}det(M_{ij})
$$
那么可以简化表示为
$$
det(A) = a_{i1}A_{i1} + a_{i2}A_{i2} + \cdots + a_{in}A_{in}
$$

例如
$$
A = \begin{bmatrix}
1 & 2 & 3 \\
1 & 0 & 1 \\
1 & 1 & 3
\end{bmatrix}
$$
那么
\begin{align*}
det(A) &= 1 \cdot det(\begin{bmatrix}
0 & 1 \\
1 & 3
\end{bmatrix}) - 2 \cdot det(\begin{bmatrix}
1 & 1 \\
1 & 3
\end{bmatrix}) + 3 \cdot det(\begin{bmatrix}
1 & 0 \\
1 & 1
\end{bmatrix})  \\
& = 1 \cdot (0 \cdot 3 - 1 \cdot 1) - 2 \cdot (1 \cdot 3 - 1 \cdot 1) + 3 \cdot (1 \cdot 1 - 0 \cdot 1)  \\
& = -2
\end{align*}

例如
$$
A = \begin{bmatrix}
1 & 2 & 3 \\
1 & 1 & 1 \\
1 & 1 & 1
\end{bmatrix}
$$
那么
\begin{align*}
det(A) &= 1 \cdot det(\begin{bmatrix}
1 & 1 \\
1 & 1
\end{bmatrix}) - 2 \cdot det(\begin{bmatrix}
1 & 1 \\
1 & 1
\end{bmatrix}) + 3 \cdot det(\begin{bmatrix}
1 & 1 \\
1 & 1
\end{bmatrix})  \\
& = 1 \cdot (1 \cdot 1 - 1 \cdot 1) - 2 \cdot (1 \cdot 1 - 1 \cdot 1) + 3 \cdot (1 \cdot 1 - 1 \cdot 1)  \\
& =  0
\end{align*}

\paragraph{}
接下来证明行列式的唯一性,根据行列式的定义,行列式是一定存在的,它是行列式的某一行的各个元素乘于它的余子式,需要证明的是不管选取哪一行,矩阵的行列式都是同一个值, 容易证明,无论选取哪一行$2\times 2$和$3 \times 3$的矩阵的行列式只有一个.  因此假设在$n > 3$的情况下,如果$(n - 1) \times (n - 1)$和$(n - 2) \times (n - 2)$ 的矩阵的行列式唯一(即与选取哪一行无关),是否可以推导出$n \times n$的矩阵行列式唯一.  考察第$i$和$j$行的情况下, 其中$i < j$
\begin{align*}
& det_i(A) = a_{i1}A_{i1} + a_{i2}A_{i2} + \cdots + a_{in}A_{in} \\
& det_j(A) = a_{j1}A_{j1} + a_{j2}A_{j2} + \cdots + a_{jn}A_{jn} 
\end{align*}
再把其中的余子式展开($det_i(A)$中的余子式,选择第$j$行, $det_j(A)$中的余子式,选择第$i$行展开), 因为前面已经假设$(n - 1) \times (n - 1)$和$(n - 2) \times (n - 2)$唯一,所以它们选择哪一行展开是没有区别的。
\begin{align*}
& det_i(A) = a_{i1}(-1)^{i+1}(a_{j2}A_{(i1)(j2)} + a_{j3}A_{(i1)(j3)} + \cdots + a_{nn}A_{(i1)(nn)}) + \cdots  \\
& det_j(A) = a_{j1}A_{j1} + a_{j2}(-1)^{j+2}(a_{i1}A_{(j2)(i1)} + a_{i3}A_{(j2)(i3)} + \cdots) + \cdots
\end{align*}
提取其中相似的两项
$$
(-1)^{i+1}a_{i1}a_{j2}A_{(i1)(j2)},  (-1)^{j+2}a_{j2}a_{i1}A_{(j2)(i1)}
$$
而
\begin{align*}
(-1)^{i+1}a_{i1}a_{j2}A_{(i1)(j2)} & = (-1)^{i+1}(-1)^{j + 2 - 2}a_{i1}a_{j2}det(M_{(i1)(j2)}) \\
& (-1)^{j+2}a_{j2}a_{i1}A_{(j2)(i1)} & = (-1)^{j+2}(-1)^{i+1 }a_{j2}a_{i1}det(M_{(j2)(i1)}) 
\end{align*}
其中$M_{(i1)(j2)}$ 和$M_{(j2)(i1)}$ 是 $(n -2) \times (n-2)$ 的矩阵,它们的元素相同.  根据我们的假设那么$det(M_{(i1)(j2)}) = det(M_{(j2)(i1)}$. 而$(-1)^{j+2}(-1)^{i+1} = (-1)^{i+1}(-1)^{j}$这两个乘数的符号相同. 其它的类似. 由此我们可以证明,当$n \times n$时,同样的行列式是唯一的. 

\subsubsection{矩阵的逆}
\paragraph{}
如果有一$n\times n$的矩阵$A$, 定义一个新的矩阵
$$
adj A = \begin{bmatrix}
A_{11} & \cdots & A_{n1} \\
& \vdots & \\
A_{1n} & \cdots & A_{nn} 
\end{bmatrix}
$$
其中$A_{ij}$就是元素$a_{ij}$的余子式, 这样的矩阵$adj A$称为矩阵$A$的\textbf{伴随矩阵}. 那么设$A' = A \cdot adj A = (a'_{ij})$, 其中
$$
a'_{ij} = a_{i1} \cdot A_{j1} + a_{i2} \cdot A_{j2} + \cdots + a_{in} \cdot A_{jn}
$$
先看如果$i = j$的情况下,就是行列式的定义,所以如果$i = j$也就是$A'$的对角元素的值是$det A$,再考虑一下$i \neq j$的情况, 那么可以把$a'_{ij}$看成这样的一个矩阵$A''$,其中第$i$行和第$j$行的元素相同, 因为这种情况下, $a_{ik} = a_{jk}$,所以说$det(A'') = a'_{ij}$.  因此现在考察一个矩阵$A''$,它的第$i$行和第$j$行的元素相同,分别计算选择第$i$行和第$j$行的行列式的值.
\begin{align*}
& det_i(A'') = a_{i1}A_{i1} + a_{i2}A_{i2} + \cdots + a_{in}A_{in} \\
& det_j(A'') = a_{j1}A_{j1} + a_{j2}A_{j2} + \cdots + a_{jn}A_{jn} 
\end{align*}
选取其中的第$1$项, 并且假设$i < j$, 再展开
\begin{align*}
(-1)^{i+1}a_{i1}det(M_{i1})  & = (-1)^{i+1}a_{i1}(\cdots +  a_{j2}A_{(i1)(j2)} + \cdots )   \\
& =  (-1)^{i+1}a_{i1}(\cdots + (-1)^{j+2 - 2} a_{j2} det(M_{(i1)(j2)}) + \cdots ) \\
 (-1)^{j+1}a_{j1}det(M_{j1}) & = (-1)^{j+1}a_{j1}(\cdots +  a_{i2}A_{(j1)(i2)} + \cdots ) \\
 & =  (-1)^{j+1}a_{j1}(\cdots +  (-1)^{i+2-1} a_{i2} det(M_{(j1)(i2)}) + \cdots ) 
\end{align*}

因为$a_{i1} = a_{j1}, a_{i2} = a_{j2}$
并且
\begin{align*}
(-1)^{j+1}(-1)^{i + 2 - 1} & = (-1)^{i+j+2} \\
(-1)^{i+1}(-1)^{j+2 - 2} & = (-1)^{i+j+1}
\end{align*}
两个符号相反,因此其它的类似的,那么
$$
det_i(A'') = -det_j(A'')
$$
但是我们知道,一个矩阵的行列式存在并且唯一,那么$det(A'') = 0$.  回过头来, 
$$
A \cdot adj A = det(A) I
$$
因此,可以发现$A \cdot \frac{1}{det(A)} adj A = I$,如果$det(A) \neq 0$, 也就是说$\frac{1}{det(A)} adj A$就是矩阵$A$的逆. 矩阵$A$的逆存在的充分必要条件是$det(A) \neq 0$.

\subsubsection{行列式的性质}
之前已经证明了行列式的几个性质,这里简要的罗列出来
\begin{enumerate}
\item 如果一个$n\times n$的矩阵$A$,有两行的元素相同,那么$det(A) = 0$
\item 如果一个$n\times n$的矩阵$A$, $det(A) = 0$那么$A$是奇异的(不可逆), 反之就是$A$是非奇异的(可逆的).
\item $det(AB) = det(A)det(B)$, 这个性质可以自己证明. 因此可以说明,如果只有两个矩阵都是非奇异的,那么它们的积才是非奇异的,只要有任何一个是奇异的,它们的积也是奇异的.
\item $det(A^T) = det(A)$
\end{enumerate}

额外的,一般来讲$det(A) + det(B) \neq det(A + B)$






\section{向量空间}

\subsection{向量空间的定义}
把一个$R^{m\times n}$的元素组成的集合,以及对这个集合定义了加法和标量乘法,称之为\textbf{向量空间}. 向量空间的元素一般用小写加粗的字母表示,一般而言,向量是一个$m\times 1$或者$1 \times n$的矩阵,但是一个$m\times n$的矩阵同样可以表示为一个行向量或者列向量,所以说,在这里为了方便也将矩阵看成是一个向量.  在(2.2节)里提到矩阵的代数运算,这里用向量的方式再列一次. 其本质上是一样,矩阵是其中的特例. 在(1.6节)里提到欧氏空间,可以看成是$R^n$中的向量空间. 
\newline
\textbf{加法}\:
\begin{enumerate}
\item 设有两个向量$\textbf{u}, \textbf{v}$,那么$\textbf{u} + \textbf{v} = \textbf{v} + \textbf{u}$, 即满足交换率
\item 设有向量$\textbf{u}, \textbf{v}, \textbf{w}$,那么$\textbf{u} + (\textbf{v} + \textbf{w}) = (\textbf{u} + \textbf{v}) + \textbf{w}$, 即分配率
\item 存在一个向量$\textbf{0}$,即加法的单位元, $\textbf{u} + \textbf{0} = \textbf{u}$
\item 存在加法的逆元,即$\textbf{-u}$为向量$\textbf{u}$的逆元
\end{enumerate}
\textbf{标量乘法}\:
\begin{enumerate}
\item 标量1为乘法的单位元,$1 \cdot \textbf{u} = \textbf{u}$
\item $(\alpha \beta)\textbf{u} = \alpha (\beta \textbf{u})$
\item $\alpha \textbf{u} =\textbf{u}  \alpha$
\end{enumerate}
\textbf{分配率}\:
\begin{enumerate}
\item $\alpha(\textbf{u} + \textbf{v}) = \alpha \textbf{u} + \alpha \textbf{v}$
\item $(\alpha + \beta)\textbf{u} = \alpha \textbf{u} + \beta \textbf{u}$
\end{enumerate}


\subsection{子空间}
\paragraph{}
如果集合$S$是向量空间$V$的一个子集,并且对加法和标量乘法封闭
\begin{enumerate}
\item 若$\textbf{x} \in S$,那么$\alpha \textbf{x} \in S$
\item 若$\textbf{x} \in S, \textbf{y} \in S$,那么$\textbf{x} + \textbf{y} \in S$
\end{enumerate}
将$S$称之为$V$的\textbf{子空间}. 如果一个空间只有一个向量并且这个向量是$\textbf{0}$,称之为\textbf{零子空间}

\paragraph{}
对于一个$m\times n$的齐次方程组$A\textbf{x} = 0$,解的集合组成的空间称为\textbf{零空间}, 记为$N(A)$.  这个和\textbf{零子空间}不一样. 但是例如
$$
A = \begin{bmatrix}
    1 & 1 \\
    1 & -1
  \end{bmatrix} 
$$
$N(A)$就是一个零空间,解集只有0,而
$$
A = \begin{bmatrix}
    1 & 1 
  \end{bmatrix} 
$$
的解集就形如
$$
\{(0, 0)^T, (1, -1)^T, (2, -2)^T, (-1, 1)^T, \cdots\}
$$

\subsection{基和维数}

\subsubsection{线性生成空间}
\paragraph{}
设向量空间$V$中某些元素组成的集合$S = \{\textbf{v}_1, \textbf{v}_2, \cdots, \textbf{v}_n\}, S \subset V$,将它们的线性组合所组成的集合$W$,显然满足向量空间的定义,称之为\textbf{线性生成空间},将$S$称为$W$的\textbf{生成集合}. 记为$W = Span(S) := \{\alpha_1 \textbf{v}_1 + \alpha_2 \textbf{v}_2 + \cdots\}$, 如果是集合$S$的生成空间为$V$,那么$S$也称为$V$的\textbf{生成集合}.

\paragraph{}
例如
$$
R^3 = Span(\{ (1, 0, 0)^T, (0, 1, 0)^T, (0, 0, 1)^T \}), R^3 = Span(\{(1, 0, 0)^T, (0, 1, 0)^T, (0, 0, 1)^T, (2, 4, 6)^T\})
$$

\subsubsection{线性无关}
\paragraph{}
对于向量空间$V$中的向量$\textbf{v}_1, \textbf{v}_2, \cdots, \textbf{v}_n$,如果标量$a_1, a_2, \cdots, a_n$全为0的时候,下面的等式才满足
$$
a_1\textbf{v}_1 + a_2\textbf{v}_2 + \cdots + a_n\textbf{v}_n = 0
$$
将$\textbf{v}_1, \textbf{v}_2, \cdots, \textbf{v}_n$是\textbf{线性无关}的,相反的只要存在非零的$a_k$,那么称它们是\textbf{线性相关}的.

\paragraph{}
例如
$$
\textbf{e}_1 = (1, 0, 0)^T, \textbf{e}_2 = (0, 1, 0)^T, \textbf{e}_3 = (0, 0, 1), \textbf{v}_1 = (1, 1, 0)^T
$$
显然,$\textbf{e}_1$和$\textbf{e}_2$是线性无关的, $\textbf{e}_1, \textbf{e}_2, \textbf{e}_3$也是线性无关的,$\textbf{e}_1, \textbf{v}_1$是线性无关的, $\textbf{e}_1, \textbf{e}_2, \textbf{v}_1$是线性相关的.

\paragraph{}
特别的若是一个$n\times n$的方阵$A$, 如果一组向量组成$A$,如果$det(A) = 0$,那么可以说明,这些向量是线性相关的,如果$det(A) \neq 0$那么这些向量是线性无关的.


\subsubsection{基和维数}
\paragraph{}
如果一组向量$\textbf{v}_1, \textbf{v}_2, \cdots, \textbf{v}_n$,它们是线性无关的,并且生成向量空间$V$. 那么称这些向量为$V$的一组\textbf{基}, 这些向量就是$V$的最小的生成集. 把形如$\textbf{e}_1, \textbf{e}_2, \cdots, \textbf{e}_n$称为\textbf{标准基}.  

\paragraph{}
把一个向量空间$V$的一组基的个数称为向量空间的$维数$记为$\dim V$, 零子空间的维数为0.  对于矩阵而言, 矩阵的维数就是矩阵的\textbf{秩}, 记为$rank(A)$,  矩阵秩有个简单的性质$rank(A) = rand(A^T)$.  维数从几何的观点看,可以看成一组基可以生成一个几维的空间.  维数为2就是我们所说的平面,维数为3就是三维的空间.  

\subsubsection{坐标变换}
\paragraph{}
坐标的概念大家应该比较熟悉,用向量空间的语言来说,例如二维的笛卡尔坐标系,就相当于用二维的标准基$\textbf{e}_1 = (1, 0)^T, \textbf{e}_2 = (0, 1)^T$为基,对于二维空间$R^2$里的任何一个向量$\textbf{p} = x \textbf{e}_1 + y \textbf{e}_2$,那么$\textbf{p}$的坐标就是$(x, y)$, 这就是笛卡尔坐标系. 那么,如果是以$\textbf{v}_1 = (2, 0), \textbf{v}_2 = (0, 2)$为基的$R^2$的向量空间, 如果有一个点$\textbf{p} = (1, 1)^T$是以$\textbf{e}_1, \textbf{e}_2$为基下的坐标,那么在$\textbf{v}_1, \textbf{v}_2$下,$\textbf{p} = 1/2 \textbf{v}_1 + 1/2 \textbf{v}_2$,也就是$(1/2, 1/2)$是$\textbf{p}$在$\textbf{v}_1, \textbf{v}_2$为基下的\textbf{坐标}.

\paragraph{}
接下来考察坐标在两组基之间的变换,有一组基$\textbf{u} = \{ (\textbf{u}_1, \textbf{u}_2, \cdots, \textbf{u}_n) \}$, 其中的第$i$个元素用标准基来表示,那么$\textbf{u}_i = c_{i1}\textbf{e}_1 + c_{i2}\textbf{e}_2 + \cdots + c_{in}\textbf{e}_n$, 同样的有一组基$\textbf{v} = \{ (\textbf{v}_1, \textbf{v}_2, \cdots, \textbf{v}_n) \}$, 第$i$个元素用标准基表示,那么$\textbf{v}_i = c_{i1}' \textbf{e}_1 + c_{i2}' \textbf{e}_2 + \cdots c_{in}' \textbf{e}_n$, 如果有一个向量$\textbf{p}$在基$U$下的坐标为$(x_1, x_2, \cdots, x_n)$,那么在基为$V$下的坐标为$(x_1', x_2', \cdots, x_n')$
因此,把它们都转换成标准基下的坐标,分别是
\begin{align*}
x_{i(e)} & = (x_1, x_2, \cdots, x_n) \cdot  \begin{bmatrix}
    c_{11} & c_{21} & \cdots & c_{n1} \\
    c_{12} & c_{22} & \cdots & c_{n2} \\
    & & \vdots & \\
    c_{1n} & c_{2n} & \cdots & c_{nn} 
  \end{bmatrix}   \\
x_{i(e)}' & =  (x_1', x_2', \cdots, x_n') \cdot  \begin{bmatrix}
    c_{11}' & c_{21}' & \cdots & c_{n1}' \\
    c_{12}' & c_{22}' & \cdots & c_{n2}' \\
    & & \vdots & \\
    c_{1n}' & c_{2n}' & \cdots & c_{nn}' 
  \end{bmatrix}
\end{align*}
分别将各自后边的矩阵记为$U, V$,那么$\textbf{x}^T U = \textbf{x}'^T V$, 其中
\begin{align*}
U & = \begin{bmatrix}
    c_{11} & c_{21} & \cdots & c_{n1} \\
    c_{12} & c_{22} & \cdots & c_{n2} \\
    & & \vdots & \\
    c_{1n} & c_{2n} & \cdots & c_{nn} 
  \end{bmatrix} \\
 V  & = \begin{bmatrix}
    c_{11}' & c_{21}' & \cdots & c_{n1}' \\
    c_{12}' & c_{22}' & \cdots & c_{n2}' \\
    & & \vdots & \\
    c_{1n}' & c_{2n}' & \cdots & c_{nn}' 
  \end{bmatrix}
\end{align*}
因此说,$\textbf{x}' = \textbf{x}^T U V^{-1}$.  这里将$U V^{-1}$将之为\textbf{基变换矩阵}. 

\paragraph{}
例如,两组基$ \textbf{u} = \{ (1, 0)^T, (0, 1)^T \}, \textbf{v} = \{ (2, 0)^T, (0, 2)^T \} $, 在$\textbf{u}$基中某个元素的坐标为$(1, 2)$,那么
$$
U = \begin{bmatrix}
   1 & 0 \\
   0 & 1  
  \end{bmatrix}, V = \begin{bmatrix}
   2 & 0 \\
   0 & 2 
  \end{bmatrix}
$$
则
$$
U V^{-1} =  \begin{bmatrix}
   1 & 0 \\
   0 & 1  
  \end{bmatrix} \cdot  \begin{bmatrix}
   1/2 & 0 \\
   0 & 1/2
  \end{bmatrix} = \begin{bmatrix}
   1/2 & 0 \\
   0 & 1/2
  \end{bmatrix}
$$
因此,新的坐标为
$$
(x, y) = (1, 2) \cdot \begin{bmatrix}
   1/2 & 0 \\
   0 & 1/2
  \end{bmatrix} = (1/2, 1)
$$
复杂一点的,如果$ \textbf{u} = \{ (1, 0)^T, (0, 1)^T \}, \textbf{v} = \{ (1, 0)^T, (1, 1)^T \} $那么
$$
U V^{-1} =  \begin{bmatrix}
   1 & 0 \\
   0 & 1  
  \end{bmatrix} \cdot  \begin{bmatrix}
  1  & -1 \\
   0 & 1
  \end{bmatrix} = \begin{bmatrix}
   1 & -1 \\
   0 & 1
  \end{bmatrix}
$$
因此,新的坐标为
$$
(x, y) = (1, 2) \cdot \begin{bmatrix}
   1 & -1 \\
   0 & 1
  \end{bmatrix} = (1, 1)
$$

\subsection{直和}
\paragraph{}
对于向量空间$V$,若其子空间$W_1, W_2, \cdots, W_n$,中各个子空间分别选取一个向量,使$\textbf{v} \in V$可以表示成
$$
\textbf{v} = \textbf{w}_1 + \textbf{w}_2 + \cdots + \textbf{w}_n
$$
则可以将$V$记成$V = W_1 \oplus W_2 \oplus \cdots \oplus W_n$. 称之为\textbf{直和}. 类似于基的概念,$W_1, W_2, \cdots, W_n$是线性无关的. 


\section{线性映射}

\subsection{线性映射}
\paragraph{}
有向量空间$V$和$W$, 将一个映射$L: V \to W$,它满足
\begin{enumerate}
\item $L(\textbf{v}_1 + \textbf{v}_2) = L(\textbf{v}_1) + L(\textbf{v}_2)$
\item $L(\alpha \textbf{v}_1) = \alpha L(\textbf{v}_1)$
\item $L(\alpha \textbf{v}_1 + \beta \textbf{v}_2) = \alpha L(\textbf{v}_1) + \beta L(\textbf{v}_2)$
\end{enumerate}
这样的映射,将为\textbf{线性映射},如果$L: V \to V$, 也就是一种映射把向量空间$V$映射到自身,称为\textbf{线性算子}. 与一般的函数类似,不过这种映射的参数为向量. 

\paragraph{}
在(3.3.4小节)中,坐标变换就是一种线性映射,因为是映射到同一个空间,所以说也称这为线性算子,其中提到了\textbf{基变换矩阵}.  这里推广出来,同样的可以把线性映射表示成一个矩阵, 只不过基变换矩阵是方阵,这里的矩阵就不受这个限制.  例如
$$
A = \begin{bmatrix}
1 & 0 \\
0 & 1 \\
1 & 1 
\end{bmatrix}
$$ 
如果$L(\textbf{x}) = A\textbf{x}$, 如果$\textbf{x}$是空间$R^2$的向量,假设为$(2, 1)^T$那么
$$
L(\textbf{x}) = \begin{bmatrix}
1 & 0 \\
0 & 1 \\
1 & 1 
\end{bmatrix} \cdot \begin{bmatrix}
2 \\
1 
\end{bmatrix} =  \begin{bmatrix}
2 \\
1  \\
3
\end{bmatrix} 
$$
也就是说,这样的一种映射,把$R^2$映射到$R^3$中的一个子空间. 

\subsection{核与象}
\paragraph{}
在(3.2节)中曾经提到过\textbf{零子空间}$N(A)$, 如果把映射表示成矩阵的话,那么把这个零子空间称为线性变换$L$的\textbf{核}. 有线性空间$V$, 如果有$S$是$V$的子空间,$S$的\textbf{象}记为$L(S)$, 整个空间$V$的像称为$L$的\textbf{值域} .

\subsection{特征值}
\paragraph{}
在向量空间$V$中,考虑一个线性算子$L(\textbf{x})$,用矩阵表示为$A\textbf{x}$, 对于一个标量$\lambda$,如果向量空间$V$中有一个非零的向量\textbf{x}, 使得$L(\textbf{x}) = \lambda \textbf{x}$,即$A(\textbf{x}) = \lambda \textbf{x}$,称$\lambda$为矩阵$A$或称为线性变换$L$的\textbf{特征值}, 这个非零的向量\textbf{x}称之为$\lambda$的\textbf{特征向量}.

\paragraph{}
在(3.3.4小节)中提到,一个线性算子(向量空间$V$映射到自身的一种映射)是一个$n\times n$的方阵, 为了求得一种线性映射的所有的特征值和特征向量,令$A$是映射矩阵,
$A \textbf{x} = \lambda \textbf{x}$,那么就意味着
$$
(A - \lambda I) \textbf{x} = 0
$$
有非零解.  看成一个齐次线性方程组,有非零解那么,形成的矩阵的特征值必为0. 也就是列向量线性无关的话只有零解,它们需要线性相关.  因此矩阵$A$有非零的特征值或者说有特征向量的充分必要的条件是
$$
det(A - \lambda I) = 0
$$
举个例子来说
$$
A = \begin{bmatrix}
1 & 0 \\
1 & 1 
\end{bmatrix}
$$
代放上面的方程,
$$
det \begin{bmatrix}
1 - \lambda & 0 \\
1 & 1 - \lambda
\end{bmatrix} = 0
$$
因此
$$
\lambda^2 - 2 \lambda + 1 = 0
$$
所以$\lambda = 1$. 只有一个特征值,它的特征向量形如$(0, a)^T$,$a$为任意标量.






\section{内积空间}

\subsection{赋范向量空间}
\subsubsection{范数}
\paragraph{}
范数对应的是长度的概念, 是对长度的一种推广,对于向量空间$V$中的向量$\textbf{v}$和$\textbf{w}$,定义一个\textbf{范数},用记号$\parallel \cdot \parallel$表示,它满足
\begin{enumerate}
\item $\parallel \textbf{v} \parallel = 0$, 当且仅当$\textbf{v} = \textbf{0}$
\item $\parallel \textbf{v} \parallel \geq 0$
\item $\parallel  \alpha \textbf{v} \parallel = |\alpha| \parallel \textbf{v} \parallel$
\item $\parallel \textbf{v} + \textbf{w} \parallel \leq \parallel \textbf{v} \parallel + \parallel \textbf{w} \parallel$, 三角不等式
\end{enumerate}
另外,如果$\textbf{v} \neq \textbf{0}$时,$\parallel \textbf{v} \parallel = 0$,那么这个时候称之为\textbf{半范数}. 

\paragraph{}
定义了范数的向量空间,称之为\textbf{赋范向量空间}.

\subsection{内积空间}
\paragraph{}
对于向量空间$V$,其中的两个元素$\textbf{v}_1, \textbf{v}_2$, 增加一种运算将两个向量映射成一个实数,这种运算,称之为\textbf{内积}, 记为$\langle \textbf{v}_1, \textbf{v}_2 \rangle$, 它有如下的性质
\begin{enumerate}
\item $\langle \textbf{x}, \textbf{y} \rangle \geq 0$, 仅当$\textbf{x}$时, 内积为0
\item $ \langle \textbf{x}, \textbf{y} \rangle = \langle \textbf{y}, \textbf{x} \rangle$
\item $\langle \alpha \textbf{x} + \beta \textbf{y}, \textbf{z} \rangle = \alpha \langle \textbf{x}, \textbf{z} \rangle + \beta \langle \textbf{y}, \textbf{z} \rangle $
\end{enumerate}
带有内积运算的向量空间,称之为\textbf{内积空间}. 例如,欧氏空间$R^n$中,将内积定义为
$$
\langle \textbf{x}, \textbf{y} \rangle = \textbf{x}^T \textbf{y}
$$

\subsubsection{内积空间中的范数}
\paragraph{}
对于内积空间$V$中的向量$\textbf{v}$,定义一个$\textbf{v}$的\textbf{范数}
$$
\parallel \textbf{v} \parallel = \sqrt{\langle \textbf{v} , \textbf{v} \rangle}
$$
因此说,内积空间可以看成是一个赋范向量空间. 再定义内积空间中两个向量$\textbf{v}_1, \textbf{v}_2$的\textbf{距离}(在分析学中称为度量)为
$$
\parallel \textbf{v}_1 - \textbf{v}_2 \parallel 
$$
将
$$
\textbf{u} = \frac{\textbf{v}}{\parallel \textbf{v} \parallel}
$$
称之为\textbf{单位向量}.


\subsubsection{投影}
定义一个向量空间$V$中两个向量$\textbf{v}, \textbf{w}$, $\textbf{w} \neq \textbf{0}$, 那么从$\textbf{v}$到$\textbf{w}$的\textbf{正交投影}为一个标量
$$
\alpha = \frac{\langle \textbf{v}, \textbf{w} \rangle  }{\parallel \textbf{w} \parallel }
$$
\textbf{向量投影}为
\begin{align*}
\textbf{u} & =  \frac{\langle \textbf{v}, \textbf{w} \rangle  }{\parallel \textbf{w} \parallel } (\frac{\textbf{w}}{\parallel \textbf{w} \parallel}) \\
                 & = \frac{\langle \textbf{v}, \textbf{w}\rangle}{\langle \textbf{w}, \textbf{w} \rangle} \textbf{w}
\end{align*}
由此,可以知道两个向量之间的夹角$\theta$, 
$$
\cos{\theta} =   \frac{\langle \textbf{v}, \textbf{w} \rangle  }{\parallel \textbf{w} \parallel \parallel \textbf{v} \parallel}
$$

\subsection{正交}
\paragraph{}
如果内积空间$V$中,两个向量$\textbf{v}, \textbf{w}$的内积为0,称它们\textbf{正交}, 对应了垂直的概念, 即$\langle \textbf{v}, \textbf{w} \rangle = 0$, 在上一节有提到向量之间的夹角$\theta$满足
$$
\cos{\theta} =  \frac{\langle \textbf{v}, \textbf{w} \rangle  }{\parallel \textbf{w} \parallel \parallel \textbf{v} \parallel}
$$
注意这里假设$\textbf{v}, \textbf{w}$都是非零向量,所以$\parallel \textbf{v} \parallel > 0, \parallel \textbf{w} \parallel > 0 $,  那么如果它们夹角为$90^\circ$,则$\langle \textbf{v}, \textbf{w} \rangle = 0$.

\paragraph{}
如果有集合向量空间$V$中,有$S = \{\textbf{v}_1, \textbf{v}_2, \cdots, \textbf{v}_n\}, S \subset V$, 并且其中任意两个不同的向量$\langle \textbf{v}_i, \textbf{v}_j \rangle = 0$, 称集合$S$为一个\textbf{正交集}. 特别的,如果集合$S$中的各个元素为单位向量,单位向量是指
$$
\textbf{u} = \frac{\textbf{v}}{\parallel \textbf{v} \parallel}
$$
那么这样的集合称为\textbf{规范正交集}. 如果$V = Span(\{\textbf{u}_1, \textbf{u}_2, \cdots, \textbf{u}_n\})$,其中的$\textbf{u}_i$为单位向量,称这样的一组集合为\textbf{规范正交基}.

\subsection{正交子空间}
\paragraph{}
对于内积空间$U$中,两上子空间$V, W$,如果对于任何的$\textbf{u} \in V, \textbf{w} \in W$,都有$\langle \textbf{u}, \textbf{w} \rangle = 0$,称$V$正交于$W$,记为$V\bot W$.  对于若对于所有与空间$V$正交的向量组成的集合记为$V^\bot$,称为$V$的\textbf{正交补}.

\subsection{零空间与列空间}
在(3.2节)中提到过\textbf{零空间}, 把由几个向量线性生成的空间, 将这几个向量组成一个矩阵的话,如果$A\textbf{x} = 0$的非零解组成的集合,它满足空间的定义(对加法和标量乘法封闭),称之为\textbf{零空间}, 记为$N(A)$.  把这几个向量线性生成的空间称为$A$的列空间, 记为$R(A)$. 行空间表示为$R(A^T)$. 由正交的定义,如果$A\textbf{x} = 0$,那么其实就是$N(A) \bot R(A^T)$,就是行零空间和行空间正交的.  因此可以说
$$
N(A^T)\bot R(A), N(A)\bot R(A^T)
$$
同时
$$
N(A) = R(A^T)^\bot, N(A^T) = R(A)^\bot
$$
这里不做证明.

\subsection{最小二乘问题}
\paragraph{}
最小二乘问题在机器学习中有重要的应用,它可以用来拟合统计数据,最简单的一种就是拟合一条直线,使用得统计数据的各个点与拟合的直接的误差最小. 假设一条直线
$$
y = a + bx
$$
我们希望这种直线可以使得己知的数据点的误差最小,那么$a, b$相当于未知数, 假设有数据点$(p_1, p_1'), (p_2, p_2'), \cdots, (p_n, p_n')$.  则
\begin{align*}
p_1' & = a + b p_1 \\
p_2' & = a + b p_2 \\
        & \vdots \\
p_n' & = a + b p_n 
\end{align*}
如果将$\textbf{x} = (a, b)^T$,即将$a, b$看成未知量,为了方便理解用向量$\textbf{x}$表示,则这个方程可以表示为
$$
\begin{bmatrix}
1 & p_1 \\
1 & p_2 \\
   & \vdots \\
1 & p_n 
\end{bmatrix} \cdot \textbf{x} = (p_1', p_2', \cdots, p_n')^T
$$
这是一个$2xn$的矩阵,一般而言$n > 2$时这个线性方程组无解. 将这个方程组简化表示成$A\textbf{x} = \textbf{b}$,也就是要求这个方程组的解,如果不存在解的话,那么退而求使得$\parallel \textbf{b} - A\textbf{x} \parallel$最小的解. 为了方便令$r(\textbf{x}) = \textbf{b} - A\textbf{x}$. 则我们知道$r(\textbf{x}) \bot R(A)$(因为从向量的角度, $\textbf{b} - \textbf{a}$是两个向量组成的三角形的第三边,要使第三边最小,那就是第三边和$\textbf{a}$垂直),也就是$r(\textbf{x}) \in N(A^T)$, 因此$A^T r(\textbf{x}) = A^T (\textbf{b} - A\textbf{x}) = 0$
那么
$$
A^T A \textbf{x} =  A^T\textbf{b}
$$
的解就是我们退而求最小误差的解.

\paragraph{}
例个特殊的例子有数据点$(1,1), (2,2)$,要拟合一条直线,使得结果的误差最小,那么看成一个方程组
$$
\begin{bmatrix}
1 & 1 \\
1 & 2
\end{bmatrix} \textbf{x} = (1, 2)^T
$$
因此,换成求解
$$
\begin{bmatrix}
1 & 1 \\
1 & 2
\end{bmatrix}^T \begin{bmatrix}
1 & 1 \\
1 & 2
\end{bmatrix} \textbf{x} = \begin{bmatrix}
1 & 1 \\
1 & 2
\end{bmatrix}^T (1, 2)^T
$$
也就是求方程组
$$
\begin{bmatrix}
2 & 3 \\
3 & 5
\end{bmatrix} \textbf{x} = (3, 5)^T
$$
求得解为$(0, 1)^T$,就就是直线$y = 0 + x$即$y = x$,这条直线也正好穿过这两个点.

\paragraph{}
再举一个复杂的例子,如果有3个数据点$(1, 1), (1, 2), (2, 1)$,那么
$$
A = \begin{bmatrix}
1 & 1 \\
1 & 1 \\
1 & 2 
\end{bmatrix}, \textbf{b} =  \begin{bmatrix}
1 \\
2 \\
1 
\end{bmatrix},
$$
转换成求
$$
 \begin{bmatrix}
3& 4 \\
4 & 6
\end{bmatrix} \textbf{x} = (4, 5)^T
$$
的解,其解为$(2, -1/2)$. 因此直线为$y = 2 - \frac{1}{2} x$

\paragraph{}
特别要注意的是,这里只举了拟合一条直线的方法,也可以用最小二乘来拟合曲线,这个将在后边提到. 











\sectionbreak
\part{微积分}

\section{度量空间}

\subsection{度量空间}

\subsubsection{度量空间}
对于集合$X$,将其中的元素称为\textbf{点}, 对于$X$中的两个点定义一个实数$d(x_1, x_2)$为\textbf{度量}或者我们熟悉的\textbf{距离},这个度量有以下的性质
\begin{enumerate}
\item $d(x_1, x_2) = 0$,那么$x_1 = x_2$
\item $d(x_1, x_2) \geq 0$
\item $d(x_1, x_2) = d(x_2, x_1)$
\item $d(x_1, x_2) \leq d(x_1, x_3) + d(x_2, x_3)$,即三角不等式
\end{enumerate}
称定义了\textbf{度量}的集合为一个\textbf{度量空间}. 度量空间可以记为$(X; d)$.

\paragraph{}
在(5.1节)中,提到了长度的概念, 这里主要研究的是距离. 赋范向量空间可以通过范数计算出距离,所以赋范向量空间也是一种度量空间.

\subsubsection{开集与闭集}
\paragraph{}
设有一个实数$r > 0$,在度量空间$X$中某一个点$a$,称距离小于$r$的点组成的集合为一个\textbf{开球}, 记为
$$
B(a; r) = \{ x\in X| d(a, x) < r\}
$$
也称之为点$a$的一个\textbf{领域}.

\paragraph{}
一个集合$G \subset X$, 如果对于任何的$x \in G$,都有一个开球$B(x; r)$,使得$B(x; r) \subset G$,称这样的集合为\textbf{开集}. 如果$G$是度量空间$X$的开集,那么把$F = X - G$或者记为$F = X \ G$为\textbf{闭集}.  

\paragraph{}
如果针对开球再定义一个类似的\textbf{闭球}
$$
\overline{B(a; r)} = \{ x\in X| d(a, x) \leq r\}
$$

\paragraph{}
如果一个点$x \in X$,并且集合$E \subset X$,那么如果这个点$x$存在一个邻域$V \subset E$,称这个点为\textbf{内点}. 如果点$x$是$E^c$或者记为$X - E$的一个内点,称它为\textbf{外点}.  如果点$x$的所有的领域都和集合$E$有交集,并且交集的元素至少有一个不为$x$,也可以说如果交集是一个无限集,这样的点称为\textbf{极限点}. 如果一个点$x$的有一个领域和$E$的交集只有一个点$x$,这样的点称为\textbf{孤立点}.

\paragraph{}
集合$E$和它的极限点组成的集合称之为集合$E$的\textbf{闭包},记为$\overline{E}$. 如果$E = \overline{E}$那么$E$是闭集.

\paragraph{}
集合$E$的极限点组成和集合称为\textbf{导集}, 记为$E'$. 内点组成的集合称为\textbf{内核}, 记为$\mathring{E}$


\subsection{紧集}
\paragraph{}
设$E$是度量空间$X$的子集,$E$的\textbf{开覆盖}指的是$X$的一组开子集$\{G_n\}$,使得$E \subset \cup_aG_a$. 

\paragraph{}
设$E$是度量空间$X$的子集,如果$E$的每个开覆盖都含有一个有限子覆盖. 称这样的集合为\textbf{紧集}.

\paragraph{}
因为紧集的概念不足够直观,直观的讲,在$R^n$的度量空间中,紧集就是一个有界的闭集. 在其它的度量空间中不一定成立. 本书也主要探讨$R^n$的空间.









\section{极限}

\subsection{数列的极限}

\subsubsection{数列}
\paragraph{}
数列的极限看成是函数极限的一种特例,它的值域是自然数集$\textbf{N}$(这里的$\textbf{N}$不包括自然数0),设一个集合$\{x_1, x_2, \cdots, x_n, \cdots \} \subset R$, $n$为$1 \to \infty$, 将数列简单的记为$\{x_n\}$, 对于任意的正实数$\varepsilon > 0$, 存在自然数$N$,使得当$n > N$时,有$|x_n - a| < \varepsilon$, 称数列$\{x_n\}$\textbf{收敛}于$a$, 将$a$称之为数列$\{x_n\}$的\textbf{极限}. 如果数列没有极限,那么称这个数列是\textbf{发散}的. 将数列的极限用符号表示为
$$
\lim_{n\to \infty} x_n = a
$$

\paragraph{}
例如,数列$\{2, 2, 2, 2, \cdots \}$的极限,显然这个数列的极限是$2$.  这样的数列称为\textbf{常数列}.  这里可以选取任何一个$N \geq 1$,显然对于任何的$n \geq N \geq 1$都是$x_n - 2 = 0 < \varepsilon$,因为$\varepsilon > 0$. 

\paragraph{}
例如,数列$\{\frac{1}{n}\}$,$\lim_{n\to \infty} \frac{1}{n} = 0$,对于$\varepsilon > 0$,那么取$n > \frac{1}{\varepsilon}$,则$\frac{1}{n} < \varepsilon$,这里的$N$取为$\frac{1}{\varepsilon}$的下取整(类似于计算机中的下取整的函数ceil). 

\paragraph{}
例如,数列$\{(-1)^n\}$,是发散的,因为显然说,这个数列的极限可能是$1$或者$-1$,假设是$1$的话,那么,取$\varepsilon = 1$,对于任何$N >= 1$,如果$n > N$的话,如果$n$为奇数,那么$|(-1)^n - 1| = 2$,显然不满足极限的定义,类似的$-1$也不是它的极限. 

\paragraph{}
有数列$\{x_n\}$,如果对任何一个数$b$,存在一个正整数$N$,使得当$n > N$时,$x_n > b$,就说数列$\{x_n\}$趋于\textbf{正无穷}, 记为$\lim_{n\to \infty} x_n = +\infty$.  反之就是负无穷, 记为$\lim_{n\to \infty} x_n = -\infty$.  如果只有绝对值的情况下满足定义,即$|x_n| > |b|$,那么称数列$\{x_n\}$趋于\textbf{无穷},记为$\lim_{n\to \infty} x_n = \infty$.

\subsubsection{数列极限的性质}
\paragraph{}
数列极限有以下的一些性质
\begin{enumerate}
\item 如果数列有极限,那么极限是唯一的
\item 收敛的数列必有界
\item 数列的极限的邻域包含了数列有限多个数之外的所有项,也就是极限的邻域包含了数列的无限多个项. 邻域是指一个开区间$(a - \delta, a + \delta)$称为$a$的一个邻域,其中$\delta > 0$. 
\item 若$\lim_{n\to \infty} x_n = a, \lim_{n\to \infty} y_n = b$, 如果$a > b$,那么存在一个$N$,当$n > N$时,$x_n > y_n$. 
\item 若$\lim_{n\to \infty} x_n = a, \lim_{n\to \infty} y_n = b, \lim_{n\to \infty} z_n = c$,  若存在一个$N$,当$n > N$时,$x_n \leq y_n \leq z_n$,如果$a = c$,那么$a = b = c$. 
\end{enumerate}

\paragraph{}
\textbf{极限的代数运算} \: 设$\lim_{n\to \infty} x_n = a, \lim_{n\to \infty} y_n = b$
\begin{enumerate}
\item  $\lim_{n\to \infty} (x_n + y_n) = a + b$;
\item $\lim_{n\to \infty} (x_n \cdot y_n) = ab$;
\item $\lim_{n\to \infty} (x_n / y_n) = a/b$,其中$y_n \neq 0, b \neq 0$.
\end{enumerate}

\subsubsection{数列极限存在的判断}
\paragraph{}
数列$\{x_n\}$, 如果对任何的$\varepsilon > 0$, 存在一个正整数$N$, 使得当$m > N, n > N$时,$|x_m - x_n| < \varepsilon$, 这样的数列称之为\textbf{柯西数列}.

\paragraph{}
\textbf{a}. 数列收敛的\textbf{柯西准则}, 即一个数列收敛的充分必要条件是它是柯西数列.

\paragraph{}
有数列$\{x_n\}$,如果对于任何的$n$,$x_{n+1} > x_n$,那么就称这样的数列是\textbf{严格递增数列}, 如果$x_{n+1} \geq x_n$,称之为\textbf{单调递增数列}. 类似的有\textbf{严格递减数列}和\textbf{单调递减数列}.

\paragraph{}
\textbf{b}. \textbf{魏尔斯特拉斯}定理,单调递增的数列有极限的充分必要条件是它有上界(1.4.1小节), 而且数列的极限为$\sup{x_n}$(上确界). 
同样的,可以得出单调递减的数列有极限的充分必要条件是它有下界,并且数列的极限为$\inf{x_n}$(下确界).

\paragraph{}
从数列$\{x_n\}$中选取无限个元素组成的新的数列称为$\{x_n\}$的\textbf{子列}.

\paragraph{}
\textbf{c}. 数列收敛的充分必要条件是它的每个子列收敛. 并且所有的子列有相的极限.

\paragraph{}
定义数列$\{x_n\}$的\textbf{上极限}为
$$
\lim_{n\to \infty} \sup{x_n} = \inf_{n \geq 1} \sup_{k \geq n} x_k 
$$
也可以记为
$$
\lim_{n\to \infty} \sup{x_n}  = \lim_{n\to \infty}(\sup_{k\geq n} x_k)
$$
同样的定义\textbf{下极限}
$$
\lim_{n\to \infty} \inf{x_n} = \sup_{n \geq 1} \inf_{k \geq n} x_k 
$$
也可以记为
$$
\lim_{n\to \infty} \inf{x_n}  = \lim_{n\to \infty}(\inf_{k\geq n} x_k)
$$

\paragraph{}
\textbf{d}. 数列$\{x_n\}$收敛的充分必要条件为它的上下极限相同.即
$$
\lim_{n\to \infty} \sup{x_n} = \lim_{n\to \infty} \inf{x_n}
$$

\subsubsection{无穷大和无穷小}
\paragraph{}
这里增加特殊的符号将$+\infty$表示为\textbf{正无穷}, 将$-\infty$表示为\textbf{负无穷}, 而$\infty$用来表示\textbf{无穷}大.  如果$x \to 0$称为\textbf{无穷小}.

\paragraph{}
对于求极限,遇到无穷大和无穷小,可以用以下的公式来简化计算.
\begin{enumerate}
\item $a + \infty = +\infty, a - \infty = -\infty$
\item $+ \infty + \infty = +\infty,  - \infty - \infty = -\infty$
\item $\frac{1}{\infty} = 0$, 即$\frac{1}{\pm \infty} = 0$
\item $a \cdot 0 = 0$
\end{enumerate}

\subsubsection{一些例子}
\paragraph{}
\textbf{1}. $\lim_{n\to \infty} \frac{1}{n^p} = 0$, 其中$p > 0$
\subparagraph{}
证明\: 对于任意的$\varepsilon > 0$, 取$n > (1/\varepsilon)^{\frac{1}{p}}$. 那么
$$
n^p > (1/\varepsilon)
$$
也就是
$$
|\frac{1}{n^p} - 0|< \varepsilon
$$
满足极限的定义. 所以$\lim_{n\to \infty} \frac{1}{n^p} = 0$

\paragraph{}
\textbf{2}. $\lim_{n\to \infty} \sqrt[n]{p} = 1, p > 0$
\subparagraph{}
证明\: 当$p = 1$时,显然成立. 对于任意的$\varepsilon > 0$, 当$p > 1$时, 取$n > \frac{\ln (p) }{\ln (\varepsilon + 1)}$.
$$
1/n < \frac{\ln (\varepsilon + 1) }{\ln (p)}
$$
而$p > 1$时,$p^{1/n} > 1$, 也就是
$$
p^{1/n} - 1 < p^{\log_p(\varepsilon + 1)} - 1 = \varepsilon + 1 - 1 = \varepsilon
$$
满足极限的定义. 当$0 < p < 1$且$\varepsilon < 1$ 时,取$n > \frac{\ln (p) }{\ln (-\varepsilon + 1)}$
$$
1 - p^{1/n} < 1 - (-\varepsilon + 1) = \varepsilon
$$

\paragraph{}
\textbf{3}. $\lim_{n\to \infty} \sqrt[n]{n} = 1$
\subparagraph{}
证明\:  设$x_n = \sqrt[n]{n} - 1$,当$n=1$的时候,显然$\sqrt[1]{1} = 1$, 因此讨论$n > 1$的情况,所以$x_n > 0$当$n > 1$
$$
n = (1 + x_n)^n
$$
而$(1 + x_n)^n$的展开当中有一项$\frac{n(n-1)}{2} x_n^2$,因为展开的所有的项都是正数的,所以
$$
n > \frac{n(n-1)}{2} x_n^2
$$
也就是$x_n < \sqrt{\frac{2}{n-1}}$, 因此对任意的$\varepsilon > 0$,只要取$n > \frac{2}{\varepsilon^2} + 1$的时候,$x_n = \sqrt[n]{n} - 1 < \varepsilon$.

\paragraph{}
\textbf{4}. $\lim_{n\to \infty} q^n = 0, |q| < 1$
\subparagraph{}
证明\:  取一个$n >\log_{|q|} \varepsilon $, 这里$\varepsilon < 1$, 因为$\varepsilon >= 1$时,任何的一个$|q|^n < 1$, 因此
$$
|q^n| = |q|^n < |q|^{\log_{|q|} \varepsilon}  =  \varepsilon
$$
满足极限的定义


\paragraph{}
\textbf{5}. $\lim_{n\to \infty} \frac{q^n}{n!} = 0$
\subparagraph{}
证明\:  因为$1 \geq q \leq 1$时,显然成立,考虑当$q > 1$的情况下,对于这个数列的第$n$项
$$
x_n = \frac{q^n}{n!}
$$
第$n+1$项
$$
x_{n+1} = \frac{q^{n+1}}{(n+1)!} = x_n \cdot \frac{q}{n+1}
$$
因此,当$n+1 > q$的时候,$x_{n+1} < x_n$,所以说,将数列的前几项去掉之后就变成一个最终单调递减的数列,并且有下界(大于$0$),于是这个极限收敛,考虑另一个数列$\{\frac{q^{n+1}}{(n+1)!}\}$的极限,这个极限和数列$\{ \frac{q^n}{n!} \}$的极限是一样的
\begin{align*}
\lim_{n \to \infty} \frac{q}{n+1}  (\frac{q^n}{n!}) & =\lim_{n \to \infty} \frac{q}{n+1}  \lim_{n \to \infty}  (\frac{q^n}{n!})  \\
&= 0 \cdot \lim_{n \to \infty}  (\frac{q^n}{n!})  = 0
\end{align*}
考虑$q < -1$的情况下,取两个子列,分别是$n$为偶数和奇数的情况下,$n$的偶数的情况下已经证明了收敛于$0$,类似的对于$n$为奇数的情况下同样的收敛于$0$.

\subsubsection{数e}
\paragraph{}
数$e$是数学中非常重要的一个实数,这里用数列给出$e$的定义
$$
e := \lim_{n\to \infty} (1 + \frac{1}{n})^n
$$


\subsection{级数}

\subsubsection{级数}
\paragraph{}
将数列$\{x_n\}$的和称为\textbf{级数}. 记为$\sum_{n=1}^\infty x_n$.  把$s_n = \sum_{k=1}^n a_k$称为级数的\textbf{部分和},即$s_n = a_1 + a_2 + \cdots + a_n$. 例如$s_1 = a_1$, $s_2 = a_1 + a_2$.  如果部分和组成的数列有极限,将$\lim_{n\to \infty} s_n = s$称之为级数的\textbf{和}. 如果部分和存在极限称这个级数是\textbf{收敛}的,反之称为\textbf{发散}的.

\subsubsection{级数极限存在的判断}
\paragraph{}
\textbf{a}. 级数收敛的\textbf{柯西准则}, 级数$\sum_{n\to \infty} a_n$收敛的充分必要条件是,对任何的$\varepsilon > 0$,存在着一个正整数$N$,使得当$m \geq n > N$时,
$$
|a_n + \cdots + a_m| < \varepsilon
$$

\paragraph{}
\textbf{b}. 级数$\sum_{n\to \infty} a_n$ 收敛的一个必要条件(并非是充分条件)是$\lim_{n\to \infty} a_n = 0$.

\paragraph{}
\textbf{c}. 如果级数$\sum_{n=1}^\infty |a_n|$收敛,那么称级数$\sum_{n=1}^\infty a_n$\textbf{绝对收敛}, 如果级数绝对收敛,那么级数必收敛.

\paragraph{}
\textbf{d}. 级数加上或者去掉有限个项,不影响其收敛或者发散.

\paragraph{}
\textbf{e}. 级数$\sum_{n=1}^\infty a_n, \sum_{n=1}^\infty b_n$是两个非负项级数(即所有的项都是正数),如果存在正整数$N$,当$n > N$时,$a_n \leq b_n$,那么如果级数$\sum_{n=1}^\infty b_n$收敛,那么$\sum_{n=1}^\infty a_n$也敛,如果$\sum_{n=1}^\infty a_n$发散,那么$\sum_{n=1}^\infty b_n$也发散.

\paragraph{}
\textbf{f}. \textbf{魏尔斯特拉斯比较检验法}, 级数$\sum_{n=1}^\infty a_n, \sum_{n=1}^\infty b_n$, 如果存在正整数$N$, 当$n > N$时,$|a_n \leq b_n$,那么如果级数$\sum_{n=1}^\infty b_n$收敛,级$\sum_{n=1}^\infty b_n$绝对收敛.

\paragraph{}
\textbf{g}. \textbf{柯西收敛准则}, 对于正项级数$\sum_{n=1}^\infty a_n$, 并且$\lim_{n\to \infty} \sqrt[n]{a_n} = p$. 
\begin{enumerate}
\item 如果$p < 1$,那么级数$\sum_{n=1}^\infty a_n$收敛;
\item 如果$p > 1$,  那么级数$\sum_{n=1}^\infty a_n$发散;
\item 如果$p = 1$,那么级数$\sum_{n=1}^\infty a_n$可能收敛也可能发散.
\end{enumerate}

\paragraph{}
\textbf{h}. \textbf{达朗贝尔比值检验法}, 对级数$\sum_{n=1}^\infty a_n$,如果极限
$$
\lim_{n\to \infty}|\frac{a_{n+1}}{a_n}| = p
$$
存在,则
\begin{enumerate}
\item 如果$p < 1$,那么级数$\sum_{n=1}^\infty a_n$绝对收敛;
\item 如果$p > 1$,  那么级数$\sum_{n=1}^\infty a_n$发散;
\item 如果$p = 1$,那么级数$\sum_{n=1}^\infty a_n$可能绝对收敛也可能发散.
\end{enumerate}

\paragraph{}
\textbf{i}. 如果$a_1 \geq a_2 \geq \cdots \geq 0$, 那么级数$\sum_{n\to \infty}a_n$收敛的充分必要条件是
$$
\sum_{k=0}^\infty 2^k a_{2^k} = a_1 + 2a_2 + 4a_4 + \cdots 
$$
收敛


\subsubsection{一些例子}
\paragraph{}
\textbf{1}. 级数$\sum_{n\to \infty} \frac{1}{n}$称为\textbf{调和级数}, 调和级数发散, 根据柯西准则,如果对于一个正整数$N$,当$m \geq n > N$时, 取$m = 2n$时
$$
|\frac{1}{n+1} + \frac{1}{n+2} + \cdots + \frac{1}{2n}| > n \cdot \frac{1}{2n} = \frac{1}{2}
$$
所以对任意的$\varepsilon > 0$, 不成立, 因为当$\varepsilon \leq \frac{1}{2}$的时候就很容易构造使极限的定义不成立. 因此调和级数发散.

\paragraph{}
\textbf{2}. 级数$\sum_{n\to \infty} q^n$,称为\textbf{几何级数}或者\textbf{等比级数}, 显然,如果$q \geq 1$时$q^n \geq 1$,显然通项的极限不为0,显然这个时候是发散的,再看如果$q \leq -1$时, 对任何$\varepsilon > 0$时,若$ q^n < 0 < \varepsilon $,$q ^{n + 1} > | q^n | > 0$,因此这个时候当$\varepsilon < q^{n+1} + q^{n}$时不满足级数极限的定义,所以说若$|q| \geq 1$时,级数发散. 考察$|q| < 1$时,级数的部分和第$n$项
$$
s_n = 1 + q + \cdots + q^n = \frac{1 - q^{n+1}}{1 -q}
$$
因为$\lim_{n\to \infty}q^{n+1} = 0$,  所以$\lim_{n\to \infty} s_n = \frac{1}{1 - q}$,所以当$|q| < 1$时,级数的和为$\frac{1}{1-q}$.


\paragraph{}
\textbf{3}. 当$p > 1$时,级数$\sum_{n=1}^\infty \frac{1}{n^p}$收敛,当$p \leq 1$时发散. 当$p = 1$时,就是调和级数,已经证明调和级数发散.  当$p \leq 0$时,对任意的$n > 0, \frac{1}{n^p} \geq 1$, 因为通项的极限不为0,所以当$p \leq 0$时必发散.  当$p > 0$时,那么级数和
$$
\sum_{k=0}^\infty 2^k \frac{1}{(2^k)^p} = 1 + 2 \frac{1}{2^p} + \cdots  = \sum_{k=0}^\infty (2^{1-p})^k
$$
同时收敛或者同时发散. 由上一个例子,只有$|2^{1-p}| < 1$时,这个级数收敛,因此当$1 -p < 0$时,级数收敛,也就是说$p > 1$时级数$\sum_{n=1}^\infty \frac{1}{n^p}$收敛. 这类的级数称为\textbf{p-级数}.


\subsubsection{幂级数}
\paragraph{}
把形如
$$
f(x) = \sum_0^\infty a_n (x - c)^n
$$
这样的级数,称为\textbf{幂级数}. 

\subsubsection{数e}
\paragraph{}
数$e$用级数的方式可以表示成
$$
e = 1 + \frac{1}{1!} + \frac{1}{2!} + \cdots + \frac{1}{n!} + \cdots
$$


\section{函数的极限和连续性}
\subsection{函数的极限}

\subsubsection{函数的极限}
\paragraph{}
有函数$f(x)$,函数上有一个点$a$, 如果对任何的$\varepsilon > 0$,存在一个数$\delta > 0$, 使得任何的$x$, 且$0 < |x -a| < \delta$,满足
$$
|f(x) - A| < \varepsilon
$$
那么,就说当$x \to a$时,函数$f(x)$的极限为$A$.  将函数的极限表示为
$$
\lim_{x\to a}f(x) = A
$$

\subsubsection{函数极限的代数运算}
\paragraph{}
设有函数$f(x), g(x)$,且$\lim_{x\to a} f(x) = A$, $\lim_{x\to a} g(x) = B$
\begin{enumerate}
\item $\lim_{x\to a}(f+ g)(x) = A + B$
\item $\lim_{x\to a}(f \cdot g)(x) = A \cdot B$
\item $\lim_{x\to a}(f / g)(x) = A / B$, 如果$B \neq 0$, 且$g(x) \neq 0$
\end{enumerate}

\subsubsection{函数极限不等式}
\paragraph{}
设有函数$f(x), g(x)$,且$\lim_{x\to a} f(x) = A$, $\lim_{x\to a} g(x) = B, $
\begin{enumerate}
\item 如果$A < B$,那么在点$a$的一个去心领域中,其中的每个点$f(x) < g(x)$
\item 如果有函数$h(x)$,并且$f(x) \leq h(x) \leq g(x)$,并且如果$\lim_{x\to a} f(x) = \lim_{x\to a} g(x) = A$,那么$\lim_{x\to a} h(x) = A$.
\item 如果在$a$的某个去心领域中,$f(x) > g(x)$那么$A \geq B$, 如果$f(x) \geq g(x)$,那么$A \geq B$.
\item 如果在$a$的某个去心领域中,$f(x) > B$那么$A \geq B$, 如果$f(x) \geq B$,那么$A \geq B$.

\end{enumerate}


\subsubsection{函数极限存在的判断}
\paragraph{}
\textbf{a}. 函数$f(x)$,当$x \to a$存在极限$A$的充分必要条件是,对任何收敛于$a$的数列$\{ x_n \}, x_n \in E\\a$,数列$\{f(x_n)\}$收敛,且收敛于$A$.

\paragraph{}
\textbf{b}. 单调递增函数$f(x)$,若有极限点$s = \sup E$,那么$x \to s$时,函数有极限的充分必要条件是它上有界. 对极限点$i = \inf E$,那么当$x \to i$时,函数有极限的充分必要条件是它下有界.


\subsubsection{函数极限的例子}

\paragraph{}
\textbf{1}. $\lim_{x\to 0} \frac{\sin x}{x} = 1$

\begin{figure}[h]
\centering
\ includegraphics[scale=0.3]{draw-08-01.png}
\caption{半径为$1$的单位圆}
\end{figure}

\paragraph{}
证明\:  如图所示,一个半径为$1$的单位圆,那么$\overline{0C} = \cos{x}$, $\overline{BC} = \sin{x}$, $\overline{OA} = \overline{OB} = 1$, $\stackrel\frown{AB} = x$, $\stackrel\frown{CD} = x \cos{x}$,  扇形$\sphericalangle{OCD}$,扇形$\sphericalangle{OAB}$,三角形$\bigtriangleup{OAB}$的面积分别为
\begin{align*}
S_{\sphericalangle{OCD}} & = \frac{1}{2} x \cos{x}^2  \\
S_{\sphericalangle{OAB}} & = \frac{1}{2}  x \\
S_{\bigtriangleup{OAB}} & = \frac{1}{2} \sin{x}
\end{align*}
因为$ S_{\sphericalangle{OCD}} < S_{\bigtriangleup{OAB}} < S_{\sphericalangle{OAB}}$,所以
$$
\frac{1}{2} x \cos{x}^2 <\frac{1}{2} \sin{x} <    \frac{1}{2} x
$$
这里只考虑$x > 0$的情况, $x < 0$的情况可以类似的证明,也就是说
$$
\cos{x}^2 < \frac{\sin{x}}{x} < 1
$$
因为$\lim_{x\to 0} cos{x}^2 = 1$,而且$\lim_{x \to 0} 1 = 1$,根据极限不等式,那么$\lim_{x\to 0} \frac{\sin{x}}{x} = 1$


\paragraph{}
\textbf{2}.  $\lim_{x \to \infty} (1 + \frac{1}{x})^x = e$

\paragraph{}
证明\: 在(7.1.6小节)里将数$e$定义为
$$
\lim_{n \to \infty} (1 + \frac{1}{n})^n
$$
这里尝试推广到任意的实数$x$而不是正整数.  为了方便,使用程序语言常用的$floor(x)$代表对数$x$向下取整,因此有不等式
$$
(1 + \frac{1}{floor(x) + 1})^{floor(x)} < (1 + \frac{1}{x})^x < (1 + \frac{1}{floor(x)})^{floor(x) + 1}
$$
这里因为$\frac{1}{floor(x) + 1} < \frac{1}{x}$, $floor(x) \leq x$, $\frac{1}{floor(x)} \geq \frac{1}{x}$, $floor(x) + 1 > x$. 并且我们知道
$$
\lim_{n\to \infty} (1 + \frac{1}{n + 1})^{n} = \lim_{n\to \infty} \frac{(1 + \frac{1}{n+1})^{n + 1}}{(1 + \frac{1}{n+1})} = \frac{ \lim_{n\to \infty} (1 + \frac{1}{n+1})^{n + 1}}{ \lim_{n\to \infty} (1 + \frac{1}{n+1}) } = \frac{e}{1} = e
$$
而
$$
\lim_{n\to \infty}  (1 + \frac{1}{n})^{n + 1} = \lim_{n\to \infty} (1 + \frac{1}{n})^{n} \cdot  \lim_{n\to \infty} (1 + \frac{1}{n}) = e \cdot 1 = e
$$
因此由极限的不等式可知$e \leq \lim_{x \to \infty} (1 + \frac{1}{x})^x \leq e$,
所以
$$
\lim_{n \to \infty} (1 + \frac{1}{n})^n
$$


\subsubsection{大O和小o}
\paragraph{}
如果对于函数$f(x)$和$g(x)$,如果
$$
\lim_{x\to a} \frac{f(x)}{g(x)} = 1
$$
则称当$x \to a$时,$f(x)$和$g(x)$渐近等价,记为$f(x) \sim g(x)$

\paragraph{}
如果对于函数$f(x)$和$g(x)$,如果存在$a$的一个领域,使得
$$
|\frac{f(x)}{g(x)}| \leq K
$$
$K$为任意的实数,则记为$f(x) = O(g(x))$, 或者说如果$\lim_{x\to a}\frac{f(x)}{g(x)}$存在则$f(x) = O(g(x))$

\paragraph{}
如果对于函数$f(x)$和$g(x)$,$\lim_{x\to a}\frac{f(x)}{g(x)} = 0$,则记为$f(x)= o(g(x))$.

\paragraph{}
大$O$和小$o$在求极限的时候很有用,今后将在函数的级数展开中再次讨论.

\subsubsection{左右极限}
\paragraph{}
对于函数$f(x)$,在定义域$E$中的某一点$x_0$,如果对于定义域中任何点$x_{0^-} < x_0$,由这些点组成的数列${x_{0^-}}$收敛于$a$,称这样的极限为$x_0$处的\textbf{左极限}, 记为$\lim_{x\to x_{0^-}} = a$,同样的定义$x_{0^+} > x_0$的数列组成的极限为\textbf{右极限},记为$\lim_{x\to x_{0^+}} = b$. 如果$\lim_{x\to x_{0^-}} = \lim_{x\to x_{0^+}}$,则函数在$x_0$处存在极限,且它们的值相同.


\subsection{连续性}

\subsubsection{连续}
\paragraph{}
定义函数$f(x)$在定义域$E$中一点$x_0$,如果
$$
\lim_{x \to x_0} f(x) = f(x_0)
$$
那么称函数$f(x)$在$x_0$这个点\textbf{连续}.

\paragraph{}
如果函数$f(x)$在定义域中$E$处处连续,称$f(x)$为集合$E$的\textbf{连续函数}.

\subsubsection{间断点}
\paragraph{}
如果函数$f(x)$在$x_0$处不连续,称$x_0$为函数$f(x)$的\textbf{间断点},  如果函数$f(x)$在$x_0$处$\lim_{x \to x_{0^-}} f(x) \neq f(x_0)$或者$\lim_{x \to x_{0^+}} f(x) \neq f(x_0)$那么称这样的间断点为\textbf{第一类间断点},也称为可去间断点.  如果$f(x)$在$x_0$处的左右极限有一个不存在,那么称之为\textbf{第二类间断点}.

\subsubsection{连续函数的性质}
\paragraph{}
\textbf{a}. 如果函数$f(x)$在集合$E$(定义域)的连续函数,那么函数在某个点$a$的邻域中有界. 

\paragraph{}
\textbf{b}. \textbf{波尔查诺定理}\, 若函数$f(x)$在闭区间$[a, b]$内连续的,如果$f(a) \cdot f(b) < 0$,那么存在一个点$c$使得$f(x) = 0$. 


\paragraph{}
\textbf{c}. \textbf{介值定理}\, 也称为波尔察诺-柯西第二定理 如果连续函数$f(x)$在点$a,b$中取值为$f(a) = A, f(b) = B, A < B$, 那么对任意的$C, A < C < B$,在$a, b$之间必然存在一点$c$,使得$f(c) = C$. 

\paragraph{}
\textbf{d}.  \textbf{魏尔斯特拉最大值定理}\, 函数$f(x)$在闭区间$[a, b]$中连续,则函数$f(x)$必有界.

\subsubsection{一致连续}
\paragraph{}
若$f(x)$在定义域$E$中连续,对于任何一个数$\varepsilon > 0$,存在一个数$\delta > 0$,使得对于$E$中的任何一个点$x_0$满足
$$
|x - x_0| < \delta, |f(x) - f(x_0)| < \varepsilon
$$
称函数$f(x)$在$E$中\textbf{一致连续},一致连续意味着$x$的变化足够小时,$f(x)$的变化也在足够小的范围内. 

\paragraph{}
若函数$f(x)$在闭区间$[a, b]$内连续,则它在这个区间内\textbf{一致连续}.





\section{导数与微分}

\subsection{导数}

\subsubsection{导数}
\paragraph{}
导数从几何意义上可以看成是函数在某一个点的斜率. 对于函数$y = f(x)$,在某一点$x_0$处,如果
$$
\lim_{\Delta x \to 0} \frac{\Delta y}{\Delta x} = \lim_{\Delta x \to 0}  \frac{f(\Delta x + x_0)  - f(x_0)}{\Delta x}
$$
这样的极限存在,称之为$f(x)$在$x_0$处的\textbf{导数}. 函数在$x_0$处的导数也可以记为$f'(x_0), \frac{\mathrm{d}f}{\mathrm{d}x}(x_0)$.  也就是
$$
f'(x_0) =  \lim_{x\to x_0}  \frac{f(x)  - f(x_0)}{x - x_0}
$$

\paragraph{}
另外,如果导数存在,又可以等价为
$$
\frac{f(x) - f(x_0)}{x - x_0} = f'(x_0) + \alpha(x)
$$
其中$\lim_{x \to x_0} \alpha(x) = 0$

\paragraph{}
类似于左极限和右极限,定义\textbf{左导数}
$$
f'(x_{0^-}) =  \lim_{\Delta x \to 0^-}  \frac{f(\Delta x + x_0)  - f(x_0)}{\Delta x}
$$
以及右导数
$$
f'(x_{0^+}) =  \lim_{\Delta x \to 0^+}  \frac{f(\Delta x + x_0)  - f(x_0)}{\Delta x}
$$
和左右极限一样,如果左导数和右导数存在并且相同,那么函数在点$x_0$处可导. 

\subsubsection{导数和连续性}
\paragraph{}
如果函数$f(x)$在点$x_0$处存在导数,那么
$$
f(x)  = (x - x_0) f'(x_0) + \alpha(x) (x - x_0) + f(x_0)
$$
因此
\begin{align*}
\lim_{x\to x_0} f(x) & = \lim_{x\to x_0} \big( (x - x_0) f'(x_0) + \alpha(x) (x - x_0) + f(x_0) \big) \\
& = \lim_{x\to x_0} (x - x_0) f'(x_0) + \lim_{x\to x_0}  \alpha(x) (x - x_0) + \lim_{x\to x_0} f(x_0) \\
& = 0 + 0 + f(x_0) \\
& = f(x_0)
\end{align*}
所以说,如果函数在某一点可导,那么可以肯定$f(x)$在这一点连续. 换句话说,连续是可导的必要条件,但是连续不一定可导.

\subsubsection{代数运算}
\begin{enumerate}
\item $(\alpha f + \beta g)'(x) = \alpha f'(x) + \beta g'(x)$, 加法
\item $(fg)'(x) = f'(x)g(x) + f(x)g'(x)$, 乘法
\item $(\frac{f}{g})'(x) = \frac{f'(x)g(x) - f(x)g'(x)}{g(x)^2}$
\end{enumerate}

\subsubsection{反函数的导数}
\paragraph{}
如果一个函数在定义域内有反函数,那么反函数的导数为
$$
x_y' = \frac{1}{y_x'}
$$
例如,$f(x) = \arcsin{x}$,是$\sin{x}$的反函数,因为$x = \sin{y}$
$$
f'(x) = \frac{1}{\cos{y}} = \frac{1}{(\sqrt{1 - \sin^2{y}} } = \frac{1}{\sqrt{1 - x^2}}
$$

\subsubsection{常见函数的导数}
\begin{enumerate}
\item $y = c, y' = 0$
\item $y = x, y' = 1$
\item $y = x^n, y' = nx^{n-1}$
\item $y = x^{\frac{1}{2}}, y' = \frac{1}{2\sqrt{x}}$
\item $y = a^x, y' = a^x \ln{a}$
\item $y = e^x, y' = e^x$
\item $y = \log_a{x}, y' = \frac{log_a{e}}{x}$
\item $y = \ln{x}, y' = \frac{1}{x}$
\item $y = \sin{x}, y' = \cos{x}$
\item $y = \cos{x}, y' = -\sin{x}$
\item $y = \tan{x}, y' = \frac{\sin{x}}{\cos{x}} = \frac{\cos^2{x} + \sin^2{x}}{\cos^2{x}} = \frac{1}{\cos^2{x}} $
\item $y = \cot{x}, y' = -\frac{1}{\sin^2{x}}$
\item $y = \arcsin{x}, y' = \frac{1}{\sqrt{1 - x^2}}$
\item $y = \arccos{x}, y' = -\frac{1}{\sqrt{1 - x^2}}$
\item $y = \arctan{x}, y' = \frac{1}{1 + x^2}$
\item $y = \mathrm{arccot} x, y' = -\frac{1}{1 + x^2}$
\end{enumerate}

\subsubsection{常见函数导数的证明}
\paragraph{}
\textbf{1}. $f(x) = x^n$

\paragraph{}
证明\: $f'(x) = \lim_{\Delta x \to 0} \frac{(x + \Delta x)^n - x^n}{\Delta x}$,因此
\begin{align*}
f'(x) & = \lim_{\Delta x \to 0}  \frac{\sum_{k = 0}^n C_n^k x^k \Delta x^{n-k} - x^n}{\Delta x} \\
& =  \lim_{\Delta x \to 0}  \frac{\Delta x^n + C_n^1 \Delta x^{n-1} x + \cdots + C_n^n x^n - x^n }{\Delta x} \\
& = \lim_{\Delta x \to 0}  \frac{\Delta x^n + C_n^1 \Delta x^{n-1} x + \cdots + C_n^{n-1} x^{n-1} \Delta x }{\Delta x} \\
&= \lim_{\Delta x \to 0} ( \Delta x^{n-1} + C_n^1 \Delta x^{n-2} x + \cdots + C_n^{n-1} x^{n-1}) \\
&= C_n^{n-1} x^{n-1} \\
&= n x^{n-1}
\end{align*}

\paragraph{}
\textbf{2}. $f(x) = e^x$

\paragraph{}
证明\: $f'(x) = \lim_{\Delta x \to 0} \frac{e^{(x + \Delta x)} - e^x}{\Delta x}$,因此
$$
 \frac{e^{(x + \Delta x)} - e^x}{\Delta x} = e^x \frac{e^{\Delta x} - 1}{\Delta x}
$$
令$t = e^{\Delta x} - 1$并且当$\Delta x \to 0$时,$t$也趋近于$0$,那么
$$
 \frac{e^{(x + \Delta x)} - e^x}{\Delta x} = e^x \frac{t}{\ln{(t + 1)}}
$$
现在考查另一个极限$\lim_{t \to 0} \frac{t}{\ln{(t+1)}}$
因为
\begin{align*}
\lim_{t \to 0} \frac{t}{\ln{(t+1)}} & = \lim_{t \to 0} \frac{1}{\frac{\ln{(t+1)}}{t}} \\
& = \lim_{t \to 0} \frac{1}{\ln{(t + 1)^{\frac{1}{t}}}} \\
& = \frac{1}{ \lim_{t \to 0} \ln{(t + 1)^{\frac{1}{t}}} } \\
& = \frac{1}{\ln{e}} \\
& = 1
\end{align*}
因此
\begin{align*}
\lim_{\Delta x \to 0}  \frac{e^{(x + \Delta x)} - e^x}{\Delta x} & = \lim_{\Delta x \to 0, t \to 0} e^x \frac{t}{\ln{(t + 1)}} \\
& = e^x
\end{align*}


\paragraph{}
\textbf{3}. $f(x) = \ln{x}$

\paragraph{}
证明\: $f'(x) = \lim_{\Delta x \to 0} \frac{\ln{(x + \Delta x)} - \ln{x}  }{\Delta x}$,因此
\begin{align*}
\frac{\ln{(x + \Delta x)} - \ln{x}  }{\Delta x} &= \frac{\ln{(1 + \frac{\Delta x}{x}}}{\Delta x} \\
& = \frac{1}{x} \frac{\ln_{(1 + \frac{\Delta x}{x} )}}{\frac{\Delta x}{x}} 
\end{align*}
上一个例子,证明过
$$
\lim_{t \to 0} \frac{t}{\ln{(t+1)}} = 1
$$
也就是
$$
\lim_{t \to 0} \frac{\ln{(t+1)}}{t} = 1
$$
因此
\begin{align*}
\lim_{\Delta x \to 0} \frac{\ln{(x + \Delta x)} - \ln{x}  }{\Delta x}  & = \lim_{\Delta x \to 0}  \frac{1}{x} \frac{\ln{(1 + \frac{\Delta x}{x} )}}{\frac{\Delta x}{x}}  \\
& = \frac{1}{x} \lim_{\Delta x \to 0}  \frac{\ln{(1 + \frac{\Delta x}{x} )}}{\frac{\Delta x}{x}} \\
&= \frac{1}{x} \cdot 1 \\
&= \frac{1}{x}
\end{align*}


\paragraph{}
\textbf{4}. $f(x) = \sin{x}$

\paragraph{}
证明\: $f'(x) = \lim_{\Delta x \to 0} \frac{\sin{(x + \Delta x)} - \sin{x}  }{\Delta x}$,这里需要用到(8.1.5小节)里的极限$\lim_{x \to 0} \frac{\sin{x}}{x} = 1$, 因此
\begin{align*}
 \lim_{\Delta x \to 0} \frac{\sin{(x + \Delta x)} - \sin{x}  }{\Delta x} & =  \lim_{\Delta x \to 0}  \frac{(\sin{x} \cos{\Delta x} + \sin{\Delta x} \cos{x}) - \sin{x} }{\Delta x} \\
 & =   \lim_{\Delta x \to 0} ( \frac{\sin{x} (\cos{\Delta x} - 1)}{\Delta x} + \cos{x}  \frac{\sin{\Delta x}}{\Delta x} ) \\
 & =  \lim_{\Delta x \to 0} ( \frac{\sin{x} (\cos{\Delta x} - 1)}{\Delta x} + \cos{x} \\
 & = \sin{x}  \lim_{\Delta x \to 0} \frac{ \cos^2{\frac{\Delta x}{2}} -  \sin^2{\frac{\Delta x}{2}} - 1}{\Delta x} + \cos{x} \\
 & =   \sin{x}  \lim_{\Delta x \to 0} (\frac{\Delta x}{2} \cdot  \frac{ - \sin^2{\frac{\Delta x}{2}}}{(\frac{\Delta x}{2})^2} ) + \cos{x} \\
 & = \cos{x} 
\end{align*}

\paragraph{}
\textbf{5}. $f(x) = \arctan{x}$

\paragraph{}
证明\:  令$x = \tan{y}$
\begin{align*}
f'(x) & = \arctan{x} \mathrm{d} x  \\
& = \frac{1}{\tan'{y}} \\
& = \cos^2{y} \\
& = \frac{1}{\tan^2{y} + 1} \\
& = \frac{1}{x^2 + 1}
\end{align*}



\subsubsection{复合函数的导数}
\paragraph{}
复合函数$f(g(x))$或者记为$f \circ g$的导数为
$$
(f \circ g)' = f'(g(x)) g'(x)
$$
用另一个符号表示可能更直观一些
$$
\frac{\mathrm{d} y}{\mathrm{d} x} = \frac{\mathrm{d} y}{\mathrm{d} u} \cdot  \frac{\mathrm{d} u}{\mathrm{d} x}
$$
这个称为导数的\textbf{链式法则}.


\subsubsection{高阶导数}
\paragraph{}
前面提到了,乘法的导数的代数运算,针对更高阶的乘法的导数,其公式为, 称为\textbf{莱布尼茨公式}
$$
(fg)^{(n)}(x) = \sum_{k = 0}^n C_n^k f^{(n -k)}(x) g^{(k)}(x)
$$
.


\subsection{微分}

\subsubsection{微分}
\paragraph{}
对于函数$y = f(x)$,如果
$$
\Delta y = a \cdot \Delta x + o(\Delta x)
$$
也就是在某一个点$x$, 如果存在一个常数$a$,使得函数值$y$的变化量和$x$的变化量呈线性关系. 称函数在这一点\textbf{可微}, 称$a \Delta x$为函数的微分.

\paragraph{}
微分和导数存在着密切的关系,可微和可导几乎是等价的,把函数的微分记为$\mathrm{d} y$或者$\mathrm{d} f$,函数在一个点可导,那么就在这一点可微,并且微分
$$
\mathrm{d} f(x) = f'(x) \mathrm{d} x
$$
因此,导数的性质和代数运算同样可以用在微分上. 这里不做详细的说明.


\subsubsection{中值定理}
\paragraph{}
如果函数$f(x)$的在开区间$(a, b)$中可导,那么存在一个点$p$使得
$$
\frac{f(b) - f(a)}{b - a} = f'(p)
$$
这个就是微分\textbf{中值定理},这个公式称为\textbf{有限增量公式}. 这个也称为\textbf{拉格朗日中值定理}.

\subsubsection{基本定理}
\paragraph{}
\textbf{a}. \textbf{费马定理}

\paragraph{}
对于定义域为$E$的函数$f(x)$,如果在点$x_0$处,存在一个$\varepsilon > 0$,使得对于$|x - x_0| < \varepsilon$的所有的点(即$x_0$的邻域),使得$f(x) \leq f(x_0)$,称$x_0$为\textbf{局部极大值点},同样的如果$f(x) \geq f(x_0)$称之为\textbf{局部极小值点},它们都称之为\textbf{局部极值点}.  如果$f(x) < f(x_0)$称为\textbf{严格局部极大值点}, 反之为\textbf{严格局部极小值点}. 

\paragraph{}
如果点$x_0$,不是$E$的上界或者下界,或者说$x_0$是一个内点,并且$x_0$是局部极值点,称$x_0$为\textbf{内极值点}.

\paragraph{}
\textbf{费马定理}, 如果函数$f(x)$在内点或者称$x_0$是$f(x)$的内值点,如果在$x_0$可导(可微),那么$f'(x_0) = 0$.

\paragraph{}
\textbf{b}. \textbf{罗尔定理}

\paragraph{}
如果函数$f(x)$在闭区间$[a, b]$内连续,并且至少在开区间$(a, b)$内可导,并且$f(a) = f(b)$,那么存在一个点$a < p < b$使得,$f'(p) = 0$.

\paragraph{}
\textbf{c}. \textbf{中值定理}在前面一节已经提到了,中值定理是罗尔定理的一个推广

\paragraph{}
简单的介绍一下如何用罗尔定理推出中值定理,在闭区间$[a, b]$的函数$f(x)$在开区间$(a, b)$内可导, 考虑一个函数$g(x) = f(x)  - \frac{f(b) - f(a)}{b-a} (x - a)$,显然$g(a) = f(a)$, $g(b) = f(a)$, 因此$g(a) = g(b)$,根据罗尔定理,存在一个$p$使得
$$
g'(p) = 0
$$
因此
$$
g'(p) = f'(p) - \frac{f(b) - f(a)}{b-a} = 0
$$
也就是中值定理

\paragraph{}
用中值定理可以推导出判断函数单调性的方法,如果在开区间$(a, b)$内,函数$f(x)$的层数都是正的,那么,函数$f(x)$在这个开区间内是单调递增的.

\paragraph{}
\textbf{d}. \textbf{广义中值定理},又称为\textbf{柯西中值定理}

\paragraph{}
如果函数$f(x)$和$g(x)$在闭区间$[a, b]$中连续,并且在开区间$(a, b)$内可导,并且在开区间$(a, b)$内,$g(x) \neq 0$,则在$(a, b)$中存在一个点$p$, $a < p < b$,使得
$$
\frac{f(b) - f(a)}{g(b) - g(a)} = \frac{f'(p)}{g'(p)}
$$
称这个定理为\textbf{广义中值定理},因为中值定理是这个定理的一个特例,即$g(x) = x$的情况下就是中值定理了.

\paragraph{}
证明广义中值定理,可以用罗尔定理来证明,考虑函数$h(x) = f(x) ( g(b) - g(a)) - g(x) (f(b) - f(a))$, 因此$h(a) = f(a) g(b) - f(a) g(a) - g(a) f(b) + g(a)f(a) = g(a)f(a) - f(a) g(a)$, $h(b) = f(b) g(b) - f(b) g(a) - g(b)f(b) + f(a)g(b) = f(a)g(b) - f(b)g(a)$,因此$h(a) = h(b)$,所以根据罗尔定理存在一个点$p$使得$h'(p) = 0$,所以
$$
h'(p) = f'(x) (g(b)- g(a)) - g'(x) (f(b) - f(a)) = 0
$$
这就证明了柯西中值定理了.


\subsection{泰勒公式和泰勒级数}
\paragraph{}
把形如
$$
p(x) = p(x_0) + \frac{p'(x_0)}{1!} (x - x_0) + \frac{p''(x_0)}{2!} (x - x_0)^2 + \cdots + 
\frac{p^{(n)}(x_0)}{n!} (x - x_0)^n
$$
的公式称为\textbf{泰勒公式}.

\paragraph{}
设有函数$f(x)$有$n+1$阶导数,令函数
$$
g(x) = f(x_0) + f'(x_0) (x - x_0) + \frac{f''(x_0)}{2} (x - x_0)^2 + \cdots + \frac{f^{(n)(x_0)}}{(n)!} (x - x_0)^{n}
$$
因此
$$
g'(x_0) = f'(x_0), g''(x_0) = f''(x_0), \cdots, g^{(n)} = f^{(n)}(x_0)
$$
设有一个数$t$,令
$$
h(x) = f(x) - g(x) - t (x - x_0)^{(n+1)}
$$
那么
$$
h^{(n + 1)}(x) = f^{(n + 1)}(x) - t (n+1)!
$$
因为$h(x_0) = h'(x_0) = h''(x_0) = \cdots = h^{(n+1)}(x_0) = 0$, 所以在$x_0$点
$$
t  =  \frac{f^{(n + 1)}(x_0)}{(n + 1)!} 
$$
考虑非$x_0$点$x_1> x_0$(相反的情况下类似,这里为了方便这样假设),求出一个$t$,并且希望让$h(x) = 0$. 这样我们就能把$f(x)$和$g(x)$关联起来.  那么因为$h(x_1) = h(x_0) = 0$,因此说根据罗尔定理,那么$(x_0, x_1)$之间存在一个点$x_2$,使得$h'(x_2) = 0$,又因为$h'(x_0) = 0$,因此在$(x_0, x_2)$之间存在一个点$x_3$使得$h''(x_3) = 0$,因此说存在一个点$x_{n+2}$,使得$h^{(n+1)} (x_{n + 2}) = 0$,所以
$$
h^{(n+1)}(x_{n+2}) = f^{(n+1)}(x_{n + 2}) - t (n + 1)! = 0
$$
此时
$$
t = \frac{f^{(n + 1)}(x_{n+2})}{(n+1)!}  
$$
因此说,对于任意的一个$x$,存在一个数$t$,这个数$t$由$x$和$x_0$中间的某个点$p$决定并且$t =  \frac{f^{(n + 1)}(p)}{(n+1)!} $ ,那么这个时候
$$
f(x) = g(x) + t ( x - x_0)^{n + 1} = g(x) +  \frac{f^{(n + 1)}(p)}{(n+1)!}  ( x - x_0)^{n + 1}
$$
再展开$g(x)$,于是
$$
f(x) = f(x_0) + f'(x_0) (x - x_0) + \frac{f''(x_0)}{2} (x - x_0)^2 + \cdots + \frac{f^{(n)(x_0)}}{(n)!} (x - x_0)^{n} +  \frac{f^{(n + 1)}(p)}{(n+1)!}  ( x - x_0)^{n + 1}
$$
这个函数的近似公式称为\textbf{带有拉格朗日余项的泰勒公式}. 其中$x > p > x_0$或者$x < p < x_0$.  用
$$
R_n(x) =  \frac{f^{(n + 1)}(p)}{(n+1)!}  ( x - x_0)^{n + 1}
$$
来表示$f(x)$和$g(x)$的差值,一般称之为误差. 

\paragraph{}
特别的,如果$x \to x_0$,那么$\lim_{x\to x_0} R_n(x) = 0$,并且说$\lim_{x\to x_0} \frac{R_n(x)}{(x - x_0)^n} = 0$,于是对于$x \to x_0$,$R_n(x) = o((x -x_0)^n)$.

\paragraph{}
举例来说函数$f(x) = e^x$,在$x = 0$处展开
$$
e^x = 1 + x + \frac{x^2}{2} + \frac{x^3}{3!} + \frac{x^n}{n!} + \frac{e^p}{(n+1)!} x^{n+1}
$$
其中的
$$
R_n(x) = \frac{e^p}{(n+1)!} x^n+1
$$
并且(7.1.5小节)有证明
$$
\lim_{n\to \infty}  \frac{x^n+1}{(n+1)!}  e^p = 0
$$
于是,如果展开无穷之项之后
$$
e^x = 1 + x + \frac{x^2}{2} + \frac{x^3}{3!} + \frac{x^n}{n!} + \cdots
$$
形如
$$
\sum _{n=0}^{\infty }{\frac {f^{(n)}(a)}{n!}}(x-a)^{n}
$$
的级数就称为\textbf{泰勒级数}(也就是泰勒公式的无穷多项). 因此$e^x$可以展开成泰勒级数.


\subsection{常见函数的泰勒公式展开}
\begin{enumerate}
\item $e^x = 1 + x + \frac{x^2}{2!} + \cdots  = \sum_{n=0}^\infty  \frac{x^n}{n!}$
\item $\sin{x} = x - \frac{x^3}{3!} + \frac{x^5}{5!} - \cdots = \sum_{n=0}^\infty (-1)^n \frac{x^{2n+1}}{(2n + 1)!}$
\item $\cos{x} = 1 - \frac{x^2}{2!} + \frac{x^4}{4!} - \cdots = \sum_{n=0}^\infty (-1)^n \frac{x^{2n}}{(2n)!}$
\item $\frac{1}{1 -x} = 1 + x + x^2 + x^3 + \cdots = \sum_{n=0}^\infty x^n$,当$|x| < 1$时
\item $\frac{1}{1 + x} = 1 - x + x^2 - x^3 + \cdots = \sum_{n=0}^\infty (-1)^n x^n$,当$|x| < 1$时
\end{enumerate}

\subsection{洛必达法则}
\paragraph{}
对于一些特别形式的函数极限,使用洛必达法则可以简化一些计算,例如$\frac{0}{0}, \frac{\infty}{\infty}$这类形式的函数极限
$$
\lim_{x\to a} \frac{f(x)}{g(x)} = \lim_{x\to a} \frac{f'(x)}{g'(x)}
$$
这个就是\textbf{洛必达法则},洛必达法则是广义中值定理的一个应用,亦可以扩展成
$$
\lim_{x\to a} \frac{f(x)}{g(x)} = \lim_{x\to a} \frac{f'(x)}{g'(x)} = \cdots =  \lim_{x\to a} \frac{f^{(n)}(x)}{g^{(n)}(x)}
$$

\paragraph{}
例如$\lim_{x\to 0} \frac{\sin{x}}{x} = \lim_{x \to 0} \frac{\cos{x}}{1} = 1$

\subsection{凸函数}

\subsubsection{凸函数}
\paragraph{}
对于函数$f(x)$的定义域$E$内,任意的两点$x_0, x_1$,对于$t \in [0, 1]$,如果
$$
f(t x_0 + (1- t) x_1) \leq t f(x_0) + (1 - t) f(x_1)
$$
称函数是\textbf{凸函数}. 如果$f(t x_0 + (1- t) x_1) < t f(x_0) + (1 - t) f(x_1)$那么函数$f(x)$称为\textbf{严格凸函数}. 同样的相反的可以称函数为\textbf{凹函数},凹函数又称为\textbf{上凸函数}, 以及\textbf{严格凹函数}. 凸函数也称为\textbf{下凸函数}.

\paragraph{}
如果函数$f(x)$在导函数$f'(x)$是单调递增的,那么$f(x)$就是凸函数,如果$f'(x)$是严格单调递增的,那么$f(x)$也就是严格凸函数.

\paragraph{}
如果函数$f(x)$存在二阶导数,那么如果$f''(x) \geq 0$,则$f(x)$是凸函数,如果$f''(x) > 0$则$f(x)$是严格凸函数.

\subsubsection{拐点}
\paragraph{}
对于函数$f(x)$,如果在$x_0$点的领域内,即存在$\varepsilon > 0$, 使得当$x_0 - x < \varepsilon$时,函数在$(x_0 - \varepsilon, x_0)$这个区间内下凸,在$(x_0, x_0 + \varepsilon)$这个区间下上凸(凹), 称这个点$x_0$为\textbf{拐点}, 反之在左区间上凸(凹), 右区间下凸, 也称之为\textbf{拐点}.

\subsubsection{詹生不等式}
\paragraph{}
如果$f(x)$是凸函数,$x_1, x_2, \cdots, x_n$是开区间$(a, b)$内的点,$\alpha_1, \alpha_2, \cdots, \alpha_n > 0 $, 并且$\alpha_1 + \alpha_2 + \cdots + \alpha_n = 1$,那么
$$
f(\alpha_1 x_1 + \alpha_2 x_2 + \cdots + \alpha_n x_n) \leq \alpha_1 f(x_1) + \alpha_2 f(x_2) + \cdots + \alpha_n f(x_n)
$$

\subsection{插值}
\subsubsection{拉格朗日插值}
\paragraph{}
这里简要的介绍一下插值这种方法,可以用于已经知几个点的情况下,函数的其它点的近似值,在(5.6节)曾提到最小二乘的方法来拟合一条直线以使这条直线离所有数据点最近,而插值法也是类似的,只不过拟合的一条曲线要经过所有的点. 今后我们将提到更多的插值公式,这里只简要的介绍一下拉格朗日插值.

\paragraph{}
假设有$n+1$个数据点$(x_0, y_0), (x_1, y_1), \cdots, (x_n, y_n)$(对这些点排序,使得$x_n > x_{n-1} > \cdots > x_1 > x_0$). 定义一个多项式
$$
l_i(x) = \frac{(x - x_0) (x - x_1) \cdots (x - x_{i-1}) (x - x_{i+1}) \cdots (x-x_n)}{(x_i - x_0) (x_i - x_1) \cdots (x_i - x_{i-1}) (x_i - x_{i+1}) \cdots (x_i -x_n)}
$$
因此
$$
l_0(x_0) =  \frac{(x_0 - x_1) (x_0 - x_2)  \cdots (x_0-x_n)}{(x_0 - x_1) (x_0 - x_2)\cdots  (x_0 -x_n)} = 1
$$
以及
$$
l_0(x_1) =  \frac{(x_1 - x_1) (x_1 - x_2)  \cdots (x_1-x_n)}{(x_0 - x_1) (x_0 - x_2)\cdots  (x_0 -x_n)} = 0
$$
由此,再定义一个多项式
$$
L(x) = \sum_{i = 0}^n y_i l_i(x)
$$
这个多项式称为\textbf{拉格朗日插值多项式},可以将函数
$$
f(x) \approx L(x)
$$
并且
$$
f(x_0) = L(x_0), f(x_1) = L(x_1), \cdots, f(x_n) = L(x_n)
$$
考虑它们的误差的话,准确的公式就是
$$
f(x) = L(x) + \frac{f^{(n+1)(p)}}{(n+1)!} (x - x_0)(x - x_1)\cdots (x - x_n)
$$
这个公式,称为\textbf{带余项的拉格朗日插值公式}

\paragraph{}
同泰勒公式的证明类似,定义一个函数
$$
g(x) = f(x) - L(x) - t (x - x_0)(x - x_1)\cdots (x - x_n)
$$
试图找到一个$t$,使得对任意点$x$,使得$g(x) = 0$,  因为
$$
g(x_0) = 0, g(x_1) = 0, \cdots, g(x_n) = 0
$$
因此存在一点$p$使得$g^{(n+1)} = f^{n+1}(p) - t (n+1)! = 0$, 所以说$t = \frac{f^{(n+1)}(p)}{(n+1)!}$ ,这个$p$可以是$x_0$和$x_n$区间中的点.

\paragraph{}
例如$f(x) = \sin{x}$,当然现实的情况很可能是我们仅有数据点,而不知道实际的函数长什么样子,这里为了对比误差,使用$\sin{x}$作为参考,因此取几个数据点$(0, 0), (\frac{\pi}{4}, \frac{\sqrt{2}}{2}), (\frac{\pi}{2}, 1)$因此
\begin{align*}
L(x) & = \frac{\sqrt{2}}{2} \frac{x (x - \frac{\pi}{2})}{\frac{\pi}{4}  (\frac{\pi}{4} - \frac{\pi}{2})} + \frac{x (x - \frac{\pi}{4})}{\frac{\pi}{2} (\frac{\pi}{2} - \frac{\pi}{4})} \\
	   & = \frac{(8 - 8\sqrt{2}) x^2 + (4\sqrt{2} \pi -2 \pi) x}{\pi^2} 
\end{align*}
其误差
$$
R(x) = - \frac{\cos{p}}{3!} (x - 0) (x - \frac{\pi}{4}) (x - \frac{\pi}{2})
$$
因此如果$x = \frac{3 \pi}{8}$的时候
\begin{align*}
L( \frac{3 \pi}{8}) &  \approx  0.905 \\
\sin{( \frac{3 \pi}{8})} &  \approx 0.92 
\end{align*}
误差为
$$
R( \frac{3 \pi}{8} ) \approx  0.03\cdot \cos{p}
$$
并且我们知道$|\cos{p}| \leq 1$的,所以说,这里的误差不会超过$0.03$.
\section{原函数(不定积分)}

\subsection{原函数}
\paragraph{}
如果一个函数$F(x)$,它的导函数为$f(x)$, 即$F'(x) = f(x)$,那么称$F(x)$为$f(x)$的\textbf{原函数}. 求一个函数的原函数称为\textbf{不定积分}. 它们的关系用符号记为
$$
\mathrm{d} F(x) = f(x) \mathrm{d} x, \, \int f(x) \mathrm{d}x = F(x)  + C 
$$
其中$C$为任意的常数. 

\subsection{原函数的代数运算}
\begin{enumerate}
\item $\int [ f(x) \pm g(x)] \mathrm{d}x = \int f(x) \mathrm{d} x  \pm \int g(x) \mathrm{d} x $
\item $\int \alpha f(x) \mathrm{d}x = \alpha \int f(x) \mathrm{d} x$
\item $\int [\alpha f(x)  + \beta g(x)]\mathrm{d}x = \alpha \int f(x) \mathrm{d} x + \beta \int g(x) \mathrm{d} x $
\end{enumerate}

\subsection{分部积分法}
\paragraph{}
对于函数$f(x), g(x)$,因为$(f(x) g(x))' = f'(x)g(x) + f(x)g'(x)$,为了方便,用微分的记号$\mathrm{d} (f(x)g(x)) = g(x) \mathrm{d}(f(x)) + f(x) \mathrm{d}(g(x))$,为两边加一个积分符号,因此
\begin{align*}
\int \mathrm{d} (f(x)g(x)) & = \int g(x) \mathrm{d}(f(x)) + \int f(x) \mathrm{d}(g(x)) \\
f(x) g(x) & = \int g(x) \mathrm{d}(f(x)) +  \int f(x) \mathrm{d}(g(x)) \\
f(x) g(x) & = \int g(x) f'(x) \mathrm{d}x +  \int f(x) g'(x) \mathrm{d}x \\
\int f(x) g'(x) \mathrm{d}x  & = f(x) g(x)  -   \int g(x) f'(x) \mathrm{d}x
\end{align*}
分部积分法可以用于求一些函数乘积的原函数.

\paragraph{}
例如,求$\int x \sin{x} \mathrm{d} x$,那么
\begin{align*}
f(x) & = x \\
g(x) &= -\cos{x}
\end{align*}
因此,根据分部积分法
$$
\int x \sin{x} \mathrm{d} x = - x \cos{x} - \int (-\cos{x}) \mathrm{d}x = \sin{x} - x \cos{x} + C
$$

\subsection{换元积分法}
\paragraph{}
对于函数$f(t), t = g(x)$,那么求积分的话
\begin{align*}
\int f(t) \mathrm{d}t  = \int f(t) g'(x) dx = \int f(g(x)) g'(x) dx
\end{align*}
因此说,这类形式的积分可以用上面的公式来简化

\paragraph{}
例如,求$\int \sin^3{x} \cos{x} \mathrm{d}x$, 假设$t = \sin{x}$,于是
\begin{align*}
\int \sin^3{x} \cos{x} \mathrm{d} x & = \int t^3 \mathrm{d}t \\
& = \frac{1}{4}t^4 + C \\
& = \frac{1}{4} \sin^4{x} + C
\end{align*}

\subsection{特殊形式的函数}
\subsubsection{有理函数}
\paragraph{}
一般将$n$次多项式记为$P_n = \alpha_0 + \alpha_1 x + \alpha_2 x^2 + \cdots + \alpha_n x^n$多项式, 这里假设有两个多项式$P(x), Q(x)$,将
$$
R(x) = \frac{P(x)}{Q(x)}
$$
称为\textbf{有理函数}. 

\paragraph{}
先介绍一些基础的有理函数的定积分

\paragraph{}
\textbf{1}. $\int \frac{1}{x + \alpha} \mathrm{d} x= \ln{|x + \alpha|} + C$

\paragraph{}
\textbf{2}. $\int \frac{1}{(x + \alpha)^n} \mathrm{d} x = - \frac{1}{(n -) (x + \alpha})^{n-1} + C$

\paragraph{}
\textbf{3}. $\int \frac{1}{x^2 + 1} \mathrm{d} x = \arctan{x} + C$

\paragraph{}
\textbf{4}. $\int \frac{1}{x^2 + \alpha} \mathrm{d} x$, 其中$\alpha \neq 0$, 可以转化为
\begin{align*}
\int \frac{1}{x^2 + \alpha} \mathrm{d} x & = \int \frac{1}{a} (\frac{1}{1 + (\frac{x}{\sqrt{a}})^2}) \mathrm{d} x  \\
& = \frac{1}{a} \arctan{\frac{x}{\sqrt{a}}}
\end{align*}


\paragraph{}
\textbf{5}. $\int \frac{1}{(x^2 + 1)^2} \mathrm{d} x $
\begin{align*}
\int \frac{1}{(x^2 + 1)^2} \mathrm{d} x & = \int \frac{(1 + x^2) - x^2}{(x^2 + 1)^2} \mathrm{d} x \\
& =  \int  \frac{1}{x^2 + 1} \mathrm{d} x  - \int \frac{x^2}{(x^2 + 1)^2} \mathrm{d} x \\
& = \arctan{x} - (-\frac{1}{2}) \int x \mathrm{d} \frac{1}{x^2 + 1} \\
& = \arctan{x} + \frac{1}{2} (x \frac{1}{x^2 + 1} - \int \frac{1}{x^2 + 1} \mathrm{d} x ) \\
& = \arctan{x} + \frac{1}{2} \frac{x}{x^2 + 1} - \frac{1}{2} \arctan{x} \\
& = \frac{1}{2} (\arctan{x} + \frac{x}{x^2 + 1})
\end{align*}

\paragraph{}
\textbf{6}. $\int \frac{1}{(x^2 + 1)^3} \mathrm{d} x $ 
\begin{align*}
\int \frac{1}{(x^2 + 1)^3} \mathrm{d} x & = \int \frac{(1 + x^2) - x^2}{(x^2 + 1)^3} \mathrm{d} x \\
& =  \int  \frac{1}{(x^2 + 1)^2} \mathrm{d} x  - \int \frac{x^2}{(x^2 + 1)^3} \mathrm{d} x \\
& = \frac{1}{2} (\arctan{x} + \frac{x}{x^2 + 1}) - (- \frac{1}{4}) \int x  \mathrm{d} \frac{1}{(x^2 + 1)^2} \\
& =  \frac{1}{2} (\arctan{x} + \frac{x}{x^2 + 1}) + \frac{1}{4} ( x \frac{1}{(x^2 + 1)^2} - \int \frac{1}{(x^2 + 1)^2} \mathrm{d} x) \\
& =  \frac{1}{2} (\arctan{x} + \frac{x}{x^2 + 1}) + \frac{1}{4} \frac{x}{(x^2 + 1)^2} - \frac{1}{8} (\arctan{x} + \frac{x}{x^2 + 1}) + C\\
& = \frac{3}{8}  (\arctan{x} + \frac{x}{x^2 + 1}) + \frac{1}{4} \frac{x}{(x^2 + 1)^2} + C
\end{align*}


\paragraph{}
\textbf{7}. $\int \frac{1}{(x + 1)(x + 2)} \mathrm{d} x $ 

\begin{align*}
\int \frac{1}{(x + 1)(x + 2)} \mathrm{d} x  & = \int (\frac{1}{x+1} - \frac{1}{x+2}) \mathrm{d} x \\
& = ln|x+1| - ln|x+2| + C
\end{align*}

\paragraph{}
\textbf{8}. $\int \ln{x} \mathrm{d} x $ 

\begin{align*}
\int \ln{x} \mathrm{d} x  & = x  \ln{x} - \int x \frac{1}{x} \mathrm{d} x \\
& = x \ln{x} - x + C
\end{align*}

\paragraph{}
\textbf{9}. $\int \frac{1}{x^3 + 1} \mathrm{d} x $ 

\begin{align*}
\int \frac{1}{x^3 + 1} \mathrm{d} x  & = \int (\frac{2 - x}{3(x^2 -x + 1)} +\frac{1}{3(x + 1)} ) \mathrm{d} x   \\
& = \frac{1}{6} \int \frac{-2x + 1 + 3}{x^2 - x + 1} \mathrm{d} x  + \int \frac{1}{3(x + 1)} ) \mathrm{d} x  \\
& = \frac{1}{6} \int \frac{-2x + 1}{x^2 - x + 1} \mathrm{d} x + \frac{1}{2} \int  \frac{1}{x^2 - x + 1} \mathrm{d} x + \frac{1}{3} \ln{|x + 1|} \\
& = -\frac{1}{6} \int  \frac{2x - 1}{x^2 - x + 1} \mathrm{d} x + \frac{1}{2} \int  \frac{1}{x^2 - x + 1} \mathrm{d} x + \frac{1}{3} \ln{|x + 1|} \\
& = -\frac{1}{6} \int \mathrm{d} \ln{(x^2 -x + 1)} +  \frac{1}{2}  \int  \frac{1}{x^2 - x + 1} \mathrm{d} x + \frac{1}{3} \ln{|x + 1|} \\
& = -\frac{1}{6} \ln{(x^2 -x + 1)} +   \frac{1}{3} \ln{|x + 1|} + \frac{1}{2}  \int  \frac{1}{x^2 - x + 1} \mathrm{d} x  \\
& = -\frac{1}{6} \ln{(x^2 -x + 1)} +   \frac{1}{3} \ln{|x + 1|} + \frac{1}{2} \int \frac{1}{(x - \frac{1}{2})^2 + \frac{3}{4})} \mathrm{d} x  \\
& =  -\frac{1}{6} \ln{(x^2 -x + 1)} +   \frac{1}{3} \ln{|x + 1|} +  \frac{1}{2} \int \frac{1}{\frac{3}{4} ( (\frac{2}{\sqrt{3}} x - \frac{1}{\sqrt{3}})^2 + 1)} \mathrm{d} x  \\
& =  -\frac{1}{6} \ln{(x^2 -x + 1)} +   \frac{1}{3} \ln{|x + 1|} +\frac{2}{3} \int  \frac{1}{ (\frac{2}{\sqrt{3}} x - \frac{1}{\sqrt{3}})^2 + 1} \mathrm{d} x  \\
& =  -\frac{1}{6} \ln{(x^2 -x + 1)} +   \frac{1}{3} \ln{|x + 1|} +\frac{2}{3} \frac{\sqrt{3}}{2} \int \frac{1}{ (\frac{2}{\sqrt{3}} x - \frac{1}{\sqrt{3}})^2 + 1} \mathrm{d} (\frac{2}{\sqrt{3}} x - \frac{1}{\sqrt{3}} ) \\
& =  -\frac{1}{6} \ln{(x^2 -x + 1)} +   \frac{1}{3} \ln{|x + 1|} + \frac{\sqrt{3}}{3} \arctan{(\frac{2x - 1}{\sqrt{3}})}  + C
\end{align*}

\paragraph{}
还有更多类似的简单的多项式,那么对于更复杂一些的有理函数,可以转换成上述的更简单的形式,这样就可以求出定积分.

\paragraph{}
例如,求$\int \frac{x^2 + 1}{x^3 + 2x^2 -x - 2} \mathrm{d} x$
\begin{align*}
\int \frac{x^2 + 1}{x^3 + 2x^2 -x - 2} \mathrm{d} x & =  \int \frac{x^2 + 1}{(x-1)(x+1)(x+2)} \mathrm{d} x \\
& =   \int (\frac{1}{3} \frac{1}{x-1} - \frac{1}{x+1} + \frac{5}{3} \frac{1}{x+2} ) \mathrm{d} x \\
& = \frac{1}{3} \ln{|x-1|} - \ln{|x+1|} + \frac{5}{3} \ln{|x+2|} + C
\end{align*}






\section{积分}

\subsection{黎曼积分}
\paragraph{}
对于函数$f(x)$, 定义闭区间$[a, b]$的一个划分$\textbf{P}$, 以此得到的点$x_0, x_1, \cdots, x_n$
$$
a = x_0 \leq x_1 \leq \cdots \leq x_{n-1} \leq x_n = b
$$
令$\Delta x_i = x_i - x_{i - 1}$, 其中$(i = 1, 2, \cdots, n)$,令$\lambda = \sup\{ \Delta x_i \}$,  $[x_i, x_{i-1}]$区间中有一个使得函数值最大值的点$\xi_i$,将函数值记为
$$
M_i = \sup f(\xi_i)
$$
使函数值最小的点,将函数值记为
$$
m_i = \inf f(\xi_i)
$$
将
$$
\overline{I} = \sum_{i = 1}^n M_i \Delta x_i
$$
称为\textbf{上积分}. 以及
$$
\underline{I} = \sum_{i = 1}^n m_i \Delta x_i
$$
称为\textbf{下积分}. 简化记号的话,$f(x)$在闭区间$[a, b]$的上积分和下积分并且有一个划分$P$,可以分别用下积分$s(f, P)$, 上积分$S(f, P)$,很多书中的记法不太一样,一般书说会有说明,有些地方用$L(f, P)$和$U(f, P)$来表示. 直观的理解的话,直角坐标中$f(x)$在闭区间$[a, b]$中的面积,对于确定的划分$P$,总是比上积分的面积小,并且总是比下积分要大. 因此对于任意一个划分$P$中的区间$[x_i, x_{i-1}]$中的任意一点$\xi$,那么
$$
\underline{I} \leq   I_\xi  \leq \overline{I}
$$
或者记为
$$
s(f, P) \leq \sigma(f, P, \xi) \leq S(f, P)
$$
之前定义了$\lambda = \sup\{ \Delta x_i \}$,将$\lambda$是某个划分$P$中的的最大区间长度,有时候为了表示和$P$的关系,也记为$\lambda(P)$,这里就简单的用$\lambda$来表示.
将
$$
I = \lim_{\lambda \to 0} \sum_{i = 1}^n f(\xi_i) \Delta x_i
$$
如果极限存在的话,称$I$为\textbf{黎曼积分}, 也就是我们通常讲的\textbf{定积分}, 或者简要的称为\textbf{积分}. 可以用符号记为
$$
I = \int_a^b f(x) \mathrm{d}x
$$
如果函数$f(x)$在闭区间$[a, b]$内存在积分,称$f(x)$为\textbf{可积函数}.

\paragraph{}
如果函数$f(x)$对于任意的划分$P$,使得
$$
\lim_{\lambda \to 0} s(f, P) = \lim_{\lambda \to 0} S(f, P)
$$
那么,函数就是可积的,并且它的积分和上积分以及下积分相同. 

\paragraph{}
前面提到闭区间$[a, b]$的一个划分$\textbf{P}$, 以此得到的点$x_0, x_1, \cdots, x_n$,如果往这个分割里再加一些点,形成另一个划分$P^*$,也就是说$P \subset P^*$,那么称$P*$为$P$的\textbf{精细划分}.

\paragraph{}
如果$P^*$是$P$的精细划分,那么
$$
s(f, P*) \geq s(f, P), \, S(f, P*) \leq S(f, P)
$$

\paragraph{}
通过精细划分,可以得到黎曼积分的另一个等价的定义, 对于任意的$\varepsilon > 0$, 存在一个数$I$和一个划分$P_n$,使得其任何的精细划分$P_{n++}^*$,都满足
$$
|\sum_{i = 1}^{n++} f(\xi_i) \Delta x_i - I| < \varepsilon
$$
这个数$I$称为黎曼积分.

\subsection{黎曼-斯蒂尔杰斯积分}
\paragraph{}
\textbf{黎曼-斯蒂尔杰斯积分}是黎曼积分的一种推广,将其中的$\Delta x_i = x_i - x_{i - 1}$替换成$g(x_i) - g(x_{i-1})$,由此其记号为
$$
\int_a^b f(x)\mathrm{d} g(x)
$$
注意$g(x)$不一定可导,不一定是连续的. 如果$g(x) = x$,那么就是黎曼积分,如果$g(x)$可导,那么黎曼-斯蒂尔杰斯积分可以转换为黎曼积分
$$
\int_a^b f(x) g'(x) \mathrm{d} x
$$

\subsection{可积函数}

\paragraph{}
为了方便,这里指的可积指的是黎曼可积,或者黎曼-斯蒂尔杰斯可积. 黎曼-斯蒂尔杰斯作为黎曼可积的推广,如果是黎曼可积的,那么必然是黎曼-斯蒂尔杰斯可积的. 

\paragraph{}
\textbf{a}. 可积函数的一个必要条件是在闭区间$[a, b]$内有界. 必要条件是指,如果函数在$[a, b]$内不是有界的,那么肯定是黎曼不可积的. 有界是可积的一个前提条件.

\paragraph{}
\textbf{b}. 可积的一个充分必要条件是,在闭区间$[a, b]$内有界的函数$f(x)$, 对于任意的$\varepsilon > 0$,存在$\delta > 0$,使得对于一个划分$P$, $\lambda(P) < \delta$,满足
$$
|\sum_{i = 1}^n |M_i - m_i| \Delta x_i | < \varepsilon
$$


\paragraph{}
\textbf{c}. 闭区间内的连续函数可积

\paragraph{}
\textbf{d}. 如果函数在闭区间$[a, b]$内只有有限个点不连续,那么函数是可积的.

\paragraph{}
\textbf{e}. 闭区间上的单调函数是可积的.


\subsection{积分的性质}
\begin{enumerate}
\item $\int_a^b (\alpha f + \beta g)(x) \mathrm{d} x = \alpha \int_a^b f(x) \mathrm{d} x + \beta \int_a^b g(x) \mathrm{d} x$
\item 如果$a < b < c$,那么$\int_a^c f(x) \mathrm{d} x = \int_a^b f(x)\mathrm{d} x + \int_b^c f(x) \mathrm{d} x$
\item $\int_a^b f(x) \mathrm{d} (g(x) + h(x)) = \int_a^b f(x) \mathrm{d} g(x) + \int_a^b f(x) \mathrm{d} h(x)$
\item 如果在闭区间$[a, b]$内,$f(x) \leq g(x)$那么$\int_a^b f(x) \mathrm{d} x \leq \int_a^b g(x) \mathrm{d} x$
\item 如果在闭区间在内$[a, b]$内,$|f(x)| \leq \alpha$, $\alpha$为任意的正数,那么
$$
|\int_a^b f(x) \mathrm{d} g(x)| \leq \alpha \cdot (g(b) - g(a))
$$
\end{enumerate}

\subsection{分部积分}
\paragraph{}
同(10.3节)不定积分一样,如果函数存在原函数的话,那么可以用不定积分类似的方法
$$
\int_a^b f(x) g'(x) \mathrm{d} x = f(x) g(x) - \int_a^b g(x) f'(x) \mathrm{d} x
$$

\subsection{黎曼-斯蒂尔杰斯积分的例子}
\paragraph{}
将下面的函数称为\textbf{单位阶跃函数}, 注意不同的地方对单位阶跃函数的定义不同,这里使用下面的这种定义:
$$
I(x) = \begin{cases}
0 & \quad x \leq 0 \\
1 & \quad x > 0
\end{cases}
$$

\paragraph{}
对于函数$f(x)$在闭区间$[a, b]$内,其中有一个点$s$, $f(x)$在$s$点连续,设$g(x) = I(x - s)$,那么
$$
\int_a^b f(x) \mathrm{d} g(x) = f(s)
$$
这个积分就是黎曼不可积,但是黎曼-斯蒂尔杰斯可积的例子. 也就是说,类似于如果$g(x)$是阶跃函数的话,其积分就变成级数,是黎曼不可积的,也正是因此有特别一类函数不可积,才需要扩展到黎曼-斯蒂尔杰斯积分.

\subsection{积分中值定理}
\paragraph{}
设$f(x)$在闭区间$[a, b]$内可积,并且在闭区间内$f(x)$有界,$m \leq f(x) \leq M$,那么在$[m, M]$内存在一个数$\mu $,使得
$$
\int_a^b f(x) \mathrm{d} x = (b - a) \mu
$$
这个就是\textbf{积分中值定理}, 特别的,如果$f(x)$在$[a, b]$内的连续,那么在$[a, b]$内存在一个点$p$,即$a \leq p \leq b$,那么
$$
\int_a^b f(x) \mathrm{d} x = (b - a) f(p)
$$

\paragraph{}
将中值定理进行推广的话,设$f(x), g(x)$在$[a, b]$内可积,并且在区间内,$m = \inf f(x), M = \sup f(x)$,如果$g(x)$在区间$[a, b]$内$f(x) \leq 0$或者$f(x) \geq 0$,那么存在一个数$\mu$并且$m \leq \mu \leq M$使得
$$
\int_a^b f(x) g(x) \mathrm{d} x = \mu \int_a^b g(x) \mathrm{d} x
$$
特别的,如果$f(x)$在$[a, b]$内连续,那么存在一个点$a \leq p \leq b$,使得
$$
\int_a^b f(x) g(x) \mathrm{d} x = f(p) \int_a^b g(x) \mathrm{d} x
$$
这个定理称为\textbf{积分第一中值定理}. 

\paragraph{}
若$f(x), g(x)$在闭区间$[a, b]$内可积,并且$f(x)$单调递增或者单调递减,那么存在一个点$\xi$,$a \leq \xi \leq b$,使得
$$
\int_a^b f(x) g(x) \mathrm{d} x = f(a) \int_a^\xi g(x) \mathrm{d} x + f(b) \int_\xi^b g(x) \mathrm{d} x
$$
称这个定理为\textbf{积分第二中值定理}. 并且如果$g(x) = 1$,那么公式就变成
$$
\int_a^b f(x)\mathrm{d} x = f(a) (\xi - a) + f(b) (b - \xi)
$$


\subsection{微积分基本定理}

\paragraph{}
如果在闭区间$[a, b]$上的函数$f(x)$存在原函数$F(x)$,那么
$$
\int_a^b f(x) \mathrm{d} x = F(b) - F(a)
$$
这个定理称为\textbf{微积分基本定理}. 也称为\textbf{牛顿-莱布尼茨公式}. 这个定理将积分和原函数联系在一起. 这个公式也可以记为
$$
\int_a^b f(x) \mathrm{d} x = F(x)|_a^b
$$

\paragraph{}
如果闭区间$[a, b]$上,$f(x)$的原函数是$F(x)$, $g(x)$的原函数是$G(x)$,那么
$$
\int_a^b F(x) g(x) \mathrm{d}x = F(b)G(b) - F(a)G(a) - \int_a^b f(x)G(x) \mathrm{d}x
$$

\subsection{泰勒公式余项的积分形式}
\paragraph{}
对于一个函数$f(x)$, 并且$f(x)$是$n+1$阶可导的,假设通过泰勒公式在$a$点展开,如果有任何一个点$b$,那么例如$f'(x)$在闭区间$[a, b]$的积分为
\begin{align*}
f(b)  - f(a) & = \int_a^b f'(x)  \mathrm{d} x \\
& =  \int_a^b f'(x) (x - b)' \mathrm{d} x \\
& =  f'(x) (x - b) |_a^b - \int_a^b (x - b) f''(x) \mathrm{d} x \\
& =  f'(a)(b - a) - \frac{1}{2} \int_a^b ((x - b)^2)' f''(x) \mathrm{d} x \\
& =  f'(a)(b - a) - \frac{1}{2} (x-b)^2 f''(x)|_a^b + \frac{1}{2} \int_a^b (x - b)^2 f^{(3)}(x) \mathrm{d} x \\
& =  f'(a)(b - a) + \frac{1}{2} (b-a)^2 f''(a) + \frac{1}{2} \int_a^b (x - b)^2 f^{(3)}(x) \mathrm{d} x \\
& \quad \vdots \\
& = f'(a) (b - a) + \frac{f''(a)}{2!} (b - a)^2 + \cdots + \frac{f^{(n)}(a)}{n!} (b - a)^n + \frac{1}{n!} \int_a^b f^{(n+1)} (x) (b - x)^n \mathrm{d} x
\end{align*}
其中的余项
$$
R_n(x) = \frac{1}{n!} \int_a^b f^{(n+1)} (x) (b - x)^n \mathrm{d} x
$$
这个余项,称为泰勒展开的积分余项,同时根据积分第一中值定理,即
$$
\int_a^b f(x) g(x) \mathrm{d} x = f(p) \int_a^b g(x) \mathrm{d} x
$$
因此
\begin{align*}
R_n(x) & = \frac{1}{n!} \int_a^b f^{(n+1)} (x) (b - x)^n \mathrm{d} x \\
& =  \frac{1}{n!} f^{(n+1)}(p) \int_a^b (b - x)^n \mathrm{d} x \\
& = \frac{1}{n!} f^{(n+1)}(p)   (- \frac{1}{n+1} (b - x)^{(n + 1)})|_a^b \\
& = \frac{1}{n!} f^{(n+1)}(p) \frac{1}{n+1} (b - a)^{(n+1)} \\
& = \frac{f^{(n+1)}(p)}{(n+1)!} (b - a)^{(n+1)}
\end{align*}
这个和之前(9.3节)用微分中值定理推导的泰勒公式的余项一样了.


\subsection{反常积分}
\subsubsection{反常积分}
\paragraph{}
如果函数$f(x)$定义域为$[a, +\infty)$,并且$f(x)$在定义内任意的一个点$b$,使得$f(x)$在$[a, b]$内总是可积的,那么
$$
\int_a^{+\infty} f(x) \mathrm{d} x := \lim_{b\to +\infty} \int_a^b f(x) \mathrm{d} x
$$
如果极限存在,这个积分称为$f(x)$在区间$[a, +\infty)$上的\textbf{反常黎曼积分},一般简单的称为\textbf{反常积分},也称为\textbf{无空限广义积分}. 

\paragraph{}
更一般的定义,在半开区间$[a, \omega)$,$\omega$可以是任意大于$a$的实数或者$+\infty$,当然同样的对于$(\omega, a]$也是一样的,不过这样为了方便,讨论积分上限是某个实数的或者无穷的情况. 如果极限
$$
\int_a^\omega f(x) \mathrm{d} x := \lim_{b \to \omega^-} \int_a^b f(x) \mathrm{d} x
$$
存在的话,称为$f(x)$在区间$[a, \omega)$上的\textbf{反常积分}.  类似的
$$
\int_\omega^b f(x) \mathrm{d} x := \lim_{a \to \omega^+} \int_a^b f(x) \mathrm{d} x
$$
也称为反常积分.  如果$\omega$是具体的实数,那么又称之为\textbf{瑕积分}. 其中的$\omega$称为瑕点. 如果存在反常积分,也称为反常积分\textbf{收敛},反之称为\textbf{发散}的.

\paragraph{}
例如
\begin{align*}
\int_1^{+\infty} \frac{1}{x^2} \mathrm{d} x & = \lim_{b \to +\infty} \int_1^b \frac{1}{x^2} \mathrm{d} x \\
& = \lim_{b \to +\infty} (- \frac{1}{x}  |_1^b) \\
& =  \lim_{b \to +\infty} (1 - \frac{1}{b} ) \\
& = 1
\end{align*}

\subsubsection{反常积分的代数运算}
\paragraph{}
若$f(x), g(x)$在区间$[a, \omega)$存在反常积分,那么
$$
\int_a^\omega(\alpha  f(x) + \beta g(x)) \mathrm{d} x  =\alpha  \int_a^\omega f(x) \mathrm{d} x + \beta \int_a^\omega g(x) \mathrm{d} x
$$
即对加法和标量乘法封闭的.

\subsubsection{反常积分的性质}
\paragraph{}
如果函数$f(x)$在$[a, \omega)$上存在反常积分,并且在$[a, \omega]$上可积,那么反常积分就和积分相同.

\paragraph{}
如果函数$f(x)$在$[a, \omega)$上存在反常积分,那么这个区间中有一个点$c$,即$a \leq c < \omega)$,那么
$$
\int_a^\omega f(x) \mathrm{d} x = \int_a^c f(x) \mathrm{d} x + \int_c^\omega f(x) \mathrm{d} x
$$

\subsubsection{换元积分法}
\paragraph{}
同(10.4节)不定积分类似的, 对于函数$f(x)$,其定义域是区间$[a, \omega )$, 设$x = g(t)$, 并且$g(a) = a$, 以及$\lim_{t\to \omega} g(t) = \lambda$,那么
$$
\int_a^\omega f(x) \mathrm{d} x = \int_a^\lambda f(g(t)) g'(t) \mathrm{d} t
$$

\subsubsection{收敛性的判断}
\paragraph{}
反常积分收敛的\textbf{柯西判别法}\: 如果函数$f(x)$的定义域$[a, \omega)$,并且在其中的任何子区间内可积,对于任何的$\varepsilon > 0$,存在$a \leq \delta < \omega$,使得对于任何的$\delta < b_1, \delta < b_2$,满足
$$
|\int_{b_1}^{b_2} f(x) \mathrm{d} x | < \varepsilon
$$
那么积分
$$
\int_a^\omega f(x) \mathrm{d} x 
$$
收敛,即存在反常积分.


\paragraph{}
如果$\int_a^\omega |f(x)| \mathrm{d} x$收敛,那么$\int_a^\omega f(x) \mathrm{d} x$\textbf{绝对收敛}

\paragraph{}
如果函数$f(x), g(x)$,在区间$[a, \omega)$上有
$$
0 \leq f(x) \leq g(x)
$$
那么,如果$\int_a^\omega g(x) \mathrm{d} x$收敛, $\int_a^\omega f(x) \mathrm{d} x$也收敛. 如果$\int_a^\omega f(x) \mathrm{d} x$发散,那么$\int_a^\omega g(x) \mathrm{d} x$也是发散的. 若$\int_a^\omega g(x) \mathrm{d} x$收敛,那么
$$
\int_a^\omega f(x) \mathrm{d} x \leq \int_a^\omega g(x) \mathrm{d} x
$$

\paragraph{}
\textbf{阿贝尔-狄利克雷准则}\: 设函数$f(x), g(x)$定义域为$[a, \omega)$,那么如果
\begin{enumerate}
\item $\int_a^\omega f(x) \mathrm{d} x$ 收敛
\item $g(x)$是单调函数
\end{enumerate}
或者
\begin{enumerate}
\item $\int_a^\omega f(x) \mathrm{d} x$ 上有界
\item $g(x)$是单调趋于零
\end{enumerate}
只要满足这两组条件中的一组,那么反常积分
$$
\int_a^\omega f(x) g(x) \mathrm{d} x
$$
收敛

\subsubsection{级数和积分}
\paragraph{}
级数
$$
\sum_{n = 1}^\infty f(n) = f(1) + f(2) + \cdots 
$$
 和反常积分
 $$
 \int_1^{+\infty} f(x) \mathrm{d} x
 $$
 同时收敛或者同时发散.








\section{勒贝格积分}

\subsection{测度}
\subsubsection{环}
\paragraph{}
在(1.4.3小节)里提到过域是加法和乘法封闭的(加法和乘法并非传统意义上的加法和乘法),并且加法和乘法都有逆元. 这里要提到另一种结构,称为\textbf{环}.  环和域的区别在于环的乘法不一定需要有逆元,也就是环对除法不封闭. 

\paragraph{}
\textbf{环}是定义了加法和乘法的集合,但是对除法不一定封闭
\begin{enumerate}
\item 有加法的单位元,用$0$表示
\item 有乘法的单位元,用$1$表示
\item 有加法逆元,即$x$的逆元记为$-x$
\item 加法满足交换率, $x + y = y + x$.
\item 加法满足结合率, $x + (y + z) = (x + y) + z$
\item 乘法满足交换率, $x \cdot y = y \cdot x$
\item 乘法满足结合率, $x \cdot (y \cdot z) = (x \cdot y) \cdot z$
\item 加法和乘法满足分配率,即$x \cdot (y + z) = x\cdot y + x \cdot z$
\end{enumerate}
如果乘法不满足交换率的话,那么称为\textbf{非交换环}. 满足交换率的称为\textbf{交换环}.

\paragraph{}
\textbf{半环}的话就是把环的定义里去除加法逆元,也就是减法不封闭. 

\paragraph{}
直观的讲,自然数是半环,整数是环,有理数和实数是域. 并且的话,我们仍然要重复说明一下,这里的加法和乘法并非传统意义上的加法和乘法, 其意义是指两种操作或者两种映射,说成加法和减法是为了直观的理解. 

\subsubsection{集合族}
\paragraph{}
把集合的子集合组成的集合称为\textbf{集合族}. 例如说集合$S = \{1, 2, 3\}$,那么集合$\{\{1\}\}, \{\{1\}, \{2\}\}, \{\{1\}, \{2\}, \{3\}\}, \{\{1, 2\}\}, \{\{1, 2, 3\}\}, \cdots$,这些集合都可以称为\textbf{集合族}.

\subsubsection{sigma-代数}
\paragraph{}
先介绍一下\textbf{$\sigma$-环},并和差在(1.1.3小节)里有介绍,假设和集合$X$,它的一个集合系$\mathcal{A}$满足下面的运算
\begin{enumerate}
\item 如果$A_n \in \mathcal{A}, (n \in N)$,那么$\cup_{n = 1}^\infty A_n \in \mathcal{A}$
\item 如果$A_n \in \mathcal{A}, (n \in N)$,那么$\cap_{n = 1}^\infty A_n \in \mathcal{A}$
\item 如果$A, B \in \mathcal{A}$,那么$A - B \in \mathcal{A}$.
\end{enumerate}
这个集合系称为一个$\sigma$-环.  这里的环和$\sigma$-环的区别在于,环没有特别的要求可数集的并也在$\mathcal{A}$内,在(1.5节)里曾提到可数集(即和自然数集等势的集合).  也就是说,如果一个环虽然是无限集,但是集合系里每个元素(就是集合)是有限集. 那么这样的环不是$\sigma$-环.

\paragraph{}
\textbf{$\sigma$-代数}又称为\textbf{$\sigma$-域}, 是指集合$X$的集合系$\mathcal{F}$中包括其全集$X$的$\sigma$-环. 定义的话就是
\begin{enumerate}
\item 如果$A_n \in \mathcal{A}, (n \in N)$,那么$\cup_{n = 1}^\infty A_n \in \mathcal{A}$
\item 如果$A_n \in \mathcal{A}, (n \in N)$,那么$\cap_{n = 1}^\infty A_n \in \mathcal{A}$
\item 如果$A \in \mathcal{A}$,那么$\overline{A} \in \mathcal{A}$, 或者$A^c \in \mathcal{A}$.
\end{enumerate}

\subsubsection{测度}
\paragraph{}
设集合$X$中的$\sigma$-代数$\mathcal{A}$,如果一个函数$\mu$满足下面的条件
\begin{enumerate}
\item $\mu(\emptyset) = 0$
\item $0 \leq \mu \leq \infty$
\item $\mu(\cup_{i=1}^\infty A_i) = \sum_{i=1}^\infty \mu(A_i)$, 其中$A_i \in \mathcal{A}$并且$A_1, A_2, \cdots$两两不相交.
\end{enumerate}
称$\mu$这个函数为\textbf{测度}. 测度是长度,面积,体积,概率等概念的一个抽象.

\paragraph{}
把集合$X$以及$X$上的$\sigma$-代数组成的集合系$\mathcal{A}$,以及一个测度$\mu$,表示一个\textbf{测度空间}. 用记号$(X, \mathcal{A}, \mu)$来表示. $\mathcal{A}$中的元素为测度空间中的\textbf{可测集合}.

\subsubsection{测度的性质}
\begin{enumerate}
\item 如果$A_1 \subseteq A_2$,那么$\mu(A_1) \leq \mu(A_2)$
\item 对任意的$A_i \in \mathcal{A}$,$\mu(\cup_{i=1}^\infty A_i) \leq \sum_{i=1}^\infty \mu(A_i)$
\item 如果对任意的$A_n \subseteq A_{n+1}, n \in N$, 那么$\mu(\cup_{i=1}^\infty A_i) = \lim_{i\to \infty} \mu(A_i)$
\item 如果对任意的$A_n \subseteq A_{n+1}, n \in N$, 并且其中至少有一个集合的测度有限,那么$\mu(\cap_{i=1}^\infty A_i) = \lim_{i\to \infty} \mu(A_i)$
\end{enumerate}

\paragraph{}
关于测度,满足这些性质的函数都可以称为测度,比如说计数测度,就是集合的元素的个数,很明显的可以满足测度的定义. 接p;[[下来主要要讨论的一种测度是勒贝格测度. 

\subsection{勒贝格测度}

\subsubsection{集函数}
\paragraph{}
把运用的对象是集合的函数称为\textbf{集函数},如果其函数值会对应一个实数的话. 这里用$\phi(X)$来表示. 

\paragraph{}
有了之前环的定义之后,可以在环上定义这样的集函数,在$R^n$空间的子集的一个特殊的集函数,其性质是
\begin{enumerate}
\item $\phi(\emptyset) = 0$
\item $0 \leq \phi(A) \leq \infty$
\item $\phi(A \cup B) = \phi(A) + \phi(B)$, 其中$A \cap B = \emptyset$, 可加
\item 如果$A \subset B$,那么$\phi(A) \leq \phi(B)$
\end{enumerate}

\paragraph{}
因此,如果$A \subset B$,并且$\phi(A) \neq \infty$,那么
\begin{align*}
\phi( (B - A) \cup A) &= \phi(B) \\
\phi(B - A) + \phi(A) &=  \phi(B) \\
\phi(B - A) &= \phi(B) - \phi(A)
\end{align*}


\paragraph{}
因为环并未规定可数可加,而$\sigma$-环上可数可加,因此集函数如果是在$\sigma$-环上,除了拥有环上的集函数的性质外,特别得多了一条\textbf{可数可加}的性质
$$
\phi(\cup_{n=1}^\infty A_n) = \sum_{n=1}^\infty \phi(A_n)
$$
其中$A_1, A_2, \cdots, A_n$之间两两不相交. 

\subsubsection{外测度}
\paragraph{}
再定义一个集函数$\mu(X)$,不仅满足$\phi(X)$的的定义外,并且,如果集合族$\mathcal{A}$是一个$\sigma$-环,对任意的集合$A$,存在集合$F, G$,满足$F \subset A \subset G$,其中$F$为闭集,$G$为开集,那么对于任何的$\varepsilon > 0$
$$
\mu(G) - \varepsilon \leq \mu(A) \leq \mu(F) + \varepsilon
$$

\paragraph{}
定义在$\sigma$-环上的一个函数$\mu^*$,如果$E\subset \cup_{n=1}^\infty A_n$
$$
\mu^*(E) = \inf \sum_{n=1}^\infty \mu(A_n)
$$
这个函数$\mu^*$称为集合$E$对应于$\mu$的外测度.

\paragraph{}
因此,可以知道$\mu^*(A) = \mu(A)$,并且,可以知道,当$E = \cup_{n=1}^\infty E_n$时
$$
\mu^*(E) \leq \sum_{n=1}^\infty \mu^* (E_n)
$$
称这个性质为\textbf{可数次加性}.



\subsubsection{勒贝格测度}
\paragraph{}
$R^n$中的区间,指的是所有向量$\textbf{x} = (x_1, x_2, \cdots, x_n)^T$满足
$$
a_i \leq x_i \leq b_i
$$
这些向量或者称之为点组成的集合称为\textbf{区间}, 用$I$表示

\paragraph{}
现在可以开始勒贝格测度的正式定义,假设$R^n$空间中的区间$I$,存在一个集函数$m(X)$,使得
$$
m(I) = \prod_{i=1}^n (b_i - a_i)
$$
如果$E = I_1 \cup \cdots \cup I_n$,并且$I_1, I_2, \cdots, I_n$两两不相交,那么
$$
m(E) = m(I_1) + \cdots + m(I_n)
$$
这样的集函数$m(X)$称为$R^n$空间中的\textbf{勒贝格测度}. 

\paragraph{}
特别的,如果$m(X)$和它的外测度$m^*(X)$满足
$$
m(X) = m^*(X)
$$
那么称集合$X$为可测的. 如果它们不相同,就是不可测的.  可以证明对于集合$X$的$\sigma$-环$\mathcal{X}$中都是勒贝格可测的,并且集合$X$本身也是可测的. 它们组成的$\sigma$-代数,称为\textbf{勒贝格可测空间}, 其中$\mathcal{A}$的元素(某个集合),称为\textbf{勒贝格可测集合}.

\paragraph{}
勒贝格测度可以很好的定义$R^n$空间中的长度,面积,体积. 以下对于$R^n$空间中的勒贝格测度为了方便简称为测度.

\subsubsection{勒贝格测度的性质}
\paragraph{}
\textbf{1}. 单个元素的集合的测度为$0$,例如$m(\{1\}) = 1 - 1 = 0$,由(12.2.3小节)的定义. 或者$m(\{(1, 1, 2)^T\}) = (1 - 1) (1 - 1) (2 - 2) = 0$. 因此单点的测度为$0$.

\paragraph{}
\textbf{2}. 假设$E$是可数集,那么$m(E) = 0$,将$E$看成单点元素的并,$E = \{x_1\} \cup \{x_2\} \cup \cdots \cup \{x_n\}$,那么$m(E) = \sum_{n=1}^\infty 0 = 0$. 实际上这个时候$m^*(E) \leq \sum_{n=1}^\infty m^*(\{x_n\}) = 0$,所以$m^*(E) = m(E)$.

\paragraph{}
\textbf{3}. 根据勒贝格测度的性质,那么对于任何的开集或者闭集,必然是可测的.

\subsection{可测函数}
\paragraph{}
如果函数$f: E \to R^n$,对于任何的开集$V \subset R^n$,对于$f^{-1}(V) \subset E$,如果$E$是可测的,那么称这个函数为\textbf{可测函数}.

\paragraph{}
因为连续函数的值域的某个开子集的逆也是一个开区间, 所以很显然\textbf{连续函数必然是可测函数}. 

\paragraph{}
如果函数$f(x), g(x)$都是可测函数,那么$f(x) + g(x), f(x) \cdot  g(x), \max(f(x), g(x)), \min(f(x), g(x))$都是可测函数. 如果$g(x) \neq 0$,那么$\frac{f}{g}$也是可测函数. 

\paragraph{}
如果函数$f(x)$可测,那么$|f(x)|$也可测.

\paragraph{}
如果函数$f(x)$可测,$g(x)$连续,那么$g \circ f$也是可测函数. 

\subsection{勒贝格积分}

\subsubsection{指示函数}
\paragraph{}
将
$$
1_E(x) =
  \begin{cases}
    1   & \quad x \in E \\
    0  & \quad x \notin E
  \end{cases}
$$
函数$1_E(x)$称为\textbf{指示函数},或者称为\textbf{特征函数}.

\subsubsection{简单函数}
\paragraph{}
设$E$是可测集,如果$f(E)$是有限集,那么称函数$f$为\textbf{简单函数}. 因此指示函数也是一个简单函数.  简单函数可以用指示函数构造出来.

\paragraph{}
对于实函数$f(x)$又可以用简单函数逼近,也就是用指示函数逼近.  考虑这样的一个数列$\{s_n\}$
\begin{align*}
s_1 &= -1 \cdot (f(x) \leq -1)  -\frac{1}{2} (-1 <  f(x) \leq  -\frac{1}{2}) + \frac{1}{2} (\frac{1}{2} \leq f(x) < 1) + 1 \cdot (f(x) \geq 1) \\
s_2 &= -2 \cdot (f(x) \leq -2) -\frac{7}{4} (-2 < f(x) \leq -\frac{7}{4}) - \frac{6}{4} (-\frac{7}{4} < f(x) \leq -\frac{6}{4}) \\
& - \frac{5}{4}( -\frac{6}{4} < f(x) \leq -\frac{5}{4}) - \frac{4}{4} (-\frac{5}{4} < f(x) \leq -\frac{4}{4}) - \frac{3}{4} ( -\frac{4}{4} < f(x) \leq -\frac{3}{4}) \\
& - \frac{2}{4} (-\frac{3}{4} < f(x) \leq -\frac{2}{4}) - \frac{1}{4}(-\frac{2}{4} < f(x) \leq -\frac{1}{4})  \\
& +\frac{1}{4} (\frac{1}{4} \leq f(x) < \frac{2}{4}) + \frac{2}{4} (\frac{2}{4} \leq f(x) < \frac{3}{4}) + \frac{3}{4} (\frac{3}{4} \leq f(x) < \frac{4}{4}) \\
& + \frac{4}{4} (\frac{4}{4} \leq f(x) < \frac{5}{4}) + \frac{5}{4} (\frac{5}{4} \leq f(x) < \frac{6}{4}) + \frac{6}{4} (\frac{6}{4} \leq f(x) < \frac{7}{4}) \\
& + \frac{7}{4} (\frac{7}{4} \leq f(x) < \frac{8}{4}) + 2\cdot (f(x) \geq 2) \\
s_3 & = \cdots \\
& \vdots \\
s_n & = \cdots 
\end{align*}
直观的理解就是用指示函数来不断的分割函数$f(x)$的值域,以达到逼近的目的. 这也正是勒贝格积分的思想,通过分割值域以达到求积分的目的,而黎曼积分是通过分割定义域来实现的.

\subsubsection{勒贝格积分}
\paragraph{}
设
$$
s(x) = \sum_{i=1}^n c_i 1_{E_i}(x), \quad (c_i > 0)
$$
可测, 并且令
$$
I_E(s) = \sum_{i=1}^n c_i m(E\cap E_i)
$$
如果函数$f(x)$可测,并且$0 \leq s \leq f$,那么将
$$
\int_E f \mathrm{d} m := \sup I_E(s)
$$
称之为函数$f$关于测度$m$的\textbf{勒贝格积分}. 

\paragraph{}
例如对于简单函数
\begin{align*}
\int_E s\, \mathrm{d} m & = I_E(s)
\end{align*}




\section{多元函数微积分}

\subsection{多元函数的极限和连续性}
\paragraph{}
在(8.1.1节)中提到过函数的极限,主要是指一元函数的极限,而对于多元函数,实际上只需要稍做更改. 对于$R^n$空间,也就是向量空间中,定义度量$d(x, y)$之后. 对于函数$f: R^n \to R$,有一个点$\textbf{a}$, 如果对于任何的$\varepsilon > 0$,存在一个数$\delta > 0$,使得任何的$\textbf{x}$, 且$0 < d(\textbf{x}, \textbf{a}) < \delta$,满足
$$
d(f(x), A) < \varepsilon
$$
那么就说,当$\textbf{x} \to \textbf{a}$时,函数$f(\textbf{x}$的极限为$A$,表示为
$$
\lim_{d(\textbf{x}, \textbf{a}) \to 0} f(\textbf{x}) = A
$$
对于$R^n$中的度量$d(x, y)$,这里使用范数(5.1.1小节)的定义,也就是对于向量$x = (x_1, x_2, \cdots, x_n)$和$y = (y_1, y_2, \cdots, y_n)$来说
$$
\parallel \textbf{x} \parallel = \sqrt{x_1^2 + \cdots + x_n^2}
$$
并且
$$
d(\textbf{x}, \textbf{y}) = \parallel \textbf{x} - \textbf{y} \parallel = \sqrt{(x_1 - y_1)^2 + \cdots + (x_n - y_n)^2}
$$
而对于$f: R^n \to R^m$而言,实际上可以看成是多个$f_i: R^n \to R$,例如说$v_1 = f_1(\textbf{x}), v_2 = f_2(\textbf{x}), \cdots, v_m = f_m(\textbf{x})$, 可以对它们独立的讨论,因此,这里将只考察$f: R^n \to R$的映射.

\paragraph{}
在(8.2.1小节)中提到函数的连续性,类似的对于$f: R^n \to R$定义域中的某一个向量$\textbf{x}_0$,如果
$$
\lim_{d(\textbf{x}, \textbf{x}_0) \to 0} f(\textbf{x}) = f(\textbf{x}_0)
$$
那么称函数$f(\textbf{x})$在$\textbf{x}_0$这个点上\textbf{连续}.

\subsection{偏导数}
\subsubsection{全导数}
\paragraph{}
和(9.1.1小节)里类似,定义$f: R^n \to R$中的导数
$$
\lim_{d(\textbf{x}, \textbf{x}_0) \to 0} \frac{f(\textbf{x}) - f(\textbf{x}_0)}{\parallel \textbf{x} - \textbf{x}_0 \parallel}
$$
这个极限存在,称为$f(\textbf{x})$在$\textbf{x}_0$处的\textbf{全导数}. 同样的记为$f'(\textbf{x}_0)$

\subsubsection{方向导数}
\paragraph{}
设有非零向量$\textbf{v}$,如果极限
$$
\lim_{d(\textbf{x}, \textbf{x}_0) \to 0} \frac{f(\textbf{x}_0 + d(\textbf{x}, \textbf{x}_0) \textbf{v}) - f(\textbf{x}_0)}{d(\textbf{x}, \textbf{x}_0)}
$$
存在,称为函数$f(\textbf{x})$在$\textbf{x}_0$处沿着向量$\textbf{v}$处可微,这个导数称为\textbf{方向导数}. 记为$D_{\textbf{v}} f(\textbf{x}_0)$.

\subsubsection{偏导数}
\paragraph{}
如果非零向量$\textbf{v}$是一个标准基底, 见(3.3.3小节)中,将这个方向导数
$$
\lim_{d(\textbf{x}, \textbf{x}_0) \to 0} \frac{f(\textbf{x}_0 + d(\textbf{x}, \textbf{x}_0) \textbf{e}_i) - f(\textbf{x}_0)}{d(\textbf{x}, \textbf{x}_0)}
$$
称为函数$f(\textbf{x})$在$\textbf{x}_0$处关于$x_i$的\textbf{偏导数}. 记为$\frac{\partial f}{\partial x_i} (\textbf{x}_0)$.

\paragraph{}
偏导数可以理解成对于某个变量的导数,将其它的变量当成常数. 例如$f(x, y) = x^2 y + y$,那么$\frac{\partial f}{\partial x} = 2xy$

\subsubsection{梯度}
\paragraph{}
将$f: R^n \to R$在$\textbf{a}$处的各个变量的偏导数组成的向量,称为$f(\textbf{x})$在$\textbf{a}$处的\textbf{梯度}. 记为
$$
\nabla f(\textbf{a}) = \big( \frac{\partial f}{x_1}(\textbf{a}), \cdots, \frac{\partial f}{x_n}(\textbf{a}) \big)
$$

\subsection{压缩映射}
\paragraph{}
在度量空间中,对于函数$f: R^n \to R^n$,如果存在一个$0 < c < 1$使得
$$
d(f(\textbf{x}), f(\textbf{y})) \leq c \cdot d(\textbf{x}, \textbf{y})
$$
那么就称这个函数是\textbf{压缩映射}, $c$为\textbf{压缩常数}.

\paragraph{}
\textbf{压缩映射定理}\, 如果$f(\textbf{x}) = \textbf{x}$,那么称$\textbf{x}$为\textbf{不动点}. 对于$f: X \to X$,如果$X$是完备的,那么严格压缩映射必然有一个不动点. 

\subsection{反函数定理}
\paragraph{}
在(1.3.2小节)中提到了反函数,一个函数有反函数的充分必要条件是它是个一一映射(双射). 但是不是所有的函数都存在反函数,例如$\sin(x)$就没有反函数,严格上讲$\arcsin(x)$不是它的反函数,它是$\sin(x)$在$[-\frac{\pi}{2}, \frac{\pi}{2}]$区间中的反函数. 所以这里我们讨论关于函数局部可逆. 

\paragraph{}
例如函数$f(\textbf{x})$如果是连续可微(或称为连续可导)的,如果$f'(\textbf{x}_0) = \textbf{0}$,那么在点$\textbf{x}_0$的附近并不单调,但是如果$f'(\textbf{x}_0) \neq 0$,那么在这个点的附近就是严格单调的,也就是说在这个点的附近的区域是可逆的.

\paragraph{}
\textbf{反函数定理}\, 设$E$是$R^n$中的开集,如果$f: E\to R^n$, 并且$f$是连续可微的,如果在$\textbf{x}_0$点处,$f'(\textbf{x}_0)$是可逆的,那么存在一个开集$U$,$\textbf{x}_0 \in U$,以及开集$V$,$f(\textbf{x}_0) \in V$,使得函数在$U$内存在逆映射,$f^{-1}: V \to U$,并且$f^{-1}$在$f(\textbf{x}_0)$处可微,而且
$$
(f^{-1})' (f(\textbf{x}_0)) = (f'(\textbf{x}_0))^{-1}
$$
在这里简单的证明$f: E \to R$的情况下,令$f'(\textbf{x}_0) = a$,$y = f(\textbf{x}')$, 现在先证明$f'(\textbf{x})$在$\textbf{x}_0$处连续,因为$f$是连续的,所以在$\textbf{x}_0$点的一个开球$B(\textbf{x}, \textbf{r})$上,$f$有界,由(8.2.3小节)的魏尔斯特拉最大值定理. 也就是在这个开球中$f(\textbf{x})$是有限的,并且因为可微所以导数也是有限的.  根据导数的定义
$$
\lim_{d(\textbf{x}, \textbf{x}_0) \to 0} \frac{f(\textbf{x}) - f(\textbf{x}_0)}{\textbf{x} - \textbf{x}_0} = f'(\textbf{x}_0)
$$
先看左侧,因为$d(\textbf{x}, \textbf{x}_0) \to 0$时
\begin{align*}
\lim_{d(\textbf{x}, \textbf{x}_0) \to 0} \big(f(\textbf{x}) - f(\textbf{x}_0\big) & = \lim_{\textbf{x} \to \textbf{x}_0} f(\textbf{x}) - f(\textbf{x}_0) \\
& = f(\textbf{x}_0) - f(\textbf{x}_0) \\
& = 0
\end{align*}
而且
$$
\lim_{\textbf{x} \to \textbf{x}_0} (\textbf{x} - \textbf{x}_0) = 0
$$
所以是一个$\frac{0}{0}$形的,根据洛必达法则
\begin{align*}
\lim_{d(\textbf{x}, \textbf{x}_0) \to 0} \frac{f(\textbf{x}) - f(\textbf{x}_0)}{\textbf{x} - \textbf{x}_0}  & = \lim_{\textbf{x} \to \textbf{x}_0}  \frac{f'(\textbf{x})}{1} \\
& = \lim_{\textbf{x} \to \textbf{x}_0} f'(\textbf{x})
\end{align*}
因为在$\textbf{x}_0$的开球上极限必然存在(不会是无限), 所以
$$
\lim_{\textbf{x} \to \textbf{x}_0} f'(\textbf{x}) = f'(\textbf{x})
$$
所以说$f'(\textbf{x})$在$\textbf{x}_0$处连续,也就是说,存在一个$\delta > 0$,使得$0 < d(\textbf{x}, \textbf{x}_0) < \delta$,对任何的$\varepsilon > 0$,都能满足
$$
|f(\textbf{x}) - a | < \varepsilon
$$
因此,如果取$\varepsilon \leq \frac{a}{2}$时,必存在一个$\delta > 0$,换句话说,存在一个开球$B(\textbf{x}_0, \delta)$使得$|f(\textbf{x}) - a | < \frac{a}{2}$. , 构造一个函数
$$
g(\textbf{x}, y) = \textbf{x} + a^{-1} (y - f(\textbf{x})) 
$$
取偏导数
$$
\frac{\partial g}{\partial \textbf{x}} (\textbf{x}, y) =  \frac{a - f'(\textbf{x})}{a}
$$
也就是说存在一个开球$B(\textbf{x}_0, \delta)$使得
$$
|\frac{\partial g}{\partial \textbf{x}} (\textbf{x}, y)| < \frac{|a - f'(\textbf{x})|}{|a|} < \frac{1}{2}
$$
根据中值定理
$$
|g(\textbf{x}_2, y) - g(\textbf{x}_1, y)| < \frac{1}{2} |\textbf{x}_2 - \textbf{x}_1|
$$
因此$g(\textbf{X}, y)$是一个压缩映射,并且仅存在一个不动点,也就是存在一个$\textbf{x}'$使得
$$
g(\textbf{x}', y) = \textbf{x}' + \frac{y - f(\textbf{x}')}{a} = \textbf{x}'
$$
因此
$$
y = f(\textbf{x})
$$
对于每一个$y$都只有一个$\textbf{x}$使其成为不动点,因此$y$是一个一一映射,也就证明了在$\textbf{x}_0$处存在一个开球使得函数的反函数存在.

\subsection{隐函数定理}
\paragraph{}
隐函数是指$f(\textbf{x}) = 0$(注意其中的$\textbf{x}$是向量)这样的多元函数,和我们一般而言的$y = f(\textbf{x})$有区别,例如$x^2 + y^2 - 1 = 0$就是隐函数,它是一个单位圆.  如果稍微转换一下那么反函数就是一个特殊的隐函数,例如$y - f(\textbf{x}) = 0$就变成隐函数了. 隐函数定理的目的是判断一个隐函数在一定的局部内是否可以用$y = f(x)$的形式给出来.  

\paragraph{}
\textbf{隐函数定理}\, 



\subsection{重积分}




\sectionbreak

\part{傅立叶分析}

\input{c14.tex}

\section{内积空间}

\section{傅立叶级数}

\section{傅立叶变换}

\section{离散傅立叶变换}


\sectionbreak

\part{复分析基础}

\section{复函数}

\section{解析函数}

\section{全纯函数和亚纯函数}

\section{洛朗级数}

\section{留数}
\sectionbreak


\part{概率论基础}
\section{排列组合}

\paragraph{}
一个集合$S$的一个划分是指这个集合的子集$S_1, S_2, \cdots, S_n$,满足
$S = S_1 \cup S_2 \cup \cdots \cup S_n$,并且$S_i \cap S_j = \emptyset (i \neq j)$. 

\paragraph{}
另外将集合$S$的元素的数量记为$|S|$.

\subsection{计数}

\subsubsection{加法}
\paragraph{}
如果一个集合的$S$的一个划分$S_1, S_2, \cdots, S_n$,那么
$$
|S| = |S_1|+|S_2| + \cdots + |S_n|
$$
例如$S =  \{1, 2, 3, 4, 5, 6\}$,那么$|S| = |\{1\}| + |\{2, 3\}| + |\{4\}| + |\{5\}| + |\{6\}| = 1 + 2 + 1  + 1 + 1  = 6$

\subsubsection{乘法}
\paragraph{}
设$S = \{(a, b) | a \in A, b \in B\}$,并且$|A| = p, |B| = q$,则
$$
|S| = p \times q
$$

\subsubsection{减法}
设集合$A \subset S$,那么集合A的\textbf{补}记为$\overline{A}$,有时候也记为$A^c$,且$\overline{A} = S \setminus A = \{x \in S: x \notin A\} $,那么
$$
|A| = |S| - |\overline{A}|
$$

\subsubsection{除法}
设集合$S$,将$S$划分成$k$个相同数量的子集,那么,那个子集的数量为
$|S_i| =  |S| / k$, 当然前提是存在这样的划分.

\subsection{排列}

\paragraph{}
一个元素个数为$n$的集合$S$的\textbf{r排列},其中$ r \leq n $,表示为将集合$ S $中的元素组成个数为$ r $的有序对组成的集合, 例如$S = \{a, b, c\}$,那么$S$的2排列为
$$
 \{(a, b), (a, c), (b, a), (b, c), (c, a), (c, b)\}
$$
将这个r排列的个数记为$P(n , r)$,也有时候记为$A_n^r$

\paragraph{}
\textbf{定理} 对于正整数$n, r, r \leq n$,那么
$$
P(n, r) = n \times (n - 1) \times \cdots \times (n - r + 1)
$$



\subsection{组合}
\paragraph{}
如果说排列是集合$S$的有序选择,那么组合就是集合的无序选择,对于排列的r排列,那么\textbf{r组合}就表示集合$S$中元素数量为$r$的子集的集合,例如$S = \{a, b, c\}$,那么$S$的2组合为
$$
\{ \{a, b\}, \{a, c\}, \{b,c\} \}
$$
一般将r组合的数量记为${n \choose r }$,也有记为$C_n^r$

\paragraph{}
\textbf{定理} 对于正整数$n, r, r \leq n$,那么
$$
{n \choose r} = \frac{n!}{r!(n-r)!}
$$

\subsubsection{二项式系数}
\paragraph{}
如果用组合的思想,展开一个$(x + y)^n = (x+y)(x+y)\cdots(x+y)$,  


\sectionbreak

\section{概率}

\subsection{样本空间和事件}
\paragraph{}
\textbf{定义} 将一个试验的所有的可能的结果组成的集合,称为该试验的\textbf{样本空间}. 样子空间的子集称之为\textbf{事件}. 空子集称为一个\textbf{不可能事件}. 如果两个事件$E$和$F$,并且$E\cap F= \emptyset$,那么称事件$E$和$F$是\textbf{互不相容的}

\paragraph{}
例如扔一次骰子,那么样本空间就是$S = \{1, 2, 3, 4, 5, 6\}$,那么将扔出3点,或者扔出大于3点,都称之为事件,扔出7则为一个不可能事件,扔出2和扔出3为互不相容的事件.

\subsection{概率}

\paragraph{}
我们将样本空间$S$中的事件$E$发生的可能性记为$P(E)$

\subsubsection{古典概率}
\paragraph{}
\textbf{古典概率}也称为\textbf{拉普拉斯概率},古典概率的概念是现代概率论的特殊的例子,如果一个试验所有的单位事件数量有限,并且是\textbf{等可能性}的,那么事件$E$在事件空间$S$中概率
$$
P(E) =  \frac{|E|}{|S|}
$$


\subsubsection{统计概率}
\paragraph{}
\textbf{统计概率}是指,如果一个试验的某个事件,重复无数次,该事件出现频率存在一个极限值,这个值就是事件的概率,也就是
$$
P(E) = \lim_{n\to \infty } f_n(E)
$$

\subsubsection{现代概率论}
\paragraph{}
现在概率论指的建立公理化系统的概率论,对于样本空间$S$,事件$E$的概率记为$P(E)$,那么
\begin{enumerate}
\item $ 0 \leq P(E) \leq 1$
\item $P(S) = 1$
\item 对互不相容的事件$E_1, E_2, \cdots$,那么
$$
P\left(\cup_{i=1}^\infty E_i\right) = \sum_{i=1}^\infty P(E_i)
$$
\end{enumerate}





\sectionbreak

\section{条件概率}

\subsection{条件概率}

\paragraph{}
对于样本空间为$S$中的两个事件$E, F$,我们将事件$F$发生的情况下事件$E$发生的概率称为\textbf{条件概率}, 记为$P(E|F)$, 如果$P(F) > 0$,那么
$$
P(E|F) = P(EF)/P(F)
$$

\paragraph{}
条件概率常用的乘法公式
$$
P(E_1E_2E_3\cdots E_n) = P(E_1)P(E_2|E_1)P(E_3|E_1 E_2)\cdots P(E_n|E_1\cdots E_{n-1})
$$

\paragraph{}
另一个常用的公式
$$
P(E) = P(EF) + P(EF^c) = P(E|F)P(F) + P(E|F^c)P(F^c) = P(E|F)P(F) + P(E|F^c)(1-P(F^c))
$$

\subsection{全概率公式}
\paragraph{}
如果样本空间$S$中某一个划分$F_1, F_2, \cdots, F_n$,并且$E$为$S$中的事件,那么
$$
P(E) = \sum_{i=1}^n P(EF_i) = \sum_{i=1}^n P(E|F_i) P(F_i) 
$$


\subsection{贝叶斯公式}
\paragraph{}
如果样本空间$S$中某一个划分$F_1, F_2, \cdots, F_n$,并且$E$为$S$中的事件,那么
$$
P(F_j|E) = \frac{P(EF_j)}{P(E)} = \frac{P(E|F_j)P(F_j)}{\sum_{i=1}^n P(E|F_i)P(F_i)}
$$

\subsection{独立事件}
\paragraph{}
如果样本空间$S$中两个事件$E, F$,若
$$
P(EF) = P(E)P(F)
$$
那么将$E, F$是\textbf{互相独立}的事件,否则称为\textbf{互相不独立}

\sectionbreak

\section{随机变量和分布}

\subsection{随机变量}
\paragraph{}
定义在样本空间上的实值函数称之为\textbf{随机变量}


\subsubsection{分布函数}
\paragraph{}
对于随机变量$X$,定义函数$F$
$$
F(x) = P\{X \leq x \},-\infty < x < \infty
$$
称为$X$的\textbf{累积分布函数},也称为\textbf{分布函数}

\subsection{离散型随机变量}
\paragraph{}
如果随机变量$X$的可能取值是可数多个,那么这个随机变量称为\textbf{离散型}的. 定义$X$上的概率分布列为$p(x)$,那么
$$
\sum_{i=1}^\infty p(x_i) = 1
$$
离散型随机变量的分页函数F为
$$
F(a) = \sum_{x \leq a} p(x)
$$

\paragraph{}
例如,一枚硬币,抛两次,如果正反的概率都是$\frac{1}{2}$,若随机变量$X$表示,正面出现的次数,那么
$$
p(0)= \frac{1}{4}, p(1) = \frac{1}{2}, p(2) = \frac{1}{4}
$$
其分布函数为
$$
F(a) =  \begin{cases}
0 & \quad a < 0 \\
\frac{1}{4} & \quad 0 \leq a < 1 \\
\frac{3}{4} & \quad 1 \leq a < 2 \\
1 & \quad a \geq 2
\end{cases}
$$

\subsubsection{期望}
\paragraph{}
随机变量$X$的\textbf{期望} 记为$E[X]$,定义为
$$
E[X] = \sum x p(x)
$$
期望也可以称为随机变量$X$的\textbf{均值}

\subsubsection{期望的性质}
\paragraph{}
若有关于随机变量$X$的函数$g(x)$,那么这个函数的期望为
$$
E[g(X)] = \sum g(x) p(x)
$$
因此例如随机变量$X^2$的期望为
$$
E[X^2] = \sum x^2 p(x)
$$
例如
$$
E[X+1] = \sum (x+1) p(x)
$$
更通用的
$$
E[aX+b] = aE[X] + b
$$

	
\subsubsection{方差}
\paragraph{}
如果随机变量$X$的期望为$\mu$,那么$X$的\textbf{方差},记为$Var(X)$,并且
$$
Var(X) = E[(X - \mu)^2] = E[X^2] - \mu^2
$$
另外
$$
Var(aX+b) = a^2Var(X)
$$

\subsubsection{标准差}
\paragraph{}
另外定义方差的平方根称为\textbf{标准差}, 记为$SD(X)$,并且$SD(X) = \sqrt{Var(X)}$.

\paragraph{}
另外,一个常用的公式
$$
Var(aX + b) = a^2 Var(X)
$$

\subsubsection{中位数}
\paragraph{}
对于离散型随机变量,中位数指的是数据中一半的数据比它大,另一半比它小,如果处于中间的刚好有两个数,那么取它们的平均值

\subsection{伯努利随机变量和二项随机变量}
\paragraph{}
如果一次试验的结果为两个
$$
X = \begin{cases}
1 & \quad  \\
0 & \quad 
\end{cases}
$$
并且设
$$
p(0) = 1 - p 
$$
$$
p(1) = p
$$

\paragraph{}
如果重复n次这样的试验,如果随机变量$X$表示实验成功的次数,那么称$X$为参数是$(n, p)$的\textbf{二项随机变量}, 所以说伯努利随机变量是参数为$(1, p)$的二项随机变量,那么二项随机变量$(n, p)$的分布列为
$$
p(i) = { n\choose i} p^i (1-p)^{n-i} \quad i = 0, 1, \cdots, n
$$
一般将二项随机变量记为$X \sim b(n, p)$或者$X \sim B(n, p)$

\subsubsection{期望}
\paragraph{}
二项随机变量的期望可以计算出来
\begin{align*}
E[X] &= \sum_{i=0}^n i {n\choose i} p^i (1-p)^{n-i} \\
		&= \sum_{i=0}^n i \frac{n!}{i! (n-i)!} p^i (1-p)^{n-i} \\
		&= 0 + \sum_{i=1}^n  i \frac{n!}{i! (n-i)!} p^i (1-p)^{n-i} \\
		&= np \sum_{i=1}^n {(n-1)\choose (i-1)} p^{i-1} (1-p)^{n-1-(i-1)} \\
		&= np (1 + (1-p))^{n-1} \\
		&= np
\end{align*}

\paragraph{}
类似的可以计算出来$E[X^2]$
$$
E[X^2] = np((n-1)p +1) 
$$


\subsubsection{方差}
\paragraph{}
根据方差和期望的公式, 二项随机变量$X$的方差为
\begin{align*}
Var(X) &= E[X^2] - (E[X])^2 \\
		   &= np((n-1)p +1)  - n^2 p^2 \\
		   &= n^2 p^2 - np^2 + np - n^2 p^2 \\
		   &= np(1-p)
\end{align*}


\subsubsection{二项分布函数}
\paragraph{}
根据分布函数的定义,那么二项分布函数为
$$
P\{X \leq i\} = \sum_{k=0}^i {n\choose k} p^k (1-p)^{n-k}\quad i = 0, 1, \cdots, n
$$

\subsection{泊松随机变量}
\paragraph{}
如果一个二项随机变量的$n$很大,并且$p$很小,记$\lambda = np$,那么
\begin{align*}
p(i) &= {n\choose i} p^i (1-p)^{n-i} \\
	  &= {n\choose i} (\frac{\lambda}{n})^i (1 - \frac{\lambda}{n})^{n-i} \\
	  &= \frac{n(n-1)(n-2)\cdots (n-i+1)}{i!} \frac{\lambda^i}{n^i} \frac{(1-\lambda/n)^n}{(1- \lambda/n)^i} \\
	  &= \frac{n(n-1)(n-2)\cdots (n-i+1)}{n^i} \frac{\lambda^i}{i!} \frac{(1-\lambda/n)^n}{(1-\lambda/n)^i} 
\end{align*}
如果$n$很大,$p$很小,那么$(1 - \lambda/n)^i \approx 1$,以及 $\frac{n(n-1)(n-2)\cdots (n-i+1)}{n^i} \approx 1$,那么
$$
p(i) \approx   \frac{\lambda^i}{i!} (1-\lambda/n)^n
$$
令$m = - \frac{n}{\lambda}$,那么
$$
(1-\lambda/n)^n = (1+1/m)^{-m\lambda} = ((1+1/m)^m)^{-\lambda} \approx e^{-\lambda}
$$
参见\textbf{7.1.6}, 因此
$$
p(i) \approx e^{-\lambda}  \frac{\lambda^i}{i!}
$$
这不是一个非常严格的证明,只是说明二项随机变量和以下泊松随机变量的关系,也即是如果$n \to \infty$, $n$为自然数(有时候为了方便会假设自然数是从0开始的),对于随机变量$X$,$X$的取值为$0, 1, 2\cdots$,那么对于某个$\lambda > 0$,其分布列为
$$
p(i) = P{X = i} = e^{-\lambda} \frac{\lambda^i}{i!}\quad i = 0, 1, 2 \cdots 
$$
则称该随机变量为\textbf{泊松随机变量},并且可知
$$
\sum_{i=0}^\infty p(i) = e^{-\lambda} \sum_{i=0}^\infty \frac{\lambda^i}{i!} = e^{-\lambda} e^\lambda = 1
$$
其中
$$
e^\lambda = \sum_{i=0}^\infty \frac{\lambda^i}{i!}
$$
参见\textbf{7.2.5}或\textbf{9.4},一般将泊松随机变量记为$X \sim \pi (\lambda)$,或者$X \sim P(\lambda)$

\subsubsection{期望}
\paragraph{}
泊松随机变量的期望为
\begin{align*}
E(X) &= \sum_{i=0}^\infty \frac{i e^{-\lambda} \lambda^i }{i!}  \\
	    &= \lambda  e^{-\lambda} \sum_{i=1}^\infty \frac{\lambda^{i-1}}{i!} \\
	    &=  \lambda  e^{-\lambda} e^\lambda \\
	    &= \lambda
\end{align*}
类似的可以计算出来
$$
E(X^2) = \lambda(\lambda + 1)
$$

\subsubsection{方差}
\paragraph{}
由此,泊松随机变量的方差为
$$
Var(X) = E[X^2] - (E[X])^2 = \lambda^2 + \lambda - \lambda^2 = \lambda
$$


\subsection{几何随机变量}
\paragraph{}
考虑某个试验,成功的概率为$p$,那么重复这个实验直到成功,这样的随机变量$X$称为参数为$p$的\textbf{几何随机变量},一般记为$X \sim G(p)$,那么它的分布列为
$$
p(i) = (1-p)^{n-1}p
$$

\subsubsection{期望}
它的期望为
$$
E(X) = 1/p
$$
并且
$$
E(X^2) = \frac{2-p}{p^2}
$$

\subsubsection{方差}
它的方差为
$$
Var(X) = \frac{2-p}{p^2} - \frac{1}{p^2} = \frac{1-p}{p}
$$

\subsection{负二项随机变量}
\paragraph{}
考虑某个试验,成功的概率为$p$,那么重复这个实验直到成功$r$次,这样的随机变量$X$称为参数为$r, p$的\textbf{负二项随机变量},一般记为$X \sim NB(r, p)$,那么它的分布列为
$$
p(n) = {(n-1)\choose (r-1)} p^r (1-p)(n-r)\quad n \geq r
$$
所以说,如果$r=1$,那么就是几何随机变量

\subsubsection{期望}
它的期望为
$$
E(X) = r/p
$$
并且
$$
E(X^2) =\frac{r}{p}(\frac{r+1}{p} - 1)
$$

\subsubsection{方差}
它的方差为
$$
Var(X) = \frac{r(1-p)}{p^2}
$$


\subsection{超几何随机变量}
\paragraph{}
考虑一个盒子里有$N$个球,其中$m$个白球,$N-m$个黑球,从中随机取出$n$个球,令$X$表示取出来的白球数量,那么随机变量$X$的分布列为
$$
p(i) = \frac{{m\choose i} {(N-m)\choose (n-i)}}{{N \choose n}} \quad i = 0, 1, \cdots, n
$$
这样的随机变量称之为\textbf{超几何随机变量},一般记为$X \sim H(n, m, N)$
\paragraph{}
若$n=1$,那么$X$就是伯努利随机变量,如果$N$和$m$远大于$n$,则可以看成是二项随机变量

\subsubsection{期望}
\paragraph{}
$$
E(X) = \frac{nm}{N}
$$
并且
$$
E(X^2) = \frac{nm}{N} (\frac{(n-1)(m-1)}{N-1} + 1)
$$

\subsubsection{方差}
$$
Var(X) = np(1-p)(1 - \frac{n-1}{N-1})
$$


\subsection{连续型随机变量}
\paragraph{}
如果随机变量的可能取值是不可数的,那么称随机变量$X$为\textbf{连续型随机变量}

\subsubsection{概率密度}
假设一个可测集合B中,如果连续型随机变量$X$满足
$$
P\{X \in B\} = \int_B f(x) \mathrm{d} x
$$
称这个函数为\textbf{概率密度函数},简称\textbf{概率密度}.
并且有
$$
P\{X \in (-\infty, +\infty)\} = \int_{-\infty}^{+\infty} f(x) \mathrm{d} x = 1
$$

\subsubsection{期望}
\paragraph{}
类似的,连续型随机变量的期望为
$$
E[X] = \int_{-\infty}^{\infty} xf(x) \mathrm{d} x
$$
另外,如果有随机变量的函数$g(x)$,那么这个函数的期望为
$$
E[g(x)] = \int_{-\infty}^\infty g(x) f(x) \mathrm{d} x
$$

\paragraph{}
类似的
$$
E[ax+b] = aE[X] + b
$$

\subsubsection{方差}
\paragraph{}
同离散型随机变量一样,连续型随机变量的方差为
$$
Var(X) = E[X^2] - (E[X])^2
$$

\subsection{均匀随机变量}
\paragraph{}
如果一个随机变量$X$的概率密度为
$$
f(x) = \begin{cases}
\frac{1}{b-a} & \quad a \leq  x \leq b \\
0 &\quad  other
\end{cases}
$$
则称$X$在$[a, b]$区间上\textbf{均匀分布},一般记为$X \sim U[a, b]$

\subsubsection{分布函数}
\paragraph{}
它的分布函数为
$$
F(x) = \begin{cases}
0 & \quad x <  a \\
\frac{x-a}{b-a} & \quad a  \leq x  \leq b \\
1 & \quad x > b
\end{cases}
$$

\subsubsection{期望}
\paragraph{}
它的期望为
$$
E[X] = \frac{a+b}{2}
$$

\subsubsection{方差}
\paragraph{}
它的方差为
$$
Var(X) = \frac{(b-a)^2}{12}
$$

\subsection{正态随机变量}


\subsection{指数随机变量}


\subsection{矩}


\subsection{协方差}

\subsection{条件期望}

\sectionbreak

\section{多维随机变量和分布}
\sectionbreak

\section{极限定理}

\sectionbreak

\section{随机过程}

\sectionbreak

\section{马尔可夫链}
\sectionbreak

\section{平稳随机过程}
\sectionbreak

\part{数理统计}
\section{抽样}
\sectionbreak

\section{参数估计}
\sectionbreak

\section{假设检验}
\sectionbreak

\section{线性回归}

\sectionbreak

\part{离散数学}
\section{逻辑与命题}

\section{图论}

\section{树}

\sectionbreak

\part{信息论}
\section{熵}

\section{随机过程}

\section{博弈}

\sectionbreak

\part{统计力学}
\section{熵}


\sectionbreak
\part{数值计算}

\section{插值}

\section{微积分}

\section{最小二乘}

\section{随机数}

\section{最优化问题}

\section{三角插值和FFT}

\section{压缩}

\sectionbreak
\part{机器学习}

\section{图像处理}
\sectionbreak

\section{语音处理}
\sectionbreak

\section{自然语言处理}
\sectionbreak

\section{回归}
\sectionbreak 

\section{分类}
\sectionbreak 


\section{聚类}
\sectionbreak 



\part{神经网络}

\section{感知器}
\sectionbreak

\section{多层感知器}
\sectionbreak

\section{支持向量机}
\sectionbreak


\section{卷积神经网络}
\sectionbreak 

\section{循环神经网络}

\section{递归神经网络}
\sectionbreak

\section{深度神经网络}
\sectionbreak


\end{document}

