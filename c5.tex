\section{内积空间}

\subsection{赋范向量空间}
\subsubsection{范数}
\paragraph{}
范数对应的是长度的概念, 是对长度的一种推广,对于向量空间$V$中的向量$\textbf{v}$和$\textbf{w}$,定义一个\textbf{范数},用记号$\parallel \cdot \parallel$表示,它满足
\begin{enumerate}
\item $\parallel \textbf{v} \parallel = 0$, 当且仅当$\textbf{v} = \textbf{0}$
\item $\parallel \textbf{v} \parallel \geq 0$
\item $\parallel  \alpha \textbf{v} \parallel = |\alpha| \parallel \textbf{v} \parallel$
\item $\parallel \textbf{v} + \textbf{w} \parallel \leq \parallel \textbf{v} \parallel + \parallel \textbf{w} \parallel$, 三角不等式
\end{enumerate}
另外,如果$\textbf{v} \neq \textbf{0}$时,$\parallel \textbf{v} \parallel = 0$,那么这个时候称之为\textbf{半范数}. 

\paragraph{}
定义了范数的向量空间,称之为\textbf{赋范向量空间}.

\subsection{内积空间}
\paragraph{}
对于向量空间$V$,其中的两个元素$\textbf{v}_1, \textbf{v}_2$, 增加一种运算将两个向量映射成一个实数,这种运算,称之为\textbf{内积}, 记为$\langle \textbf{v}_1, \textbf{v}_2 \rangle$, 它有如下的性质
\begin{enumerate}
\item $\langle \textbf{x}, \textbf{y} \rangle \geq 0$, 仅当$\textbf{x}$时, 内积为0
\item $ \langle \textbf{x}, \textbf{y} \rangle = \langle \textbf{y}, \textbf{x} \rangle$
\item $\langle \alpha \textbf{x} + \beta \textbf{y}, \textbf{z} \rangle = \alpha \langle \textbf{x}, \textbf{z} \rangle + \beta \langle \textbf{y}, \textbf{z} \rangle $
\end{enumerate}
带有内积运算的向量空间,称之为\textbf{内积空间}. 例如,欧氏空间$R^n$中,将内积定义为
$$
\langle \textbf{x}, \textbf{y} \rangle = \textbf{x}^T \textbf{y}
$$

\subsubsection{内积空间中的范数}
\paragraph{}
对于内积空间$V$中的向量$\textbf{v}$,定义一个$\textbf{v}$的\textbf{范数}
$$
\parallel \textbf{v} \parallel = \sqrt{\langle \textbf{v} , \textbf{v} \rangle}
$$
因此说,内积空间可以看成是一个赋范向量空间. 再定义内积空间中两个向量$\textbf{v}_1, \textbf{v}_2$的\textbf{距离}(在分析学中称为度量)为
$$
\parallel \textbf{v}_1 - \textbf{v}_2 \parallel 
$$
将
$$
\textbf{u} = \frac{\textbf{v}}{\parallel \textbf{v} \parallel}
$$
称之为\textbf{单位向量}.


\subsubsection{投影}
定义一个向量空间$V$中两个向量$\textbf{v}, \textbf{w}$, $\textbf{w} \neq \textbf{0}$, 那么从$\textbf{v}$到$\textbf{w}$的\textbf{正交投影}为一个标量
$$
\alpha = \frac{\langle \textbf{v}, \textbf{w} \rangle  }{\parallel \textbf{w} \parallel }
$$
\textbf{向量投影}为
\begin{align*}
\textbf{u} & =  \frac{\langle \textbf{v}, \textbf{w} \rangle  }{\parallel \textbf{w} \parallel } (\frac{\textbf{w}}{\parallel \textbf{w} \parallel}) \\
                 & = \frac{\langle \textbf{v}, \textbf{w}\rangle}{\langle \textbf{w}, \textbf{w} \rangle} \textbf{w}
\end{align*}
由此,可以知道两个向量之间的夹角$\theta$, 
$$
\cos{\theta} =   \frac{\langle \textbf{v}, \textbf{w} \rangle  }{\parallel \textbf{w} \parallel \parallel \textbf{v} \parallel}
$$

\subsection{正交}
\paragraph{}
如果内积空间$V$中,两个向量$\textbf{v}, \textbf{w}$的内积为0,称它们\textbf{正交}, 对应了垂直的概念, 即$\langle \textbf{v}, \textbf{w} \rangle = 0$, 在上一节有提到向量之间的夹角$\theta$满足
$$
\cos{\theta} =  \frac{\langle \textbf{v}, \textbf{w} \rangle  }{\parallel \textbf{w} \parallel \parallel \textbf{v} \parallel}
$$
注意这里假设$\textbf{v}, \textbf{w}$都是非零向量,所以$\parallel \textbf{v} \parallel > 0, \parallel \textbf{w} \parallel > 0 $,  那么如果它们夹角为$90^\circ$,则$\langle \textbf{v}, \textbf{w} \rangle = 0$.

\paragraph{}
如果有集合向量空间$V$中,有$S = \{\textbf{v}_1, \textbf{v}_2, \cdots, \textbf{v}_n\}, S \subset V$, 并且其中任意两个不同的向量$\langle \textbf{v}_i, \textbf{v}_j \rangle = 0$, 称集合$S$为一个\textbf{正交集}. 特别的,如果集合$S$中的各个元素为单位向量,单位向量是指
$$
\textbf{u} = \frac{\textbf{v}}{\parallel \textbf{v} \parallel}
$$
那么这样的集合称为\textbf{规范正交集}. 如果$V = Span(\{\textbf{u}_1, \textbf{u}_2, \cdots, \textbf{u}_n\})$,其中的$\textbf{u}_i$为单位向量,称这样的一组集合为\textbf{规范正交基}.

\subsection{正交子空间}
\paragraph{}
对于内积空间$U$中,两上子空间$V, W$,如果对于任何的$\textbf{u} \in V, \textbf{w} \in W$,都有$\langle \textbf{u}, \textbf{w} \rangle = 0$,称$V$正交于$W$,记为$V\bot W$.  对于若对于所有与空间$V$正交的向量组成的集合记为$V^\bot$,称为$V$的\textbf{正交补}.

\subsection{零空间与列空间}
在(3.2节)中提到过\textbf{零空间}, 把由几个向量线性生成的空间, 将这几个向量组成一个矩阵的话,如果$A\textbf{x} = 0$的非零解组成的集合,它满足空间的定义(对加法和标量乘法封闭),称之为\textbf{零空间}, 记为$N(A)$.  把这几个向量线性生成的空间称为$A$的列空间, 记为$R(A)$. 行空间表示为$R(A^T)$. 由正交的定义,如果$A\textbf{x} = 0$,那么其实就是$N(A) \bot R(A^T)$,就是行零空间和行空间正交的.  因此可以说
$$
N(A^T)\bot R(A), N(A)\bot R(A^T)
$$
同时
$$
N(A) = R(A^T)^\bot, N(A^T) = R(A)^\bot
$$
这里不做证明.

\subsection{最小二乘问题}
\paragraph{}
最小二乘问题在机器学习中有重要的应用,它可以用来拟合统计数据,最简单的一种就是拟合一条直线,使用得统计数据的各个点与拟合的直接的误差最小. 假设一条直线
$$
y = a + bx
$$
我们希望这种直线可以使得己知的数据点的误差最小,那么$a, b$相当于未知数, 假设有数据点$(p_1, p_1'), (p_2, p_2'), \cdots, (p_n, p_n')$.  则
\begin{align*}
p_1' & = a + b p_1 \\
p_2' & = a + b p_2 \\
        & \vdots \\
p_n' & = a + b p_n 
\end{align*}
如果将$\textbf{x} = (a, b)^T$,即将$a, b$看成未知量,为了方便理解用向量$\textbf{x}$表示,则这个方程可以表示为
$$
\begin{bmatrix}
1 & p_1 \\
1 & p_2 \\
   & \vdots \\
1 & p_n 
\end{bmatrix} \cdot \textbf{x} = (p_1', p_2', \cdots, p_n')^T
$$
这是一个$2xn$的矩阵,一般而言$n > 2$时这个线性方程组无解. 将这个方程组简化表示成$A\textbf{x} = \textbf{b}$,也就是要求这个方程组的解,如果不存在解的话,那么退而求使得$\parallel \textbf{b} - A\textbf{x} \parallel$最小的解. 为了方便令$r(\textbf{x}) = \textbf{b} - A\textbf{x}$. 则我们知道$r(\textbf{x}) \bot R(A)$(因为从向量的角度, $\textbf{b} - \textbf{a}$是两个向量组成的三角形的第三边,要使第三边最小,那就是第三边和$\textbf{a}$垂直),也就是$r(\textbf{x}) \in N(A^T)$, 因此$A^T r(\textbf{x}) = A^T (\textbf{b} - A\textbf{x}) = 0$
那么
$$
A^T A \textbf{x} =  A^T\textbf{b}
$$
的解就是我们退而求最小误差的解.

\paragraph{}
例个特殊的例子有数据点$(1,1), (2,2)$,要拟合一条直线,使得结果的误差最小,那么看成一个方程组
$$
\begin{bmatrix}
1 & 1 \\
1 & 2
\end{bmatrix} \textbf{x} = (1, 2)^T
$$
因此,换成求解
$$
\begin{bmatrix}
1 & 1 \\
1 & 2
\end{bmatrix}^T \begin{bmatrix}
1 & 1 \\
1 & 2
\end{bmatrix} \textbf{x} = \begin{bmatrix}
1 & 1 \\
1 & 2
\end{bmatrix}^T (1, 2)^T
$$
也就是求方程组
$$
\begin{bmatrix}
2 & 3 \\
3 & 5
\end{bmatrix} \textbf{x} = (3, 5)^T
$$
求得解为$(0, 1)^T$,就就是直线$y = 0 + x$即$y = x$,这条直线也正好穿过这两个点.

\paragraph{}
再举一个复杂的例子,如果有3个数据点$(1, 1), (1, 2), (2, 1)$,那么
$$
A = \begin{bmatrix}
1 & 1 \\
1 & 1 \\
1 & 2 
\end{bmatrix}, \textbf{b} =  \begin{bmatrix}
1 \\
2 \\
1 
\end{bmatrix},
$$
转换成求
$$
 \begin{bmatrix}
3& 4 \\
4 & 6
\end{bmatrix} \textbf{x} = (4, 5)^T
$$
的解,其解为$(2, -1/2)$. 因此直线为$y = 2 - \frac{1}{2} x$

\paragraph{}
特别要注意的是,这里只举了拟合一条直线的方法,也可以用最小二乘来拟合曲线,这个将在后边提到. 









