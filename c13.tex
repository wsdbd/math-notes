\section{多元函数微积分}

\subsection{多元函数的极限和连续性}
\paragraph{}
在(8.1.1节)中提到过函数的极限,主要是指一元函数的极限,而对于多元函数,实际上只需要稍做更改. 对于$R^n$空间,也就是向量空间中,定义度量$d(x, y)$之后. 对于函数$f: R^n \to R$,有一个点$\textbf{a}$, 如果对于任何的$\varepsilon > 0$,存在一个数$\delta > 0$,使得任何的$\textbf{x}$, 且$0 < d(\textbf{x}, \textbf{a}) < \delta$,满足
$$
d(f(x), A) < \varepsilon
$$
那么就说,当$\textbf{x} \to \textbf{a}$时,函数$f(\textbf{x}$的极限为$A$,表示为
$$
\lim_{d(\textbf{x}, \textbf{a}) \to 0} f(\textbf{x}) = A
$$
对于$R^n$中的度量$d(x, y)$,这里使用范数(5.1.1小节)的定义,也就是对于向量$x = (x_1, x_2, \cdots, x_n)$和$y = (y_1, y_2, \cdots, y_n)$来说
$$
\parallel \textbf{x} \parallel = \sqrt{x_1^2 + \cdots + x_n^2}
$$
并且
$$
d(\textbf{x}, \textbf{y}) = \parallel \textbf{x} - \textbf{y} \parallel = \sqrt{(x_1 - y_1)^2 + \cdots + (x_n - y_n)^2}
$$
而对于$f: R^n \to R^m$而言,实际上可以看成是多个$f_i: R^n \to R$,例如说$v_1 = f_1(\textbf{x}), v_2 = f_2(\textbf{x}), \cdots, v_m = f_m(\textbf{x})$, 可以对它们独立的讨论,因此,这里将只考察$f: R^n \to R$的映射.

\paragraph{}
在(8.2.1小节)中提到函数的连续性,类似的对于$f: R^n \to R$定义域中的某一个向量$\textbf{x}_0$,如果
$$
\lim_{d(\textbf{x}, \textbf{x}_0) \to 0} f(\textbf{x}) = f(\textbf{x}_0)
$$
那么称函数$f(\textbf{x})$在$\textbf{x}_0$这个点上\textbf{连续}.

\subsection{偏导数}
\subsubsection{全导数}
\paragraph{}
和(9.1.1小节)里类似,定义$f: R^n \to R$中的导数
$$
\lim_{d(\textbf{x}, \textbf{x}_0) \to 0} \frac{f(\textbf{x}) - f(\textbf{x}_0)}{\parallel \textbf{x} - \textbf{x}_0 \parallel}
$$
这个极限存在,称为$f(\textbf{x})$在$\textbf{x}_0$处的\textbf{全导数}. 同样的记为$f'(\textbf{x}_0)$

\subsubsection{方向导数}
\paragraph{}
设有非零向量$\textbf{v}$,如果极限
$$
\lim_{d(\textbf{x}, \textbf{x}_0) \to 0} \frac{f(\textbf{x}_0 + d(\textbf{x}, \textbf{x}_0) \textbf{v}) - f(\textbf{x}_0)}{d(\textbf{x}, \textbf{x}_0)}
$$
存在,称为函数$f(\textbf{x})$在$\textbf{x}_0$处沿着向量$\textbf{v}$处可微,这个导数称为\textbf{方向导数}. 记为$D_{\textbf{v}} f(\textbf{x}_0)$.

\subsubsection{偏导数}
\paragraph{}
如果非零向量$\textbf{v}$是一个标准基底, 见(3.3.3小节)中,将这个方向导数
$$
\lim_{d(\textbf{x}, \textbf{x}_0) \to 0} \frac{f(\textbf{x}_0 + d(\textbf{x}, \textbf{x}_0) \textbf{e}_i) - f(\textbf{x}_0)}{d(\textbf{x}, \textbf{x}_0)}
$$
称为函数$f(\textbf{x})$在$\textbf{x}_0$处关于$x_i$的\textbf{偏导数}. 记为$\frac{\partial f}{\partial x_i} (\textbf{x}_0)$.

\paragraph{}
偏导数可以理解成对于某个变量的导数,将其它的变量当成常数. 例如$f(x, y) = x^2 y + y$,那么$\frac{\partial f}{\partial x} = 2xy$

\subsubsection{梯度}
\paragraph{}
将$f: R^n \to R$在$\textbf{a}$处的各个变量的偏导数组成的向量,称为$f(\textbf{x})$在$\textbf{a}$处的\textbf{梯度}. 记为
$$
\nabla f(\textbf{a}) = \big( \frac{\partial f}{x_1}(\textbf{a}), \cdots, \frac{\partial f}{x_n}(\textbf{a}) \big)
$$

\subsection{压缩映射}
\paragraph{}
在度量空间中,对于函数$f: R^n \to R^n$,如果存在一个$0 < c < 1$使得
$$
d(f(\textbf{x}), f(\textbf{y})) \leq c \cdot d(\textbf{x}, \textbf{y})
$$
那么就称这个函数是\textbf{压缩映射}, $c$为\textbf{压缩常数}.

\paragraph{}
\textbf{压缩映射定理}\, 如果$f(\textbf{x}) = \textbf{x}$,那么称$\textbf{x}$为\textbf{不动点}. 对于$f: X \to X$,如果$X$是完备的,那么严格压缩映射必然有一个不动点. 

\subsection{反函数定理}
\paragraph{}
在(1.3.2小节)中提到了反函数,一个函数有反函数的充分必要条件是它是个一一映射(双射). 但是不是所有的函数都存在反函数,例如$\sin(x)$就没有反函数,严格上讲$\arcsin(x)$不是它的反函数,它是$\sin(x)$在$[-\frac{\pi}{2}, \frac{\pi}{2}]$区间中的反函数. 所以这里我们讨论关于函数局部可逆. 

\paragraph{}
例如函数$f(\textbf{x})$如果是连续可微(或称为连续可导)的,如果$f'(\textbf{x}_0) = \textbf{0}$,那么在点$\textbf{x}_0$的附近并不单调,但是如果$f'(\textbf{x}_0) \neq 0$,那么在这个点的附近就是严格单调的,也就是说在这个点的附近的区域是可逆的.

\paragraph{}
\textbf{反函数定理}\, 设$E$是$R^n$中的开集,如果$f: E\to R^n$, 并且$f$是连续可微的,如果在$\textbf{x}_0$点处,$f'(\textbf{x}_0)$是可逆的,那么存在一个开集$U$,$\textbf{x}_0 \in U$,以及开集$V$,$f(\textbf{x}_0) \in V$,使得函数在$U$内存在逆映射,$f^{-1}: V \to U$,并且$f^{-1}$在$f(\textbf{x}_0)$处可微,而且
$$
(f^{-1})' (f(\textbf{x}_0)) = (f'(\textbf{x}_0))^{-1}
$$
在这里简单的证明$f: E \to R$的情况下,令$f'(\textbf{x}_0) = a$,$y = f(\textbf{x}')$, 现在先证明$f'(\textbf{x})$在$\textbf{x}_0$处连续,因为$f$是连续的,所以在$\textbf{x}_0$点的一个开球$B(\textbf{x}, \textbf{r})$上,$f$有界,由(8.2.3小节)的魏尔斯特拉最大值定理. 也就是在这个开球中$f(\textbf{x})$是有限的,并且因为可微所以导数也是有限的.  根据导数的定义
$$
\lim_{d(\textbf{x}, \textbf{x}_0) \to 0} \frac{f(\textbf{x}) - f(\textbf{x}_0)}{\textbf{x} - \textbf{x}_0} = f'(\textbf{x}_0)
$$
先看左侧,因为$d(\textbf{x}, \textbf{x}_0) \to 0$时
\begin{align*}
\lim_{d(\textbf{x}, \textbf{x}_0) \to 0} \big(f(\textbf{x}) - f(\textbf{x}_0\big) & = \lim_{\textbf{x} \to \textbf{x}_0} f(\textbf{x}) - f(\textbf{x}_0) \\
& = f(\textbf{x}_0) - f(\textbf{x}_0) \\
& = 0
\end{align*}
而且
$$
\lim_{\textbf{x} \to \textbf{x}_0} (\textbf{x} - \textbf{x}_0) = 0
$$
所以是一个$\frac{0}{0}$形的,根据洛必达法则
\begin{align*}
\lim_{d(\textbf{x}, \textbf{x}_0) \to 0} \frac{f(\textbf{x}) - f(\textbf{x}_0)}{\textbf{x} - \textbf{x}_0}  & = \lim_{\textbf{x} \to \textbf{x}_0}  \frac{f'(\textbf{x})}{1} \\
& = \lim_{\textbf{x} \to \textbf{x}_0} f'(\textbf{x})
\end{align*}
因为在$\textbf{x}_0$的开球上极限必然存在(不会是无限), 所以
$$
\lim_{\textbf{x} \to \textbf{x}_0} f'(\textbf{x}) = f'(\textbf{x})
$$
所以说$f'(\textbf{x})$在$\textbf{x}_0$处连续,也就是说,存在一个$\delta > 0$,使得$0 < d(\textbf{x}, \textbf{x}_0) < \delta$,对任何的$\varepsilon > 0$,都能满足
$$
|f(\textbf{x}) - a | < \varepsilon
$$
因此,如果取$\varepsilon \leq \frac{a}{2}$时,必存在一个$\delta > 0$,换句话说,存在一个开球$B(\textbf{x}_0, \delta)$使得$|f(\textbf{x}) - a | < \frac{a}{2}$. , 构造一个函数
$$
g(\textbf{x}, y) = \textbf{x} + a^{-1} (y - f(\textbf{x})) 
$$
取偏导数
$$
\frac{\partial g}{\partial \textbf{x}} (\textbf{x}, y) =  \frac{a - f'(\textbf{x})}{a}
$$
也就是说存在一个开球$B(\textbf{x}_0, \delta)$使得
$$
|\frac{\partial g}{\partial \textbf{x}} (\textbf{x}, y)| < \frac{|a - f'(\textbf{x})|}{|a|} < \frac{1}{2}
$$
根据中值定理
$$
|g(\textbf{x}_2, y) - g(\textbf{x}_1, y)| < \frac{1}{2} |\textbf{x}_2 - \textbf{x}_1|
$$
因此$g(\textbf{X}, y)$是一个压缩映射,并且仅存在一个不动点,也就是存在一个$\textbf{x}'$使得
$$
g(\textbf{x}', y) = \textbf{x}' + \frac{y - f(\textbf{x}')}{a} = \textbf{x}'
$$
因此
$$
y = f(\textbf{x})
$$
对于每一个$y$都只有一个$\textbf{x}$使其成为不动点,因此$y$是一个一一映射,也就证明了在$\textbf{x}_0$处存在一个开球使得函数的反函数存在.

\subsection{隐函数定理}
\paragraph{}
隐函数是指$f(\textbf{x}) = 0$(注意其中的$\textbf{x}$是向量)这样的多元函数,和我们一般而言的$y = f(\textbf{x})$有区别,例如$x^2 + y^2 - 1 = 0$就是隐函数,它是一个单位圆.  如果稍微转换一下那么反函数就是一个特殊的隐函数,例如$y - f(\textbf{x}) = 0$就变成隐函数了. 隐函数定理的目的是判断一个隐函数在一定的局部内是否可以用$y = f(x)$的形式给出来.  

\paragraph{}
\textbf{隐函数定理}\, 



\subsection{重积分}


