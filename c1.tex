\section{集合}

\paragraph{}
集合是数学的一个基础概念. 大部分数学分支都可以用集合论的语言来方便的描述. 集合通俗的讲就是一些具有类似性质的事物组成的整体. 

\subsection{符号以及集合的初等运算}

\subsubsection{符号}
\paragraph{}
一般将集合记为$\{...\}$, 如果集全只有一个元素$a$,可以将集合$\{a\}$简单的记为$a$.
\paragraph{}
集合$\{1, 2\}$是无序的,它和$\{2, 1\}$是同一个集合,有时需要用到有序的数对,因此为了方便引入了$(x, y)$来表示有序对,假设两个集合$A$和$B$,则先后从$A$和$B$选取一个元素,产生了一组有序的元素,这些元素组成的集合$M$,例如$a \in A$并且$b \in B$,以及$(a, b) \in M$, 有序对用集合可定义为:
$$
(a, b) := \{\{a\},\{a, b\}\}
$$
通常用符号$:=$来表示定义.

\subsubsection{包含}
\paragraph{}
假设$x$为集合$X$的一个元素,用符号记为:
$$
x \in X
$$, 相反的,如果$y$不是集合$X$的元素,记为$y \notin X$.
假设两个集合$X$和$Y$, 它们的元素相同, 记为:
$$
X = Y
$$
假设集合$A$是集合$B$的\textbf{子集},即集合$A$中的元素都是集合$B$中的元素, 记为
$$
A \subset B
$$
如果$A \subset B$并且$A \neq B$, 则称$A$为$B$的\textbf{真子集}. 另外, 一般将\textbf{空集}记为$\emptyset$

\subsubsection{集合的初等运算}
\paragraph{}
\textbf{并} \:  将集合$A$和$B$的所有元素组成的集合$M$称为$A$和$B$的并集, 记为:
$$
A \cup B := \{x\in M | (x \in A) \lor (x \in B)\}
$$

\paragraph{}
\textbf{交} \:  假设有集合$A$和$B$, 如果元素$x$同时是$A$和$B$的元素, 而这些元素组成的集合$M$称为$A$和$B$的交集, 记为:
$$
A \cap B := \{ x \in M | (x \in A) \land (x \in B) \}
$$

\paragraph{}
\textbf{差} \:  假设有集合$A$和$B$, 如果元素$a$是$A$的元素, 并且$a$不是$B$的元素,这些元素组成的集合$M$称为$A$和$B$的差, 记为:
$$
A - B := \{ x \in M | (x \in A) \land (x \ni B) \}
$$
有时候差也用$A\setminus B$表示
$$
A \setminus B := \{ x \in M | (x \in A) \land (x \ni B) \}
$$

\paragraph{}
\textbf{对称差}\: 假设有集合$A$和$B$, 如果元素$c$只属于$A$或者只属于$B$,定义这样元素组成的集合为$A$和$B$的对称差记为
$$
A\Delta B = (A - B) \cup (B - A)
$$


\paragraph{}
\textbf{补} \:  假设有集合$X$, $A$为它的子集,则$A$在集合$X$中的补集, 表示所有的$x$是$X$中的元素,但不是$A$的元素, 记为$A^c$


\paragraph{}
\textbf{直积(笛卡尔积)} \:  假设两个集合$X$和$Y$, 分别从$X$和$Y$中取一个元素, 组成一个有序对, 称之为集合的直积. 记为:
$$
X \times Y := \{(x, y) | (x \in X) \land (y \in Y) \}
$$
例如,$R$可以表示为数轴上的点,也就是一条直线,面$R \times R$简记为$R^2$,就可以表示一个平面,而$R \times R \times R$简记为$R^3$,就可以表示为一个三维的空间了.


\subsection{公理化集合论}

\paragraph{}
1870年代由康托尔(Georg Cantor)和理察·戴德金(Richard Dedekind)提出的朴素集合论, 而朴素集合论定义并不严格,例如罗素悖论: 由所有不包含集合自身的集合所构成的集合,因此数学家们开始研究更严格的理论.  目前广泛认同的一个公理化的集合论是策梅洛-弗兰克尔集合论(Zermelo-Fraenkel), 并且加上选择公理,这一套公理称为ZFC. 

\begin{enumerate}
\item \textbf{外延公理}  \: 两个集合相同, 当且仅当它们的元素相同
\item \textbf{分类公理}  \: 给出任何的集合$S$以及命题$P(x)$, 存在一个\textbf{子集}包含使命题$P(x)$成立的元素. 即子集存在.
\item \textbf{配对公理} \:  对于集合$X, Y$, 存在一个集合$Z = \{X, Y\}$, 即存在一个集合, 它的元素是集合$X$和$Y$.
\item \textbf{并集公理} \: 对于一个集合的集合$S$, 存在一个并集, 使它的元素是$S$的元素的元素.
\item \textbf{空集公理} \: 存在一个不包含任何元素的集合, 称为\textbf{空集}, 记为$\{\}$.
\item \textbf{无穷公理} \: 存在着集合$S$, 空集$\{\}$为其中的一个元素,且对于任何元素$x$, $x \cup \{x\}$也是其元素.
\item \textbf{替代公理} \: 假设$F(x, y)$是一个命题,它使得集体$X$中的任何的元素$x_0$, 存在一个元素$y_0$, 使得$F(x, y)$成立, 而这些$y$所组成的集合存在.
\item \textbf{幂集公理} \: 假设集合$X$, 则存在着一个集合, 它的元素为$X$的一切子集.
\item \textbf{正规公理} \: 非空集合$X$中存在一个元素$x$, 它与$X$本身的交集为空集. 
\paragraph{}
\item \textbf{选择公理} \: 对于任意多个非空集组成的族, 存在一个集合$C$, 其元素为族中每个集合$X$的一个元素, 即$C \cap X = \{x\}$. 
\end{enumerate}

\subsection{映射}
\paragraph{}
对于集合$X$和$Y$,将某种对应关系$f$使每一个$x \in X$都有一个$y \in Y$与之对应. 这种对应关系称之为\textbf{映射}, \textbf{映射}也称之为\textbf{函数}. 其中$X$将为函数的\textbf{定义域}, $Y$称为函数的\textbf{到达域}. 而
$$
f(X) := \{ y \in Y | (x \in X) \land (y = f(x)) \}
$$
称为函数的\textbf{值域}, 注意到,$Y$是到达域,而且$f(X) \subset Y$. 
另外,也常用这些符号来表示:
$$
f: X \to Y, X \stackrel{f}{\to} Y
$$

\subsubsection{映射的分类}
\paragraph{}
\textbf{单射} \: 指的是将不同的变量映射到不同的值,也就是说,如果$x_1 \neq x_2$则$f(x_1) \neq f(x_2)$

\paragraph{}
\textbf{满射} \: 指的是值域就是到达域,这意味着,是否满射和选取的到达域其实是有关系的.

\paragraph{}
\textbf{双射} \:  即是单射又是满射的映射称为\textbf{双射}, 也称之为\textbf{一一映射}.

\subsubsection{反函数}
\paragraph{}
既然有$f: X \to Y$,那么应该也存在一个集合$Y$到集合$X$的对应. 根据函数的定义,每一个$x$只有一个$y$与之对应,所以反函数不一定存在,考虑仅单射而非满射的情况下,那么必然存在一些元素$y \in Y$,无法映射到$X$之中,而考虑仅满射而非单射的情况下,那么必然存在一些元素$y \in Y$,存在多个$x \in X$, 不满足函数的定义,所以有反函数的充分必要条件是双射.  将函数$f$的反函数记为$f^{-1}$,也称函数$f$\textbf{可逆}.

\subsubsection{函数的复合}
\paragraph{}
定义一种函数的复合运算,若有$f: X \to Y$,$g: Y \to Z$,将函数的复合记为:
$$
f \circ g := f(g(x))
$$

\subsection{实数与复数}

\subsubsection{有序集}
\paragraph{}
定义一种关系,它的符号是$<$, 如果一个集合$S$的元素$x, y$,它至少在下面的其中一个关系式中成立,
$$ 
x < y, \, x = y, \,  y < x
$$
并且,如果$x < y, y < z$其中,$x,y,z \in S$,则$x < z$

\paragraph{}
如果一个集合,定义了一个序的关系,如果是部分元素满足这种关系,称之为\textbf{偏序集}. 如果所有的元素都满足这种关系的,称之为\textbf{全序集}. 例如复数就是一个偏序集,实数就是全序集. 以下为了方便通常把全序集称为有序集.

\paragraph{}
如果$S$是一个有序集,$E \subset S$,如果存在$b \in S$使得任意的$x \in E$,满足$x <= b$,则称$E$有\textbf{上界},$b$为$E$的一个上界,若对于任何的$c \in S$并且$c < b$,$c$不是$E$的上界,则我们称$b$为$E$的上确界,也称之为最小上界. 同样的可以定义\textbf{下界},\textbf{下确界}. 上确界和下确界用符号表示为
\begin{align*}
b &= supE  \\
a &= infE  
\end{align*}



\subsubsection{归纳集}
\paragraph{}
 如果有一个集合$S$,如果$a \in S$,并且$a + 1 \in S$,则称这样的集合为\textbf{归纳集}.  这里将包括$0$的最小归纳集,称为\textbf{自然数},使用符号$N$表示(为了方便,我们将0也视为自然数).  如果我们将$\emptyset$记为0,而$\emptyset \cup \{\emptyset\} = \{\emptyset\}$记为1,2就可以表示成$ 2 := \{0, 1\} = \{\emptyset, \{\emptyset\}\}, \cdots, n + 1 := \{0, 1, \dotsc, n\}$


\subsubsection{域}
\paragraph{}
将定义了\textbf{加法}和\textbf{乘法}的集合将之为\textbf{域}. 

\paragraph{}
\textbf{加法} \: 用符号$+$表示,对于集合$S$, 有$x, y, z \in S$它满足以下的条件
\begin{enumerate}[itemindent=2em]
\item 有0元,称0为单位元
\item $x + y \in S$
\item  $x + y = y + x$,即满足交换率.
\item $x + (y + z) = (x + y) + z$, 满足结合率.
\item $x$有\textbf{逆元}记为$-x$,且$x + (-x) = 0$.
\end{enumerate}

\paragraph{}
\textbf{乘法} \: 用符号$\cdot$表示,对于集合$S$, 有$x, y, z \in S$它满足以下的条件
\begin{enumerate}[itemindent=2em]
\item 有单位元1
\item $x \cdot y \in S$
\item  $x \cdot y = y \cdot x$,即满足交换率.
\item $x \cdot (y \cdot z) = (x \cdot y) \cdot z$, 满足结合率.
\item 对$x \in S$,且$x \neq 0$,有\textbf{逆元}记为$x^{-1}$,且$x \cdot x^{-1} = 1$.
\end{enumerate}
需要说明的是,这里的加法和乘法并非传统意义上的加法和乘法, 其意义是指两种操作或者两种映射,说成加法和减法是为了直观的理解. 

\paragraph{}
同时,对于加法和乘法,又满足分配率,即
$$
x \cdot (y + z) = x \cdot y + x \cdot z
$$

\subsubsection{整数}
\paragraph{}
\textbf{整数}是包含自然数以及它们的负元的集合,用符号$Z$表示. 

\paragraph{}
\textbf{算术基本定理}\: 对于任何$n > 1$的整数,都可以表示成素数的积
$$
n = p_1 \cdots p_i
$$

\subsubsection{实数域}
\paragraph{}
先介绍一下实数集的一个子集\textbf{有理数}. 整数的商$a/b, b \neq 0$称之为\textbf{有理数},用符号$Q$表示. \textbf{实数域}是这样的一个\textbf{域}不仅有加法和乘法,并且实数集是一个\textbf{全序集},另外这个实数集满足\textbf{完备性}即\textbf{最小上界公理},这样的实数集记为$R$.

\paragraph{}
\textbf{最小上界公理}\:  令$A \subset R$,并且$A \neq \emptyset$,如果$A$有上界,则$A$有最小上界.

\paragraph{}
有了最小上界公理,就可以把有理数$Q$扩充为实数$R$,令$S \subset Q$,且$S = \{x| x < \sqrt{2}\}$, 由有理数的定义可以知道$\sqrt{2}$不是有理数,所以集合$S$在有理数内有上界, 但是没有上确界. 也就是说有理数中间是有缝隙, 因此在有理数的基础上扩充把不是有理数的实数称为\textbf{无理数},这就形成了实数集. 

\paragraph{}
顺便,根据有理数的定义,证明为什么$\sqrt{2}$不是有理数,显然$\sqrt{2}$不是整数,令$\sqrt{2} = p/q$,其中$p,q$互素,等式变成$2q^2 = p^2$,也就是说$p$必然是偶数,令$p = 2m$,则$q^2 = 2m^2$,那个$q$也必然是偶数,这与$p, q$互素矛盾,故$\sqrt{2}$不是有理数. 虽然这个证明很简单,不过这里中间忽略了幂的操作的合法性以及为什么没有上确界,这里就不加证明了. 

\subsubsection{复数域}
\paragraph{}
\textbf{复数集}表示有序对$(x, y)$组成的集合,\textbf{复数域}定义了以下的加法和乘法
\begin{align*}
(x_1, y_1) + (x_2, y_2) &= (x_1 + x_2, y_1 + y_2) \\
(x_1, y_1) \cdot (x_2, y_2) &= (x_1 \cdot x_2 - y_1 \cdot y_2, x_1 \cdot y_2 + x_2 \cdot y_1) 
\end{align*}
其中的0元用$(0, 0)$表示,1元用$(1, 0)$表示,用符号$i := (0, 1)$,根据乘法, $i^2 = (-1, 0)$,一般将形如$(a, 0)$的复数等同于实数$a$,因此$i^2 = -1$.
\paragraph{}
\textbf{定义}\, 复数的共轭,设$z = (a, b)$,则它的共轭$\overline{z} = (a, -b)$,和实数一样,定义一个复数的绝对值,$|z| = \sqrt{z\cdot \overline{z}}$. 因此,如果$z = (a, b)$,那么$|z| = \sqrt{a^2 + b^2}$.
\paragraph{}
\textbf{定义}\, 对于复指数$e^z$,$z = x + iy$,则复指数为一个复数$e^z = e^x(\cos{y} + i\sin{y})$.

\paragraph{}
\textbf{定义}\, 设$z = (x, y) = x + iy$,是一个非零复数,则
$$
Log(z) = \log{|z|} + iarg(z)
$$


\paragraph{}
\textbf{定义}\, 设$z = (x, y) = x + iy$,是一个非零复数,存在
$$
x = |z|\cos{\theta}, y = |z|\sin{\theta}, -\pi < \theta <= \pi
$$
的实数$\theta$称之为$z$的\textbf{辐角主值}, 记为$\theta = arg(z)$,通常也使用$z = r \exp{i\theta}$, 其中$r = |z|, \theta = arg(z) + 2\pi n$,$n$为任意整数.
\paragraph{}
\textbf{定义}\, 设$z$为复数,则
\begin{align*}
\cos{z} = \frac{e^{iz} + e^{-iz}}{2} \\
\sin{z} = \frac{e^{iz} - e^{-iz}}{2i} 
\end{align*}

\subsection{可数集}
\paragraph{}
称集合里元素的个数称为集合的\textbf{基数}或者\textbf{势}, 记为$card(S)$. 如果一个集合和自然数集合等势,那么称这个集合为\textbf{可数集}. 即$card(S) = card(N)$,如果集合是有限的,那么可以称为\textbf{至多可数集}. 如果不是有限的可数集称为\textbf{无限可数集}. 通常为了方便如果$card(S) \leq card(N)$都称为是\textbf{可数集}. 通过可数集可以知道无限和无限之间也是有区别的.

\subsection{欧氏空间}

一个$n$元的有序组$(x_1, x_2, \cdots, x_n)$的集, 以及定义了加法和标量乘法,称这样的集合为一个$n$维的空间. 为了方便,用向量$\textbf{x}$来表示, 因此也可以称之为\textbf{向量空间}. $n$维空间的加法是这样定义的
$$
\textbf{x} + \textbf{y} = (x_1 + y_1, \cdots, x_n + y_n
$$
其标量乘法
$$
a\textbf{x} = (ax_1, \cdots, ax_n)
$$
它们的\textbf{内积},表示为
$$
\textbf{x} \cdot \textbf{y} = \sum_{i = 1}^n x_i y_i
$$
将$x$的\textbf{范数}定义为
$$
|\textbf{x}| = \sqrt{\textbf{x} \cdot \textbf{x}}
$$


