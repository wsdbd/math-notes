\section{随机变量和分布}

\subsection{随机变量}
\paragraph{}
定义在样本空间上的实值函数称之为\textbf{随机变量}


\subsubsection{分布函数}
\paragraph{}
对于随机变量$X$,定义函数$F$
$$
F(x) = P\{X \leq x \},-\infty < x < \infty
$$
称为$X$的\textbf{累积分布函数},也称为\textbf{分布函数}

\subsection{离散型随机变量}
\paragraph{}
如果随机变量$X$的可能取值是可数多个,那么这个随机变量称为\textbf{离散型}的. 定义$X$上的概率分布列为$p(x)$,那么
$$
\sum_{i=1}^\infty p(x_i) = 1
$$
离散型随机变量的分页函数F为
$$
F(a) = \sum_{x \leq a} p(x)
$$

\paragraph{}
例如,一枚硬币,抛两次,如果正反的概率都是$\frac{1}{2}$,若随机变量$X$表示,正面出现的次数,那么
$$
p(0)= \frac{1}{4}, p(1) = \frac{1}{2}, p(2) = \frac{1}{4}
$$
其分布函数为
$$
F(a) =  \begin{cases}
0 & \quad a < 0 \\
\frac{1}{4} & \quad 0 \leq a < 1 \\
\frac{3}{4} & \quad 1 \leq a < 2 \\
1 & \quad a \geq 2
\end{cases}
$$

\subsubsection{期望}
\paragraph{}
随机变量$X$的\textbf{期望} 记为$E[X]$,定义为
$$
E[X] = \sum x p(x)
$$
期望也可以称为随机变量$X$的\textbf{均值}

\subsubsection{期望的性质}
\paragraph{}
若有关于随机变量$X$的函数$g(x)$,那么这个函数的期望为
$$
E[g(X)] = \sum g(x) p(x)
$$
因此例如随机变量$X^2$的期望为
$$
E[X^2] = \sum x^2 p(x)
$$
例如
$$
E[X+1] = \sum (x+1) p(x)
$$
更通用的
$$
E[aX+b] = aE[X] + b
$$

	
\subsubsection{方差}
\paragraph{}
如果随机变量$X$的期望为$\mu$,那么$X$的\textbf{方差},记为$Var(X)$,并且
$$
Var(X) = E[(X - \mu)^2] = E[X^2] - \mu^2
$$
另外
$$
Var(aX+b) = a^2Var(X)
$$

\subsubsection{标准差}
\paragraph{}
另外定义方差的平方根称为\textbf{标准差}, 记为$SD(X)$,并且$SD(X) = \sqrt{Var(X)}$.

\paragraph{}
另外,一个常用的公式
$$
Var(aX + b) = a^2 Var(X)
$$

\subsubsection{中位数}
\paragraph{}
对于离散型随机变量,中位数指的是数据中一半的数据比它大,另一半比它小,如果处于中间的刚好有两个数,那么取它们的平均值

\subsection{伯努利随机变量和二项随机变量}
\paragraph{}
如果一次试验的结果为两个
$$
X = \begin{cases}
1 & \quad  \\
0 & \quad 
\end{cases}
$$
并且设
$$
p(0) = 1 - p 
$$
$$
p(1) = p
$$

\paragraph{}
如果重复n次这样的试验,如果随机变量$X$表示实验成功的次数,那么称$X$为参数是$(n, p)$的\textbf{二项随机变量}, 所以说伯努利随机变量是参数为$(1, p)$的二项随机变量,那么二项随机变量$(n, p)$的分布列为
$$
p(i) = { n\choose i} p^i (1-p)^{n-i} \quad i = 0, 1, \cdots, n
$$
一般将二项随机变量记为$X \sim b(n, p)$或者$X \sim B(n, p)$

\subsubsection{期望}
\paragraph{}
二项随机变量的期望可以计算出来
\begin{align*}
E[X] &= \sum_{i=0}^n i {n\choose i} p^i (1-p)^{n-i} \\
		&= \sum_{i=0}^n i \frac{n!}{i! (n-i)!} p^i (1-p)^{n-i} \\
		&= 0 + \sum_{i=1}^n  i \frac{n!}{i! (n-i)!} p^i (1-p)^{n-i} \\
		&= np \sum_{i=1}^n {(n-1)\choose (i-1)} p^{i-1} (1-p)^{n-1-(i-1)} \\
		&= np (1 + (1-p))^{n-1} \\
		&= np
\end{align*}

\paragraph{}
类似的可以计算出来$E[X^2]$
$$
E[X^2] = np((n-1)p +1) 
$$


\subsubsection{方差}
\paragraph{}
根据方差和期望的公式, 二项随机变量$X$的方差为
\begin{align*}
Var(X) &= E[X^2] - (E[X])^2 \\
		   &= np((n-1)p +1)  - n^2 p^2 \\
		   &= n^2 p^2 - np^2 + np - n^2 p^2 \\
		   &= np(1-p)
\end{align*}


\subsubsection{二项分布函数}
\paragraph{}
根据分布函数的定义,那么二项分布函数为
$$
P\{X \leq i\} = \sum_{k=0}^i {n\choose k} p^k (1-p)^{n-k}\quad i = 0, 1, \cdots, n
$$

\subsection{泊松随机变量}
\paragraph{}
如果一个二项随机变量的$n$很大,并且$p$很小,记$\lambda = np$,那么
\begin{align*}
p(i) &= {n\choose i} p^i (1-p)^{n-i} \\
	  &= {n\choose i} (\frac{\lambda}{n})^i (1 - \frac{\lambda}{n})^{n-i} \\
	  &= \frac{n(n-1)(n-2)\cdots (n-i+1)}{i!} \frac{\lambda^i}{n^i} \frac{(1-\lambda/n)^n}{(1- \lambda/n)^i} \\
	  &= \frac{n(n-1)(n-2)\cdots (n-i+1)}{n^i} \frac{\lambda^i}{i!} \frac{(1-\lambda/n)^n}{(1-\lambda/n)^i} 
\end{align*}
如果$n$很大,$p$很小,那么$(1 - \lambda/n)^i \approx 1$,以及 $\frac{n(n-1)(n-2)\cdots (n-i+1)}{n^i} \approx 1$,那么
$$
p(i) \approx   \frac{\lambda^i}{i!} (1-\lambda/n)^n
$$
令$m = - \frac{n}{\lambda}$,那么
$$
(1-\lambda/n)^n = (1+1/m)^{-m\lambda} = ((1+1/m)^m)^{-\lambda} \approx e^{-\lambda}
$$
参见\textbf{7.1.6}, 因此
$$
p(i) \approx e^{-\lambda}  \frac{\lambda^i}{i!}
$$
这不是一个非常严格的证明,只是说明二项随机变量和以下泊松随机变量的关系,也即是如果$n \to \infty$, $n$为自然数(有时候为了方便会假设自然数是从0开始的),对于随机变量$X$,$X$的取值为$0, 1, 2\cdots$,那么对于某个$\lambda > 0$,其分布列为
$$
p(i) = P{X = i} = e^{-\lambda} \frac{\lambda^i}{i!}\quad i = 0, 1, 2 \cdots 
$$
则称该随机变量为\textbf{泊松随机变量},并且可知
$$
\sum_{i=0}^\infty p(i) = e^{-\lambda} \sum_{i=0}^\infty \frac{\lambda^i}{i!} = e^{-\lambda} e^\lambda = 1
$$
其中
$$
e^\lambda = \sum_{i=0}^\infty \frac{\lambda^i}{i!}
$$
参见\textbf{7.2.5}或\textbf{9.4},一般将泊松随机变量记为$X \sim \pi (\lambda)$,或者$X \sim P(\lambda)$

\subsubsection{期望}
\paragraph{}
泊松随机变量的期望为
\begin{align*}
E(X) &= \sum_{i=0}^\infty \frac{i e^{-\lambda} \lambda^i }{i!}  \\
	    &= \lambda  e^{-\lambda} \sum_{i=1}^\infty \frac{\lambda^{i-1}}{i!} \\
	    &=  \lambda  e^{-\lambda} e^\lambda \\
	    &= \lambda
\end{align*}
类似的可以计算出来
$$
E(X^2) = \lambda(\lambda + 1)
$$

\subsubsection{方差}
\paragraph{}
由此,泊松随机变量的方差为
$$
Var(X) = E[X^2] - (E[X])^2 = \lambda^2 + \lambda - \lambda^2 = \lambda
$$


\subsection{几何随机变量}
\paragraph{}
考虑某个试验,成功的概率为$p$,那么重复这个实验直到成功,这样的随机变量$X$称为参数为$p$的\textbf{几何随机变量},一般记为$X \sim G(p)$,那么它的分布列为
$$
p(i) = (1-p)^{n-1}p
$$

\subsubsection{期望}
它的期望为
$$
E(X) = 1/p
$$
并且
$$
E(X^2) = \frac{2-p}{p^2}
$$

\subsubsection{方差}
它的方差为
$$
Var(X) = \frac{2-p}{p^2} - \frac{1}{p^2} = \frac{1-p}{p}
$$

\subsection{负二项随机变量}
\paragraph{}
考虑某个试验,成功的概率为$p$,那么重复这个实验直到成功$r$次,这样的随机变量$X$称为参数为$r, p$的\textbf{负二项随机变量},一般记为$X \sim NB(r, p)$,那么它的分布列为
$$
p(n) = {(n-1)\choose (r-1)} p^r (1-p)(n-r)\quad n \geq r
$$
所以说,如果$r=1$,那么就是几何随机变量

\subsubsection{期望}
它的期望为
$$
E(X) = r/p
$$
并且
$$
E(X^2) =\frac{r}{p}(\frac{r+1}{p} - 1)
$$

\subsubsection{方差}
它的方差为
$$
Var(X) = \frac{r(1-p)}{p^2}
$$


\subsection{超几何随机变量}
\paragraph{}
考虑一个盒子里有$N$个球,其中$m$个白球,$N-m$个黑球,从中随机取出$n$个球,令$X$表示取出来的白球数量,那么随机变量$X$的分布列为
$$
p(i) = \frac{{m\choose i} {(N-m)\choose (n-i)}}{{N \choose n}} \quad i = 0, 1, \cdots, n
$$
这样的随机变量称之为\textbf{超几何随机变量},一般记为$X \sim H(n, m, N)$
\paragraph{}
若$n=1$,那么$X$就是伯努利随机变量,如果$N$和$m$远大于$n$,则可以看成是二项随机变量

\subsubsection{期望}
\paragraph{}
$$
E(X) = \frac{nm}{N}
$$
并且
$$
E(X^2) = \frac{nm}{N} (\frac{(n-1)(m-1)}{N-1} + 1)
$$

\subsubsection{方差}
$$
Var(X) = np(1-p)(1 - \frac{n-1}{N-1})
$$


\subsection{连续型随机变量}
\paragraph{}
如果随机变量的可能取值是不可数的,那么称随机变量$X$为\textbf{连续型随机变量}

\subsubsection{概率密度}
假设一个可测集合B中,如果连续型随机变量$X$满足
$$
P\{X \in B\} = \int_B f(x) \mathrm{d} x
$$
称这个函数为\textbf{概率密度函数},简称\textbf{概率密度}.
并且有
$$
P\{X \in (-\infty, +\infty)\} = \int_{-\infty}^{+\infty} f(x) \mathrm{d} x = 1
$$

\subsubsection{期望}
\paragraph{}
类似的,连续型随机变量的期望为
$$
E[X] = \int_{-\infty}^{\infty} xf(x) \mathrm{d} x
$$
另外,如果有随机变量的函数$g(x)$,那么这个函数的期望为
$$
E[g(x)] = \int_{-\infty}^\infty g(x) f(x) \mathrm{d} x
$$

\paragraph{}
类似的
$$
E[ax+b] = aE[X] + b
$$

\subsubsection{方差}
\paragraph{}
同离散型随机变量一样,连续型随机变量的方差为
$$
Var(X) = E[X^2] - (E[X])^2
$$

\subsection{均匀随机变量}
\paragraph{}
如果一个随机变量$X$的概率密度为
$$
f(x) = \begin{cases}
\frac{1}{b-a} & \quad a \leq  x \leq b \\
0 &\quad  other
\end{cases}
$$
则称$X$在$[a, b]$区间上\textbf{均匀分布},一般记为$X \sim U[a, b]$

\subsubsection{分布函数}
\paragraph{}
它的分布函数为
$$
F(x) = \begin{cases}
0 & \quad x <  a \\
\frac{x-a}{b-a} & \quad a  \leq x  \leq b \\
1 & \quad x > b
\end{cases}
$$

\subsubsection{期望}
\paragraph{}
它的期望为
$$
E[X] = \frac{a+b}{2}
$$

\subsubsection{方差}
\paragraph{}
它的方差为
$$
Var(X) = \frac{(b-a)^2}{12}
$$

\subsection{正态随机变量}


\subsection{指数随机变量}


\subsection{矩}


\subsection{协方差}

\subsection{条件期望}
