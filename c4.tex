\section{线性映射}

\subsection{线性映射}
\paragraph{}
有向量空间$V$和$W$, 将一个映射$L: V \to W$,它满足
\begin{enumerate}
\item $L(\textbf{v}_1 + \textbf{v}_2) = L(\textbf{v}_1) + L(\textbf{v}_2)$
\item $L(\alpha \textbf{v}_1) = \alpha L(\textbf{v}_1)$
\item $L(\alpha \textbf{v}_1 + \beta \textbf{v}_2) = \alpha L(\textbf{v}_1) + \beta L(\textbf{v}_2)$
\end{enumerate}
这样的映射,将为\textbf{线性映射},如果$L: V \to V$, 也就是一种映射把向量空间$V$映射到自身,称为\textbf{线性算子}. 与一般的函数类似,不过这种映射的参数为向量. 

\paragraph{}
在(3.3.4小节)中,坐标变换就是一种线性映射,因为是映射到同一个空间,所以说也称这为线性算子,其中提到了\textbf{基变换矩阵}.  这里推广出来,同样的可以把线性映射表示成一个矩阵, 只不过基变换矩阵是方阵,这里的矩阵就不受这个限制.  例如
$$
A = \begin{bmatrix}
1 & 0 \\
0 & 1 \\
1 & 1 
\end{bmatrix}
$$ 
如果$L(\textbf{x}) = A\textbf{x}$, 如果$\textbf{x}$是空间$R^2$的向量,假设为$(2, 1)^T$那么
$$
L(\textbf{x}) = \begin{bmatrix}
1 & 0 \\
0 & 1 \\
1 & 1 
\end{bmatrix} \cdot \begin{bmatrix}
2 \\
1 
\end{bmatrix} =  \begin{bmatrix}
2 \\
1  \\
3
\end{bmatrix} 
$$
也就是说,这样的一种映射,把$R^2$映射到$R^3$中的一个子空间. 

\subsection{核与象}
\paragraph{}
在(3.2节)中曾经提到过\textbf{零子空间}$N(A)$, 如果把映射表示成矩阵的话,那么把这个零子空间称为线性变换$L$的\textbf{核}. 有线性空间$V$, 如果有$S$是$V$的子空间,$S$的\textbf{象}记为$L(S)$, 整个空间$V$的像称为$L$的\textbf{值域} .

\subsection{特征值}
\paragraph{}
在向量空间$V$中,考虑一个线性算子$L(\textbf{x})$,用矩阵表示为$A\textbf{x}$, 对于一个标量$\lambda$,如果向量空间$V$中有一个非零的向量\textbf{x}, 使得$L(\textbf{x}) = \lambda \textbf{x}$,即$A(\textbf{x}) = \lambda \textbf{x}$,称$\lambda$为矩阵$A$或称为线性变换$L$的\textbf{特征值}, 这个非零的向量\textbf{x}称之为$\lambda$的\textbf{特征向量}.

\paragraph{}
在(3.3.4小节)中提到,一个线性算子(向量空间$V$映射到自身的一种映射)是一个$n\times n$的方阵, 为了求得一种线性映射的所有的特征值和特征向量,令$A$是映射矩阵,
$A \textbf{x} = \lambda \textbf{x}$,那么就意味着
$$
(A - \lambda I) \textbf{x} = 0
$$
有非零解.  看成一个齐次线性方程组,有非零解那么,形成的矩阵的特征值必为0. 也就是列向量线性无关的话只有零解,它们需要线性相关.  因此矩阵$A$有非零的特征值或者说有特征向量的充分必要的条件是
$$
det(A - \lambda I) = 0
$$
举个例子来说
$$
A = \begin{bmatrix}
1 & 0 \\
1 & 1 
\end{bmatrix}
$$
代放上面的方程,
$$
det \begin{bmatrix}
1 - \lambda & 0 \\
1 & 1 - \lambda
\end{bmatrix} = 0
$$
因此
$$
\lambda^2 - 2 \lambda + 1 = 0
$$
所以$\lambda = 1$. 只有一个特征值,它的特征向量形如$(0, a)^T$,$a$为任意标量.





